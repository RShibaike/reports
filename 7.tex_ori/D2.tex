\section{はじめに}

現在の日本で女性が働くことを望んだとき、男性と同じスタートラインに立てるだろうか。答えは否である。結婚や妊娠や出産といったライフイベントは、女性が自らのキャリアを考える上で避けては通れない``課題''となっている。また、自身のキャリアが配偶者やパートナーのキャリアの両立が転勤などで難しくなったらどんな選択をすればよいのだろうか。女性が職場に進出して以降、多くの女性がキャリアとライフにまつわるこのような悩みに直面してきた。採用での門前払いや結婚退職制や男女別定年制などの会社制度上の明らかな差別はなくなった21世紀においても、日本がいまだに女性の潜在的な能力を完全に発揮できる環境ではなく、原因の一端は高度経済成長期に作り上げられた家事・育児は女性が行うべきタスクという常識、そしてそれを前提とした社会(専業主婦前提社会)があるというのが私たちの主張である。

本稿ではまず、「男性は正社員として、女性は家事や育児を一手に引き受け(そうでないときは短期労働者として働く)る」という構造が出来上がった経緯を辿る。次に労働分野における女性の権利闘争を反映した1960年以降の裁判例を参照し、その成果が刻み込まれた 「雇用の分野における男女の均等な機会及び待遇の確保等に関する法律」(以下、男女雇用機会均等法と表記する)と「育児休業、介護休業等育児又は家族介護を行う労働者の福祉に関する法」(以下、育児介護休業法と表記する)について概観する。第3章では、女性の社会進出が進まない原因を社会学的見地から説明するとともに、平成以降の日本の現状について政府出典のデータなどを活用しながらより詳しい分析を行う。第4章で「働き方改革」という直近の国の政策が、女性の働き方と男女格差の解消にどのように影響を与えているのかを分析する。

\section{男女雇用機会均等法と女性労働者の権利}

\subsection{男女雇用機会均等法以前の日本}

\subsubsection{アメリカ占領による改革}

戦前から戦中にかけて、女性は低賃金で雇える労働者、また男手がいない環境での働き手として企業の利潤追求に貢献してきた。しかし、その基盤となった財閥体制はアメリカの占領の下で解体され、日本の産業基盤それ自体に大変革が起きる。それ以上に女性の法的地位そのものが占領下で大きく改善されたのがこの時期の大きな特徴である。1945年10月に占領軍の指令によって男女平等な参政権が認められ、よく1946年4月の選挙では39名の女性国会議員が誕生した。大日本帝国憲法に代わる日本国憲法の制定は、それまでのイエを中心とした家族制度を{法的に}否定したものである。\footnote{毛利透他『憲法Ⅱ人権{[}第2版{]}』(有斐閣、2017年)92-95頁。} GHQによる女性に関する民主化は1945年から1949年の間にほぼ完了したと言えるだろう。特に労働面に関するものとしては、1947年に公布された労働基準法が第4条で「男女同一労働同一賃金」の原則を導入した。同年に成立した職業安定法では、職業紹介・職業指導における男女差別を禁止した。しかし、このような女性解放の流れは急激な国際情勢の変化によって妨げられる。米国とソ連の間の国際対立はアメリカの対日政策を早急に転換させ、女性解放の機運は盛り上がることなく「新家族主義」というイデオロギーが女性の職業活動に立ちはだかる新たな壁になったのである。(『日本型企業社会と女性労働』藤井治枝p51~54)

\subsubsection{女性の職場進出と権利闘争}

\paragraph{高度経済成長期(1955~1973)}

この間、女性雇用労働者の量的増加には著しいものがあった。この背景には、機械設備の近代化に伴う単純労働分野の拡大、流通部門への機械導入などが女性に新たな職域を開拓したことがある。しかし、当時の女性雇用には戦前並みの男女差別が残っていたことを示す一つの特色、「雇用の短期化」があったと藤井は述べる。設備の近代化により速度の上がったベルトコンベア・システムでは、肉体的・精神的疲労が大きく、視力や注意力の減退が労働効率を落とすため、企業は労働力の速やかな新陳代謝を求めたのである。そのうえ、既婚女性は企業での労働の他に家事や育児を負担し、一層の能率低下が懸念されるため、企業は若年未婚女性の就労の実を原則として求めた。結局、高度経済成長期の前期\footnote{いわゆる「神武景気」から「岩戸景気」にかけての約11年間(1954年12月~1961年12月)を指す。}において、女性の雇用は未婚である間かつ短期間のみにとどまることになった。

主婦が労働力として立ち現れてくるのは高度経済成長期の後期においてである。1961年に設備投資の停滞と国内需要の行き詰まりが不況を深刻化させると、企業は一時帰休や大量の配置転換によってこれを乗り越えた。これらの施策は若年未熟労働者への需要を増大させたが、当時すでに若年労働力不足が深刻化していたため、これに代わる労働力として主婦の活用が始まったのである。1964年ごろに電器産業部門への主婦パートの導入が始まり、やがてほかの産業部門にも拡大されていった。この流れは長きにわたって若年且つ未婚であることを女性が働くうえでの条件としてきた日本社会を変化させ、女性労働者数を増加させた。しかし、女性は未婚若年グループと、出産・子育て後の中高年主婦パートに大きく分けられ、結婚・出産による勤続中断はそのまま残った。

高度経済成長期の確立された日本的経営は世界的に注目されたが、本稿では特徴的なシステムのうち女性に関する制度を三つ、その有効性が問われた裁判例と共に紹介する。どれも現在では法律で明示的に禁止されているが、当時では標準的と考えられ、労働者・使用者双方の自主的な話し合いの下慣行として実施されていたり、雇用契約として明記されていたりした制度である。

\paragraph{結婚退職制}

男女雇用機会均等法以前には、女性労働者に「結婚し、あるいは〇歳に達したら退職する\footnote{〇には企業が設定する任意の年齢が入る。}」と約束させる会社が少なからず存在した。そのようないわゆる結婚退職制度が民法90条の公の秩序に違反し、効力を生じないとの判決が出たのは1966年のことである。住友セメント雇用関係確認等請求事件\footnote{東京地裁1966年12月20日判時467号26頁。}では、「結婚又は満35歳に達したときは退職する」ことを労働契約の内容とすることに合意して雇用され、結婚後に解雇された原告が雇用契約上の権利の確認等を請求した事件である。原告の請求は東京地方裁判所によって認容された。またこの事件は日本初の女性雇用差別に関する裁判例であり\footnote{三成美保他『ジェンダー法学入門{[}第3版{]}』(法律文化社、2019年)228頁。}、男女雇用機会均等法が制定され結婚退職制が禁止された現在、判例としての価値が薄れたが、女性の権利獲得の闘いを示すうえでの重要性はいまだに健在だろう。

\paragraph{男女別定年制}

男女別定年制とは、男性労働者と女性労働者で異なる定年を設ける制度である。伊豆シャボテン公園事件\footnote{東京高裁1975年2月26日判時770号18頁。}では、伊豆シャボテン公園の観光レジャー企業は男性従業員について57歳、女性従業員について47歳の定年制を就業規則で定めており、この退職規定に従って退職扱いにされた女性従業員らが、退職規定は憲法14条、労基法3条・4条に違反し、民法90条により無効と主張して、従業員足る地位保全、賃金支払いにつき仮処分を申請した。一審の静岡地裁沼津支部は会社側主張の合理化理由をすべて退けて女性従業員の申請を認めたため、会社側が控訴した。なお、本件以前にも男女別定年制による裁判例は存在しており、前掲の結婚退職制はその派生と捉えられる他、本件のように女性の定年を男性よりも若年に設定している事案としては、盛岡地判1971年3月18日判時626号や東京地判1973年3月23日判時698号などがある。男女別定年制の有効性を争った裁判でこれを採用する会社側の主張は似通っており、本事案における企業の男女別定年制の合理化理由もそこから大きく外れてはいない(①企業合理化の必要性、②組合の同意がある、③女子向きの職場と男子向きの職場は完全に区分され、そこには若い女性が必要、④女子従業員は管理的能力や専門的業務習得能力を欠き配転不能、⑤女子は40代後半に肉体的更年期を迎え労働能力が低下する、⑦女子労働は家計補助的労働に過ぎず40代後半まで労働する者が少ない、⑧他の企業でも一般的に男女別の定年制を定めている)。本事案の判決でこのような主張に対する判決の枠組みはほぼ出尽くしていると言える\footnote{中山勲「私的団体の性別による差別と憲法」別冊ジュリスト 憲法判例百選Ⅰ68号。}。年齢差がわずかな場合に会社側の合理化理由を認めた判例もある\footnote{日産自動車仮処分事件(東京地判1971年4月8日判時)。}が、判例の大多数はそれらの主張を認めない。本事案の判決もその流れに則ったものであった。この判決について会社側は上告したが、棄却されて判決が確定した。後の日産自動車男女別定年制事件では、本事案の10歳差よりも差が小さい、男女間での定年の5年の差が問題となった。最高裁第三小法廷は「少なくとも60歳前後までは、男女とも通常の職務であれば企業経営上要求される職務遂行能力に欠けるところはな(い)」と判断し、専ら女子であることを理由とする性別による不合理な差別として民法90条によって無効と判断している。このように、男女別定年制は判例で無効とされた。

\paragraph{賃金差別}

秋田相互銀行賃金差別事件\footnote{秋田地裁1975年4月10日労民集26巻2号388頁。}は、女性に対する賃金差別に関する先駆的な判例である。当該銀行は男女別に適用される2つの賃金体系を設けて女子職員全体に低賃金の体系を適用したが、そのような適用は労働基準法4条・13条に基づいて無効であるとして、原告の女性行員らの請求が認容された。

\paragraph{小活}

第二次世界大戦後、女性の法的地位は大きく改善し、また労働市場への進出も著しかった。しかし、女性は市場において若年・未婚である層と育児後の主婦層に大別され、男性と比べて短期的に回転可能な労働力として企業に雇用されていたことが見えてくる。結婚退職制や男女別定年制、男女別賃金などの雇用慣行に加え、国の政策や税制\footnote{配偶者控除や年金、また育児介護を主婦の役割として自助努力を強調する政策など。}も女性が周縁的な労働者でいることを前提として組み立てられていたのである。しかし、差別的な雇用慣行は当事者により問題視され、女子労働者による権利闘争が展開され、上記の裁判例のような目覚ましい成果を上げるに至った。しかし、男は外、女は内(家)でそれぞれの役割を果たす、という性別役割分業は、企業構造、労働市場構造、政府の政策、人々の意識に至るまで深く入り込んでしまったのである。中野はこれを専業主婦前提の社会と呼び、図1でそれを説明した(2019、9頁)。専業主婦前提社会では、多くの男性が正社員になり長時間労働をする半面、女性に全面的に家庭を任せ、労働市場では制約のある人材として排除されたり、周縁に追いやられる=パートなどの形態によって短期的な労働力として使われたりする。その結果、男女間の賃金格差は縮まらず、ますます女性の主婦化を合理化・正当化するのである。

それでは国際的な女性の権利への注目の高まりと国内での女性の権利闘争の結果、制定された男女雇用機会均等法はどのような法律だったのか、このような構造を変える力を持っていた/るのかを見ていく。

図 1 専業主婦前提社会の循環構造

\includegraphics[width=5.18506in,height=2.91667in]{media/image1.png}

\subsection{女性労働者の権利に関する法律}

\subsubsection{男女雇用機会均等法}

男女雇用機会均等法は1985年に「勤労夫人福祉法」を改正する形で成立した。女性の家庭での役割を強調していた1条は改められ、法の下の平等を保障する日本国憲法を根拠として労働分野における女性への差別を改善することが目的とされたのは、大きな変化と言える。上述した住友セメント事件判決\footnote{2.1.2.2.を参照せよ。}等の数々の裁判闘争の結果が刻み込まれた結果、解雇・退職については女性差別禁止を明確に義務付けている。しかし、他の雇用における様々な場面においては「差別しないように努力をしなければならない」という努力義務しか規定しておらず、その実効性には疑問が残された。男女雇用機会均等法は女性の社会進出をより推進する社会潮流の後押しを受け、複数の改正を経て充実していくことになる。また、労働という市場の流動性が関わる事項であるから、国が禁止事項を定め、実際に取り組むのは企業などの事業主であるのが特徴である。

1997年改正は、1990年代半ばにバブルが弾けた結果、就職活動は氷河期に突入した。女子学生たちにとってはいっそう険しい状況であった。そのような状況を背景として、募集・採用にはじまる雇用のあらゆる段階での差別の禁止が明記されるに至った。

2007年改正では、7条で間接差別が、9条で婚姻、妊娠、出産等を理由とする不利益的取扱いの禁止が取り入れられた。間接差別とは、一見中立的な基準による区別であっても、特定の属性の集団に不利益にはたらく規定や基準、慣行について表す言葉である。1971年にアメリカの最高裁判所が下した画期的な判決をきっかけとして、欧州で深められた概念である。ただし、男女雇用機会均等法は「性に中立的な基準であって一方の性に不利益にはたらくようなもの」であって、厚生労働省令で定める三つの基準のみを間接差別として禁止している\footnote{厚生労働省令にないものであっても裁判で争うことは可能である。}。妊娠に関してはその後の2016年改正でいわゆるマタニティハラスメントの防止義務が追加された(11条の2)。

\subsubsection{育児介護休業法}

民間企業における育児休業は1972年施行の勤労婦人福祉法において「育児休業等育児に関する便宜の供与」が事業主の努力義務として規定されたことが始まりである(11条)\footnote{制定当初の勤労婦人福祉法は以下のURLを参照せよ。

  \href{http://www.shugiin.go.jp/internet/itdb_housei.nsf/html/houritsu/06819720701113.htm}{{http://www.shugiin.go.jp/internet/itdb\_housei.nsf/html/houritsu/06819720701113.htm}}。}。1986年に勤労夫人福祉法が「男女雇用機会均等法」に改められた際にも、努力義務として同法に盛り込まれた。その後、女性の職場進出や核家族化の進行による過程機能の変化、さらに少子化に伴う労働力不足への懸念を背景に、「育児休業法」として単独の法律となったのが1992年のことである。目的としては1条で「\ldots{}子の養育\ldots{}を行う労働者等の雇用の継続お補備再就職の促進を図り、もってこれらの者の職業生活と家庭生活との両立に寄与することを通じて、これらの者の福祉の増進を図り、合わせて経済及び社会の発展に資することを目的とする」

現在の育児介護休業法は幾度の改正を経て、対象者・補償の拡充、取得の推進を図ってきた。当初は女性且つ無期契約労働者のみが対象だったが、現在は男性と有期契約労働者も対象に入っている。また、休業中の給付金は賃金の25%から50%に引き上げられている。このように日本の育児休業制度は世界レベルで考えても充実している。しかし、実際の取得率は高いとは言えず、特に男性については著しく低く、女性についても出産前に就いていた職に復帰した割合は低い。

\subsubsection{憲法}

\paragraph{憲法14条1項}

日本国憲法14条1項は、「すべて国民は、法の下に平等であつて、人種、信条、性別、社会的身分又は門地により、政治的、経済的、又は社会的関係において、差別されない」と規定する。女子労働者による権利闘争において、この条文が女性差別を訴える根拠となってきた。判例・通説ではここでいう平等とは、いかなる場合においても各人を絶対的に等しく扱うという絶対的平等の意味ではなく、「等しいものは等しく、等しからざる者は等しからざるように」扱うという相対的平等を意味するものと解してきた。ゆえに、合理的な理由によって異なる取扱いは許されるとされるが、何が合理的な区別で何が不合理な差別になるのかという基準を設定するのは容易ではない。今の価値判断からすると多くの人が眉を顰めるだろう男女別定年制が合理的とされた判例があることからも、それは明らかである。性差別について憲法14条の相対的平等論の下で論じるのは限界があるとする研究者も存在する(辻村、2010)。さらに、憲法14条1項の平等の意味について考えるには、それが形式的平等であるのか実質的平等であるのかという議論も重要である。一般には形式的平等は、法律上の均一的取扱いを意味し、事実上の違いがある場合でも同等に扱うことを求めるのに対して、実質的平等は劣位のものを合理的な範囲内で有利に扱うなどして、事実上・実質上の均等を求めるものである。通説的見解は、14条1項の要請は、裁判規範においては形式的平等にとどまり、実質的平等の実現は立法に委ねられている、と解されてきた。しかし、近年は14条1項を形式的平等の実を保障していると解するのではなく、実質的平等保障も含まれていると解する見方も有力になってきた。とはいえ、実質的平等保障が裁判規範としてどこまで認められうるかは不明確であり、また積極的格差是正措置(アファーマティブアクションないしポジティブアクション)との関係も含めて、更なる研究が待たれるところである。

\paragraph{憲法27条1項}

日本国憲法27条1項は、「すべて国民は、勤労の権利を有し、義務を負ふ」と規定する。一般に、労働権又は労働三権と区別して勤労権と呼ばれるこの権利の理解としては大きく二つに分かれる。一つは、勤労権を自由権と解し、憲法が保障しているのは国民が自己の欲する労働に従事することを国家に妨げられない自由であるとする。この立場では国家による国民への労働の強制を禁止する意味を持つ。もう一つは、勤労権を社会権の一種と解し、資本主義経済体制を前提とした上で、①私企業への就職の機会が得られるような法制度を政府に対して要請し、②就労機会を求めて得られない国民に対しては、雇用保険制度を整備するなどの適切な措置を講ずることを要求できる権利とする説明である。少数ではあるものの、後者の説明からさらに進んで、国又は地方公共団体に対し、就労を請求する権利と解し、また国又は地方公共団体がこれに対する義務を履行しない場合には相当な生活費の支給を請求することができるとする説もある\footnote{オーストリアの法律学者アントン・メンガー(Anton
  Menger 1841-1906)が著名。}。日本国憲法は憲法18条で奴隷的拘束並びに苦役を禁止し、22条で職業選択の自由を保障しているため、単なる自由権としての勤労権を更に制定する理由はなく、社会権として制定されたと考えるのが自然である。もっとも上述のように27条1項から直接政府に対して雇用の機会の給付などを請求できる具体的権利説を提唱する説はほとんどなく、学説の多くは法的意味を持たないとするプログラム規定説又は法律によってはじめて具体的な権利となるとする抽象的権利説をとる(毛利他2017p370,371)。

勤労権が国又は地方公共団体に対する国民の就労請求権を認めるものではないとしても、片岡は勤労権の意義が次の三点に求められるという(2007,p29)。第一に、勤労権の保障は。第二に、勤労権実現のために制定された法律\footnote{片岡は雇用対策法や職業安定法、雇用保険法を例として挙げている。}がある場合、国民は国に対して訴えによってその履行を求めることができ、さらに国家機関がこれらの法律を勤労権の趣旨に反して解釈・運用する場合は、それによって生じた損害の賠償請求その他の救済が認められる。第三に、勤労権は正当な理由に基づかない、もしくは権利濫用に該当するような解雇の効力を失わしめる法律上の根拠となる。本稿では第二の意義に注目したい。なぜなら、男女雇用機会均等法や育児介護休業法をはじめとして現在は女性の障害をなくすための法律が多く制定されている。しかし、それでもなお、上述したような男女間での差があるのは、例えば育児休暇を自らの権利をとして捉えていないからではないだろうか。

\paragraph{日本人の権利意識}

川島は「日本人の法意識」で、そもそも権利という言葉は江戸時代までの日本語には存在しておらず、明治期に西欧の書物の翻訳で「権利」と訳されるようになったと述べている(1976年、16頁)。もちろん江戸時代においても雇用関係は存在していたが、使用者は被用者に対してその労働を「請求する権利」をもっているというふうには考えられていなかったのだと言う。これに対応するように、被用者は賃金を「請求する権利」をもっているとは考えておらず、「はたらかせて頂いて」「お給金を頂く」という意識だったのである。このような意識は江戸時代以降、日本の近代化が進み資本主義企業が登場してからも残り続けたのである。制度上は使用者と被用者の関係は労働契約によって成り立つものとなったが、当事者の意識の上では使用者が権力を持ち、被用者は「はたらかせて頂き」、使用者からの恩恵として「給与を頂く」という構図が残っていた。一度雇い入れたら原則的に被用者の定年まで雇用を続ける終身雇用制や福利施設の充実は、このような旧来の家父長的労働関係が維持されていることの表れである。

川島は「権利」と、西洋においては権利と表裏一体と考えられている「法」をAとBの二人の個人を用いて次のように説明する。

\begin{quote}
すなわち、法は、Bが或る行為をする義務を負っていることについてAが持っている利益を、AとBとのあいだの事実上の力の強弱にかかわりなく、Bに対し、一定の(Bが義務付けられている)行為を要求することが、社会的に是認されることになる。AとBとのあいだの関係を、そのようなものとして是認\ldots{}する場合に、このAの地位が「権利」と呼ばれるのである。

 \ldots{}AとBの事実上の力の強弱は、Bに対するAの要求の「正当性」の評価にあたって「考慮」される(むしろ、事実上の力の劣っているものを保護するという立場から)が、事実上の力の優位が正当性を決定することはない。したがって、その意味において、「権利」の平面においてはAとBは平等者として取扱われる、と言うことができる。(1967年、24頁)
\end{quote}

このように現代法は、すべての個人が「法」の平面では平等、すなわち事実上の権力関係にかかわらず平等であるという基本原則に立脚して構成され、また適用される。しかし、まさにこの点において、日本における人々の意識と現代法のシステムにずれがあった、というのが川島の主張である。西洋では人が自分の権利を擁護することは正しいこと足されるのに対して、日本では自己中心的な・平和を乱す・不当に政治権力の救済を求める・行為として非難されるのである。

川島が指摘した現代法システムと人々の意識のずれは、この書籍が出版されてから40年以上がたった今でも存在しているように思える。被用者としての労働者と使用者間の事実上の力関係は、当然使用者のほうが強い。それでも被用者が法賃金や休暇、労働災害補償を使用者に対等な関係で請求できるのが本来的な「権利」である。果たして、現代を生きる我々は自身が有する「権利」をそのように捉えることができているのだろうか。

\section{労働分野で見る女性の抱える問題とその解決策}

\subsection{性的役割分業に基づく価値観}

\subsubsection{性的役割分業}

農業社会では家族で畑仕事などの家業を行い、女性の行う仕事も炊事・掃除だけでなく、針仕事や食品加工などの収入につながるものもあった。しかし、工業化が進むと、夫が外で働き、妻は専業主婦として料理や洗濯など収入にはつながらない家事を行うようになった。夫が養い、妻が家を守る性的役割分業が定着し、この仕組みは社会や個人の意識だけではなく、ステレオタイプ的な価値観を形成していった。

\subsubsection{ステレオタイプ}

ステレオタイプとは、ある社会のメンバーに関する知識で、その内容が社会に共有されているもののことである。男性についてのステレオタイプは自信がある・独立している・冒険をしたがる・支配的である・強さを求める・競争をするなど、一人の独立した人間として個人が目指すべき特性が多く挙げられる。一方、女性のステレオタイプは配慮をする・相互依存になりやすい・温かさがある・養育に長ける・従属性がある・協力できるなど、他者とのかかわりの中で個人が目指すべき特性が多く挙げられる。決断を求められる仕事柄から、有力なリーダーの特性だと一般的に考えられている要素は男性のタイプであることが多い。\footnote{大沢真知子編『なぜ女性管理職は少ないのか
  女性の昇進を考える』(青弓社、2019年)34-44頁〔坂田桐子〕。}

\subsubsection{差別主義}

性的役割分業やステレオタイプから派生する、女性に対する差別的な考え方にも様々な種類がある。統計的差別主義は、出産や妊娠という生物学的な特徴からそれらによる仕事等への不参加を理由に女性に期待しないことである。これは、女性であることを理由に入試得点を減点したり、昇進や働き方を区別するコース別人事制度を行ったりすることにつながる。敵対的差別主義は、男性のライバルとして女性を捉えることで、女性を優遇する措置を直ちに男性にとっての逆差別だとする考え方である。好意的差別主義は、男性の下にあるものとして女性を捉えることである。これは、女性に対して女子力・女性らしさを求め、保護的父性主義で女性を保護すべきものとするため、一見思いやりと受け取られやすい。だが、女性を男性と対等な存在とせず、女性の活躍を推進する制度を取ろうとするときには、実際には十分な実力がある場合でも実力不足だと考えやすくなる。\footnote{大沢・前掲注(1)45-52頁〔坂田桐子〕。}

\subsection{女性管理職が少ない問題について}

\subsubsection{現状分析}

管理職に占める割合、昇進についての男女差に着目すると、いわゆるガラスの天井の問題といえる。能力を測るテストの結果、経営や管理能力には男女の差異はない。また、学生までのリーダー経験も男女での差異はない。だが、社会におけるリーダー経験は男性のほうが豊富になっており、労働環境では性別という要素の影響があると考えられる。\footnote{筒井淳也『仕事と家族
  なぜ日本は働きづらく、産みにくいのか』(中公新書、2016年)。}

年々管理職に占める女性の割合は増加していて、係長級では平成元年には5%だったが、今では20%にまで上昇している。しかし、女性の社会進出を示す指針とされている30%の目標も達成してない。また、役職上がるごとに女性の占める割合は少ない。女性の社会進出は進んでいるものの、依然として日本の女性の職場での地位は低いままだといえる。\footnote{男女共同参画白書
  令和2年度版Ⅰ-2-11図}

\subsubsection{女性の反応}

一つの反応が男性化することである。男性化とは、女性がステレオタイプを肯定することになることへの不安やもどかしさから、自己の行動を本来の自分と異なる、男性的な働き方に順応するためのものに変えることである。これによって、女性のリーダーは増えるものの、リーダー像の男性化はそのままになる。

二つ目はリーダーから逃げることである。男性的な働き方に順応することの難しさから、当初あった管理職志向も低下し、生活には適合するもののキャリアにつながらない仕事をすることになる。また、チャンスを失うことで自己効力感も得られない。\footnote{大沢・前掲注(1)〔本間道子〕。}

\subsubsection{自己効力感}

自己効力感とは、自分が役に立っている、会社から求められていると実感することで、管理職を志向するには必要となる要素である。自己効力感に必要となるものは、挑戦的な仕事や昇進に結びつくような仕事経験、有益な情報や人脈など昇進に役立つような資源、職場での所属感や上司からの承認や支持を感じられることなどだ。\footnote{大沢・前掲注(1)〔大槻奈巳〕25-33,87-110頁

  大沢真知子『女性はなぜ活躍できないのか』(東洋経済新報社,2015年)125-129頁。}

自己効力感はこのような職場の対人関係の中で組織に所属していることから得られる責任感や幸福感だが、性的役割分業的な価値観の下では職場でもこれらを感じにくいために管理職志向は低下する。

\subsubsection{解決策としてのポジティブアクション}

ここにおけるポジティブアクションは、管理職に昇進する人を先行する際に同程度の能力の男女がいたときに、全体において少ないほうの性別である女性を積極的に登用することで、全体としての性別に基づく差別を是正していく措置である。

このポジティブアクションを行うことは、実際に管理職に女性が就くことで管理職のイメージを変える・女性の自己効力感を高めることになる。また、該当女性だけではなく、ロールモデルを作ることで後続女性の評価方法や考え方に影響を与えたり、女性は守られるべきものだという好意的差別主義をなくしたりと社会構造にも影響する。

ポジティブアクションはただ女性を優遇する措置ではない。なぜなら、ステレオタイプにとらわれることなく、個々人に目を向ける多様性を認め、従来の男性的仕事人間像を押し付けないことは、男性の多様性も認められる労働環境の形成につながる。こうして、性別にかかわらず、すべての人が働きやすくなるはずだからである。この点においては、働き方改革と共通している。

社会構造や風潮が変わる価値観の転換が起きれば、実際上の取り扱いが変わって多様な働き方が可能になり、性別を問わずに能力を発揮できるようになって、女性の登用が進むはずだ。このように価値観の転換と多様な働き方、能力発揮、女性の登用はつながっている。しかし、価値観の転換それ自体が単体で起きることは、従来の性的役割分業に基づく現在の働き方の中では、根幹にある考え方を見直すことになるので難しい。そのため、女性の登用という手段を先に取ることで価値観の転換を促していくことが必要なのだと当班は考える。

\subsection{労働分野全体として女性が抱える問題について}

\subsubsection{現状分析}

就業率は男女で差はないのに、女性には非正規雇用者の割合が高く、管理職に占める割合が少ない。このことから男女の間には労働の質的な差があると言える。以下では、出産と就労、正規と非正規という雇用形態について見ていく。

\paragraph{出産と就労}

有職率のデータにおけるM字カーブは、多くの女性が結婚・出産を理由に25歳~35歳で1度退職することを示すが、年々改善傾向にあることから、出産後に職場復帰・再就職する人は増えているといえる。出産後の就業状況は、出産後半年で有職率は大きく下がるが、子供が小学校3年生になるまでに、徐々に有職率は上がっていき、出産前と同水準まで戻る。しかし、常勤という形態を選択する人よりもパート・アルバイトという形態を選択する人が多い。\footnote{男女共同参画白書
  令和2年度版(Ⅰ-2-3図)

   厚生労働省 平成22年出生児縦断第9回調査 3頁。}

\paragraph{正規・非正規}

人口に対する有職率に男女の差はほとんどないものの、正規・非正規という雇用の形態には男女差がある。25歳以降にこの雇用形態の差が明確になり、35歳以降も差は拡大している。男性のうち非正規雇用の職に就くのは、定年退職後の55~60歳以上では7割以上だがそれより若い年代ではほぼ2割以下となっており、就職から定年退職までは正規雇用で働くのがほとんどである。一方で、女性は働き盛りとされる25歳から非正規雇用の職に就く人が半数以上を占める。また、女性が非正規雇用を選択する理由の大半は家庭の事情であることから、女性の業務形態の選択肢は家事育児のために狭められている。\footnote{男女共同参画白書
  令和2年度版(Ⅰ-2-6図) 

   総務省 労働力調査詳細集計2019年分18,25頁。}

非労働力人口のなかで、就業を希望しているが,求職活動をしていない者の数である就業希望者数においても、総数では女性が男性の2倍と多く、25\textasciitilde{}54歳女性は男性の5倍とさらに差が大きくなっている。女性の就職希望者のうち、72%は働く形態として非正規雇用を希望し、求職していない理由は出産育児のためが31%、適当な仕事がありそうにないからが29%である。働きたいのに家事負担の多さにより、働く時間を作ることが難しかったり、勤務時間や形態の都合がつく仕事が世の中に少なかったりして、働けない状況にあることがわかる。\footnote{男女共同参画白書
  令和2年度版(Ⅰ-2-8図)

  総務省 労働力調査詳細集計2019年分28,29頁。}

\subsubsection{家庭との両立困難による離職・転職}

育児の負担は大きいことは様々な調査結果に現れている。子供が一人の時は仕事を続けていても、子供が二人以上に増えると専業主婦になる割合が高い。都心だと田舎と比較して子供を持たない夫婦の割合が高い。これは、都心での生活を維持するためには共働きを続ける必要があり、働き続けるには育児の負担は大きいと考えているからだと考えられる。最難関大学を卒業した女性は初婚も遅く、少子傾向にある。男性と同様に仕事をすることと家庭や育児の両立の難しさから、仕事か子供かの選択を迫られることになっているようだ。\footnote{株式会社日本総合研究所「高学歴女性の働き方調査」(2015年)。}

女性の家庭での役割は重い。大学卒業以上の高学歴な女性でも8割が離職や転職をしている。大学を卒業して就職しても仕事の内容は高度になったとしても、家事や育児の負担は変わらない。このことは女性管理職でもその9割が家庭の家事の6割以上をしていることからも明らかである。従来よりも女性が外で働くようになり、それを期待して男性と変わらずに努力し働いている女性でも家庭では当たり前のように家事・育児をしている。\footnote{小島明子『女性初の働き方改革で男性も変わる、企業も変わる』(産労総合研究所出版
  部経営書院、2018年)22-39頁。}

一方で、仕事を続けられた女性の家庭では平均した男性の家事参加率が高い\footnote{小島・前掲注(11)51-87頁。}ことから、やはり家庭内での家事・育児の分担ができれば女性の負担も減り、仕事と家庭を両立できる。このことから、女性にとっても男性にとっても、仕事と家庭を両立するために時間的制約を緩和することができる、フレックス制度の拡充や長時間の残業を減らすことは有益であるといえる。

\subsubsection{経済政策と女性の労働}

スウェーデンなどの大きな政府においては、高福祉高負担な社会保障制度がとられており、育休などの制度が発達している。女性にも労働力として期待しており、公的機関で雇う制度が整っていて、不況時でもこれは変わらない。

アメリカなどの小さな政府では、特に育休などは政治的には保障されていないが、その代わりに外部のサービスが充実している。私企業に男女を同条件で雇う制度が整っており、不況時には規制緩和で市場の活性化を図るため、有職率について男女差は大きくない。

異なる政策を取っているが、いずれも女性を労働力だととらえている構造があり、女性が男性と変わらずに働くための制度・サービスが整い、家事育児の外部化に成功している。

一方で、どちらでもない政府の多くは、出産などの一時的な福祉はあるが、不況時には雇用の縮小で性別役割分業が進むため、女性の失業や非正規雇用化が進む。日本ではさらに税の配偶者控除や保険料の控除制度が出産後の女性が家庭に入ることを後押ししている。家事育児をする女性を男性と同様には労働力として期待しないため、労働と並行した家事育児の外部化が進みにくい。\footnote{筒井・前掲注(3)22-29頁。}

\subsubsection{家事内容と家事の外部化}

産業の工業化からサービス業化により、雇用における働き手としての女性の需要は拡大する。それに伴い、家事育児の外部化の需要も高まるが、教育・育児や介護サービスには同時性が必要なので、効率化は難しい。\footnote{筒井・前掲注(3)161-171頁。}実際に介護を苦にした自殺率は年々増加していて、介護の負担によりカップル形成や出産・育児への余裕ない人も増加している。また、三世代同居の割合の高い都道府県は女性管理職の割合が高く、これは実祖父母に家事育児を委ねられるからだと考えられる。このように、現在の日本の主婦は担っている家事の内容からして家庭内部での委託は容易である一方、家事の外部化には限界があるといえる。

そもそも家事育児の外部化によって解決されるのか。当班の問題意識は女性が「外」で働くことを望んだ時に男性と同じスタートラインに立てているのか、というところにある。これまで男女雇用機会均等法などの制定により制度上の差別は禁止されてきたが、妊娠出産と仕事を両立できない現状がある。これは専業主婦前提の社会構造の定着から女性が家事・育児もすることが慣習化し、教育制度の中でも家庭科は女子のみが受講する科目となっていたこともあった。それらによって以前よりも外で働くようになった今でも女性が家事・育児の大部分をやっている。家事は女性がするものであるという風潮、慣習が変わることで解決するのである。そこで、祖父母に頼るなど家事の外部化は、その女性が働ける環境を作るという面では一時的な解決策ではあるが、本来は女性がするものだが働く上では困難なので外部に委託しているということになるので根本的な解決策ではない。家事・育児は女性がするものという慣習や風潮を変えるには、実際に夫も家事育児に参加することでしか解決できない面があると考える。

\subsubsection{家庭内での家事・育児分担について}

\paragraph{夫の家事負担の少なさ}

家庭内での夫の家事負担について注目すると、家事や育児を行うための家事関連時間は専業主婦世帯では妻が7時間56分なのに対して夫は50分と圧倒的に少ない。専業主婦世帯であるため、家事は妻が行う家庭内の分担になっていると考えられ、慣習からすると当然の結果であるといえる。一方共働き家庭における夫婦の家事関連時間は、妻が4時間54分なのに対して夫は46分、6歳未満の子のいる共働き夫婦では妻が6時間5分なのに対して夫は1時間22分である。妻が働いていても働いていなくても夫の家事関連時間は変わらず、夫の家事負担は少ない。全体として夫の家事関連時間は昔より増加しているものの、未だに夫婦間の差は大きい。\footnote{総務省統計局「社会生活基本調査」(平成28年)。}

\paragraph{夫の家事負担の少なさの要因とその解決策}

日本の男性の労働時間は長いから家事ができないとよく言われるが、労働時間の長さと家事関連時間に相関はない。自分はたくさん稼いでいるから家事しなくてもいいだろうと考える男性もいるかもしれないが、収入の多さと家事関連時間にも相関はない。専業主婦を前提としてきた性的役割分業社会の価値観・社会構造から女性の経済的な不安定さが生じ、男性は家事を女性がするものと考え、女性自身も生計維持分担の意識をもつ。日本人女性は先進国の中でも家事負担の不公平を感じにくいのはこのためだといえる。問題点は、社会構造からの女性の経済的不安定さと女性の家事への責任感、また夫婦間の家事スキルの差で、性的役割分業が原因となっている。解決策としては、性的役割分業にならない働き方の実現や女性の雇用の安定、夫婦間の家事スキルの差を埋めるための教育における家事トレーニングが挙げられる。\footnote{筒井・前掲注(3)171-186頁。}

\paragraph{夫の家事・育児負担に影響を及ぼす要因}

夫が家事・育児を負担することに影響を及ぼす要因について調べた調査がある。この調査では影響しそうな要因について仮説を立て、それぞれが夫の家事・育児負担との相関関係があるかを調べたものである。そのうち相関関係が見られたものが4つあった。

一つ目は相対的資源仮説であり、夫婦間の学歴差や所得差が少なくなるほど、つまり境遇が近いほど、夫婦間の家事分担も平等化している。二つ目は情緒関係仮説で、夫婦の仲が良いと仲が悪い夫婦に比べて夫の家事参加が多かった。これらは、境遇が近かったり、仲が良かったりすると、家事・育児の負担の大きさを共有しやすいからだと思われる。

三つ目は、時間的制約仮説で、夫の時間的制約が少なければ時間的制約が多い人よりも家事を行っている。四つ目は性役割イデオロギー仮説で、夫が性的役割分業を強く支持している場合はそうでない場合に比べて家事参加が低かった。三つ目と四つ目は、環境や価値観によって夫自身が家事・育児について責任感を持っていれば負担をするからだと思われる。時間的制約が少なければ家事・育児を自分のやるべきことだと捉えやすく、逆に自らの育った環境などから性的役割分業を支持している場合には家事・育児を自分のやるべきことだと捉えにくいからだ。\footnote{永井暁子「家事と仕事をめぐる夫婦の関係」日本労働研究雑誌719号(2020年)。}

\paragraph{政策による対処の可否}

では、これらの夫が家事・育児をすることの要因に政策的にアプローチをすることができるのかを考える。

相対的資源仮説での夫婦の境遇を平等に近づけることや時間的制約仮説での夫の時間的制約を減らすことについては、女性を男性と同様に活躍できるように積極的に登用していくポジティブアクションや男女ともに働きやすい環境を作ることを目的とする働き方改革で直接的に解消可能である。

一方で、夫婦の価値観に基づく性役割イデオロギー仮説や夫婦の仲の良さに左右されるとする情緒関係仮説に関しては、個人の価値観に影響を及ぼすのは社会構造や風潮であったり、夫婦の仲の良さは家庭内の問題であったりすることから、政策によって直接的に変化をもたらすことは不可能である。しかしながら、社会構造や風潮は女性の積極的登用や働き方改革によって変化しうるし、夫婦の仲の良さはよりストレスや負担の少ない労働環境の形成が進めばよりよくなりうるので、政策も間接的にいい影響を与えることができるはずだ。

\subsubsection{新性的役割分業}

専業主婦を前提としていた性的役割分業から、現状では女性が非正規雇用で働くことが当たり前となっており性的役割分業にも変化があると思われる。現状では、女性が主に家事負担をするという従来の性的役割分業の価値観はそのままに、「男性は仕事」「女性は仕事と家事・育児」をするという新性的役割分業になってきている。平成28年の総務省統計局「社会生活基本調査」によると、有業者の休日を含む週平均仕事時間は、男性は6時間49分であるのに対して、女性は4時間47分と短い。先に述べたように家事関連時間の長さ等から見ては女性に偏っていることと合わせて考えると、実際に男性は仕事、女性は仕事と家事をしていることがわかる。また、男女共同参画に関する世論調査によると男性の希望は必ずしも仕事優先ではなく、家庭の占める位置は大きいのに、現実には男性は女性に比べて仕事を優先している者が多い。

男性自身が家庭に時間を使いたいと思っているのに使えないのはなぜなのだろうか。男性自身が家庭に時間を使いたいと思っていてもそれができないのは、勤務の内容や形態からその希望をかなえることが不可能となってしまっているからだ。性的役割分業観からも男性は外で働くものとされ、男性の6割が「家族を養うのは夫の役割だ」ということに同意している。また、実際の業務の多さや働き方から、有給休暇を取りづらいと感じる人も多く、日本の有給休暇取得率は先進国中最下位である。まして育児休暇の取得率はとても低いし制度として整っていない会社も多い。男性は男性で「男のくせに」というような性的役割分業観に辛さを感じている。\footnote{小島・前掲注(11)92-102頁。}

このような男女問わずに抱えている働きづらさについて現職の管理職はどう考えているのだろうか。女性管理職の登用への意識調査からわかることがあった。

男性管理職のうち85%は女性管理職を登用することに賛成している。その理由としてイノベーション創出が53%、消費活動を実際にしている女性が関わることでの生産性・売り上げの向上が38%、成果主義への期待が33%と、女性管理職の必要性やメリットは浸透している。

その一方で、賛成した人のうち、業務の多さから長時間労働など従来通りの働き方はやめられないと思うと答えた人は6割にも上った。また、「家事・育児は妻がするべきものだ」ということを肯定したのは3割にとどまり、夫の家事・育児への参加の必要性を理解する一方で、「子供が3歳になるまでは女性は家庭にいるべきだ」ということを肯定したのは6割にも上り、家事育児の主体は妻にあると考えていることが分かる。\footnote{小島・前掲注(11)39-44頁。}

こうして、職場での女性活躍の必要性は理解しているものの、家事負担が女性に偏っている現状で実際に女性の活躍を推進していくのに必要な労働環境の見直しは難しいと考えていて、実際には女性が活躍するには負担が大きくなる働き方を強いられることになる。

\section{働き方改革 ~男女ともに仕事と家庭を両立できる社会へ~}

\subsection{働き方改革と女性の社会進出}

\subsubsection{働き方改革の概要}

働き方改革関連法は2018年6月29日に成立し、同年7月6日に公布された。労働基準法など計8本の法律を一括で改正するという内容である。働き方改革関連法それ自体は「労働者がそれぞれの事情に応じた多様な働き方を選択できる社会を実現する働き方改革を総合的に推進するため、長時間労働の是正、多様で柔軟な働き方の実現、雇用形態にかかわらない公正な待遇の確保等のための措置を講ずる」(厚生労働省ホームページより)ものであり、女性の社会進出を目指すことを目的とするものではない。 しかしながら、女性の社会進出を阻む理由として家事・育児といった家庭の事情が大きく関わっていることは上述の通りであり、働き方改革によって性別に関わりなく家庭や子育ての事情に応じた多様な働き方が選択できる社会の実現を目指すことは、女性の一方的な家事・育児の負担と軽減させるとともに仕事との両立を容易にさせることにつながると解する。以下では働き方改革の内容のうち、女性の社会進出に関係する分野について、その概要と女性の社会進出に対してもたらすメリットを検討する。

\subsubsection{時間外労働の是正}

労働基準法の改正により、労使協定(36協定)による時間外・休日労働について、労働者の健康確保、労働生産性の向上、ワークライフバランスの改善などを図ることを目的として、罰則付きで時間外・休日労働の上限時間が設定された。具体的には、時間外労働の上限について、月45時間、年360時間を原則とし、臨時的な特別の事情がある場合でも年720時間、単月100時間未満(休日労働含む)、複数月平均80時間(休日労働含む)を限度に設定している。また、月45時間を超えることができる月数を1年につき6ヶ月以内に制限している。上限時間の設定によって無限定な時間外・休日労働ついて制限を加え、さらに罰則規定を定めることで実効性の確保を目指したといえる。臨時的な特別の事情とは、予算、決算業務、
ボーナス商戦に伴う業務の繁忙、納期のひっ迫、大規模なクレームへの対応、機械のトラブルへの対応など、「当該事業場における通常予見することのできない業務量の大幅な増加等に伴い臨時的に第3項の限度時間を超えて労働する必要がある場合」(労働基準法36条5項)に当たらなければならない。特に事由を限定せずに「業務の都合上必要なとき」といった曖昧な理由はこれに当たらないとされており、より明確な時間外・休日労働を行う必要性を伴う理由が求められている。

また、月60時間を超える時間外労働に係る割増賃金率(50\%以上)について、中小企業への猶予措置を廃止している。これにより中小企業においても時間外労働の削減に対する取組みが促進されることが期待される。

家庭において、夫は時間的制約がなければ家事・育児に参画する傾向にあることは上述の通りである(時間的制約仮説)。有業者の週平均労働時間は男性より女性の方が2時間ほど多い一方で、共働き夫婦においては妻の家事関連時間が夫よりもかなり多い。このことから働き方改革の規定の着実な実行によって男性の長時間労働が是正されることになれば、夫である男性の家事・育児への参画が促進され、妻である女性の負担が軽減されることが考えられる。共働き夫婦において男女間の家事関連時間の差が縮まることになれば、その分妻の労働に対する制約が少なくなり、仕事を継続することが容易になると解される。また、これまで長時間労働を行っており仕事に支障を来したくないとの理由から家庭をもつことができなかった女性にとっても、今回の改革はプライベートを確保する時間を生み出す機会になりうる。このように男性の家事・育児への参画が促進されることにより、家庭内における女性の負担を軽減させることで、女性の家庭と仕事の両立を促し、キャリアアップを目指せる環境を創出し、維持し続けることが女性の管理職の割合を増やすために必要であろう。

\subsubsection{フレックスタイム制の見直し}

フレックスタイム制とは、労働者が日々の始業・終業時刻、労働時間を自ら決めることができる制度のことである。今回の改正ではフレックスタイム制の「清算期間」の上限を1か月から3か月に延長することとなった。清算期間を延長することによって、2か月、3か月といった期間の総労働時間の範囲内で、労働者の都合に応じた労働時間の調整が可能となった。

これによって、例えば子どもが家庭にいる時間が長くなる夏休みの期間の労働時間を減らし、その前後の期間において労働時間を増やすなど、家庭の事情に合わせた期間内の労働時間の設定が行いやすくなった。すなわち家庭の事情を理由とした労働の制約が少なくなったといえる。夫婦のいずれか一方が、あるいは夫婦共に柔軟な働き方が可能になれば家事・育児に対する負担は下がり、負担を多く被っていた女性の社会進出につながりうる。

\subsubsection{雇用形態にかかわらない公正な待遇の確保}

\paragraph{概要}

働き方改革では、パートタイム労働法、労働契約法、労働者派遣法の改正により、同一企業内における正規雇用労働者と非正規雇用労働者の間の不合理な待遇差の実効ある是正を図っている。具体的には、短時間・有期雇用労働者に関する同一企業内における正規雇用労働者との不合理な待遇の禁止に関し、個々の待遇ごとに、当該待遇の性質・目的に照らして適切と認められる事情を考慮して判断されるべき旨を明確化することを求めている。また、有期雇用労働者について、正規雇用労働者と職務内容、職務内容・配置の変更範囲が同一である場合の均等待遇の確保を義務化している。

「個々の待遇ごとに」、「正規雇用労働者と職務内容、職務内容・配置の変更範囲が同一である場合」と規定されていることからもわかるように、改正によって正規雇用労働者と非正規雇用労働者の待遇差が無条件になくなることを求めるものではない。また企業内における給与や各種手当ての位置づけや支給基準は企業ごとに異なるものであるから、合理的な待遇差であると判断できる一律の基準をあらかじめ設けておくことも難しい。したがって事案に応じて個別に各種の待遇差の合理性について検討していくことが求められる。

以下において正規雇用労働者と非正規雇用労働者の待遇差の合理性が問題となった2つの判例を紹介する。

\paragraph{メトロコマース事件(最判令2.10.13)}

この事件は有期労働契約を締結して東京地下鉄株式会社(東京メトロ)の駅構内の売店における販売業務に従事していた原告らが、無期労働契約を締結している労働者のうち販売業務に従事している者と原告らの間で、退職金等に相違があったことは労働契約法20条(改正前)に違反するものであったなどと主張して、被告(東京メトロの完全子会社)に対して。不法行為等に基づき、上記相違に係る退職金に相当する額等の損害賠償等を求めたものである。

最高裁は本件においては、正社員が契約社員と異なり、代務業務やエリアマネージャー業務を任され、本社の各部署や事業本部が所管する事業所等に配置され,業務の必要により配置転換等を命ぜられることもあったことからすれば、退職金はこれらの業務遂行能力や責任の程度等を踏まえた労務の対価の後払いや継続的な勤務等に対する功労報償等の複合的な性質を有するものであり、正社員としての職務を遂行し得る人材の確保やその定着を図る目的で支給されるものであって、契約社員から正社員へ段階的に職種を変更するための開かれた試験による登用制度を設けていたことも踏まえると、売店業務に従事する正社員に対して退職金を支給する一方で、契約社員である原告らに対してこれを支給しないという労働条件の相違は「労働契約法20条にいう不合理と認められるものに当たらない。」とした。

\paragraph{日本郵便事件(最判令2.10.15)}

この事件は有期労働契約を締結して勤務し、または勤務していた時給制契約社員又は月給制契約社員である原告らが、無期労働契約を締結している労働者と原告らの間で、年末年始勤務手当、祝日給、扶養手当等に相違があったことは労働契約法20条(改正前)に違反するものであったと主張して、被告(日本郵便株式会社)に対し、不法行為に基づき、上記相違に係る損害賠償を求めるなどの請求をする事案である。

最高裁はまず年末年始勤務手当について、郵便業務の最繁忙期であり、多くの労働者が休日として過ごしている上記の期間において、同業務に従事したことに対し、その勤務の特殊性から基本給に加えて支給される対価としての性質を有するのであって業務の内容や難易度に関わらずに一律に支給されるものであるから、「郵便の業務を担当する正社員に対して年末年始勤務手当を支給する一方で本件契約社員に対してこれを支給しないという労働条件の相違は労働契約法20条にいう不合理と認められるものに当たると解するのが相当である」とした。

次に祝日給について、最繁忙期における労働力確保の観点から、契約社員に対して祝日に対する特別休暇を付与しないこと自体には理由があるということはできるが、特別休暇が与えられることとされているにもかかわらず最繁忙期であるために年始期間に勤務したことについて、その代償として、通常の勤務に対する賃金に所定の割増しをしたものを支

給するという祝日給の趣旨は契約社員にも妥当するため、「郵便の業務を担当する正社員に対して年始期間の勤務に対する祝日給を支給する一方で、本件契約社員に対してこれに対応する祝日割増賃金を支給しないという労働条件の相違は、労働契約法20条にいう不合理と認められるものに当たると解するのが相当である」とした。

さらに扶養手当については、正社員が長期にわたり継続して勤務することが期待されることから、その生活保障や福利厚生を図り、扶養親族のある者の生活設計等を容易にさせ

ることを通じて、その継続的な雇用を確保するという目的で支給されるものであるが、扶養親族があり、かつ、相応に継続的な勤務が見込まれる契約社員についてもこの趣旨は妥当するとして、「郵便の業務を担当する正社員に対して扶養手当を支給する一方で、本件契約社員に対してこれを支給しないという労働条件の相違は、労働契約法20条にいう不合理と認められるものに当たると解するのが相当である」とした。

\paragraph{教育訓練における待遇の均一化}

以上は賞与や手当など、金銭面における待遇差に焦点をあてたものであるが、正規雇用労働者と非正規雇用労働者の間における待遇差はそれ以外の分野においても見られる。ここではその一つとして職場での教育訓練をあげる。

「令和元年度能力開発基本調査」における事業所調査によると、事業所での教育訓練のうち、業務命令に基づき通常の仕事を一時的に離れて行う教育訓練である「OFF-JT」を実施している事業者の割合は、正規雇用労働者に対しては75.7%、正規雇用労働者以外の者に対しては40.4%であった。また、日常の業務に就きながら行われる教育訓練である「計画的なOJT」を実施している事業者の割合は正規雇用労働者に対しては62.9%、それ以外の者に対しては28.3%であった。

これらの数値を見ると、正規雇用労働者とそれ以外の者で職場における教育訓練の環境に大きな差があるといえる。教育訓練は基本給や賞与、手当といった金銭面での待遇とは異なり、一度手にした技能が労働者個人のキャリアアップに直結し、その者の現在の待遇のみならず生涯にわたる収入にも関わるものであるといえる。特に有期契約労働者においては、働いていた企業との契約終了後に当該企業と契約を再締結する場合であれ、他企業に転職する場合であれ、新たな労働契約の締結に際して条件面を向上させることになるため、職場において教育訓練を受ける機会は重要である。

他方、企業側から見れば、正社員としての職務を遂行し得る人材の確保やその定着を図る目的でなされる教育訓練を、契約満了によって企業から離れる可能性のある有期契約労働者に対しても同様に導入することは難しいと解される。しかし、すべての有期契約労働者が短期間で勤めている企業を退職するとは考えにくく、職務内容や職務内容・配置の変更範囲が同一であれば、その職務に対する教育訓練を正社員と同じように一律に行う必要があるし、相応に継続的な勤務が見込まれる契約社員については趣旨が妥当する限り正社員との待遇差を設けることは不合理であるとする上述の日本郵便事件の判決の射程が及ぶと解される。したがって、教育訓練においても適切な形で正規雇用労働者と有期雇用労働者をはじめとする非正規雇用労働者との待遇差を是正することが求められる。

\paragraph{女性の社会進出との関係}

上述のように正規雇用労働者と非正規雇用労働者の間における待遇差が合理的かどうかであるかを判断するにあたっては、問題となっている待遇の性質・目的に照らして個別具体的に検討する必要があるのであり、非正規雇用労働者の待遇が正規雇用労働者のそれと全く同じになるとは言いがたい。しかしながら、個々の手当ごとに待遇差を不合理とする判例が蓄積されることにより、待遇差を設けることが法令に違反するかどうかについて予測可能性が高まれば、各企業において自主的に両者の待遇差を改善しようとする風潮が高まると考えられる。

家庭において、夫婦間の学歴差や所得差が少なくなるほど、夫が家事・育児に参画しやすくなるといえる(相対的資源仮説)。大学進学率における男女間の差は未だ開きがあるものの年々縮まっていることから、学歴差は少なくなる傾向にある。他方で上述のように雇用形態には男女間で大きな差があり、非正規雇用労働者に占める女性の割合は男性よりも圧倒的に大きいといえる。女性に非正規雇用労働者が多いのは家庭の事情が大きく関わっているため、この問題への対処が重要になるといえるが、それと同時に正規雇用労働者と非正規雇用労働者の間の待遇差を適切な形で是正し、非正規雇用労働者の待遇が改善されることで、男女間の所得差を減らし、夫である男性の家事参画を推進させることも進めていく必要がある。こうして家庭内における夫婦の家事・育児の適切な分担が達成されれば、妻である女性の負担が軽減され、職場においてさらなるキャリアアップを目指すことができるようになろう。

\subsection{働き方改革関連法以外の女性の社会進出を促進する法律、取組み}

\subsubsection{女性活躍推進法}

働き方改革関連法を直接構成するものではないが、働き方改革と密接に関連し、女性の職業生活における活躍を推進することを目的とする法律として、女性活躍推進法があげられる。この法律の中には、働き方改革のうち、女性活躍にスポットを当てた規定が置かれている。

まず、101人以上の労働者を雇用する事業主に対して、女性の活躍に関する状況の把握、改善すべき事情の分析を踏まえた「事業主行動計画」の策定・公表を義務づけている。計画に含まれる内容としては、管理職、再雇用者、昇給者のそれぞれに占める女性の割合といった女性労働者の職業生活や、育児休業取得率、フレックスタイム制・テレワークの導入率、残業時間といった職業生活と家庭生活の両立に資する雇用環境の整備などが含まれる。

また、国は女性の活躍推進について優れた取組みを行う一般事業主の認定(「えるぼし」認定)を行い、優良事業主に対しインセンティブを付与している。認定を受けた事業者は採用活動において自らの女性活躍推進に対する取組みをアピールできたり、公共調達での優遇を受けたりするなどのメリットが生ずると考えられている。

女性活躍推進法の規定では、ポジティブアクションにおけるクオータ制のように法律によって強制的な形で実効的に女性の社会進出を促進することはできないが、企業の自主的な取組みを促す点で、性的役割分業をはじめとする女性の活躍に対する社会の風潮を自然な形で、根本的に変化させることができると考える。

\subsubsection{男女雇用機会均等法に基づく指針の改正}

2015年に男女雇用機会均等法に基づく「労働者に対する性別を理由とする差別の禁止等に関する規定に定める事項に関し、事業主が適切に対処するための指針」の改正が行われた。

以前は募集・採用において、総合職、一般職などそれぞれの雇用管理区分でみて、労働者に占める女性の割合が4割を下回っている場合のみ、特例として女性のみを対象としたり、女性を有利に取り扱ったりすることが認められていたが、改正によって上記の場合に加え、係長、課長、部長などそれぞれの役職でみて、その役職の労働者に占める女性の割合が4割を下回っている場合も、特例として女性のみを対象としたり、女性を有利に取り扱ったりすることが認められるようになった。

これによって女性中途採用が促進されることになれば、我が国で大きな問題となっている管理職における女性の割合の向上が見込まれる。もっとも、女性のキャリアアップを阻害する主な要因となるのは、新生的役割分業に基づく家事・育児の負担であるため、これを解消するために働き方改革の取組みを着実に実施していく必要がある。

\subsubsection{NPOによる助成の再就職支援}

上述のように現代においては専業主婦の約6割が再就職を希望する一方で、体力的な要因や家庭の事情によりフルタイムの労働が困難であることから再就職を諦める者が多いという現状である。

これらの状況の下、出身によって一度仕事を離れた者に対する職場復帰の支援を行うNPO法人の活動が見受けられる。具体的な活動内容としては、企業の人事部への説明会・講習会や、出身・育児休業中にできるスキルアップへのアドバイスや関連講座の開講、体力回復のための運動を行う講座の開講などがあげられる。

こうした活動に対する支援を明確に規定する法令はないが、政府としてはこうした活動を行うNPO法人への助成を検討しつつ、NPO法人の活動が女性の再就職支援に対して有効なものとなるよう、助成基準を的確に定めることが求められる。

\subsection{小括}

女性の社会進出を阻む要因として、新生的役割分業に基づく社会の風潮が大きな割合を占めているといえるのは上述の通りである。働き方改革関連法をはじめとする各法令によって、女性のキャリア形成が容易になり、一方で女性管理職の登用が促進され、他方で出産・育児を経た女性の職場復帰や再就職が促進されることになれば、社会において重要な役割を果たす女性が増え、意思決定に女性の意見が加えられることによって、男性の性的役割分業感・責任感からの解放、職場でのイノベーション創出といったメリットが見込まれる。他方女性の社会進出が進めば、実力主義をはじめとする人事評価の透明化やリモートワークの拡充、育児休業取得の促進、コース別人事の改革など、女性に配慮した職場環境が一層整備されることが期待できる。職場環境が整備されればさらに女性の社会進出がすすむ、といった好循環を経て、現代における問題の根本にある社会の価値観の転換が実現するといえる。

\section{総括}

現代において働き方改革の施策を実行することは、男性による家事・育児への参加を促進させ女性の負担を減らすことにより、働きたい女性が制約無く働ける社会の実現につながる。一方で人材育成や再就職支援など、現状の施策では不十分な分野に対する対策をさらに講ずる必要がある。働き方改革という今話題の政策がなぜ必要になったかを探ると、長時間労働ひいてはそれを可能にしてきた「専業主婦前提の社会構造」が見えてくる。憲法で問題になりそうな形式的不平等がなくなった今も、働きたい女性が不利益を被る状況は残っている。社会における女性の参画を通じて、女性の「新・性的役割分業」をはじめとする社会の風潮を変えることが根本的な解決にとって重要である。

\section{参考文献}

・大沢真知子『女性はなぜ活躍できないのか』(東洋経済新報社,2015年)。

・大沢真知子編『なぜ女性管理職は少ないのか
女性の昇進を考える』(青弓社、2019年)。

・小島明子『女性初の働き方改革で男性も変わる、企業も変わる』(産労総合研究所出版
部経営書院、2018年)。

・川島武宜『日本人の法意識』(岩波書店、1967年)。

・片岡曻『労働法(1){[}第4版{]}総論・労働団体法』(有斐閣、2007年)。

・株式会社日本総合研究所「高学歴女性の働き方調査」(2015年)。

・久保桂子「共働き夫婦の家事・育児分担の実態」日本労働研究雑誌689号(2017年)。

・小林明子『女性初の働き方改革で男性も変わる、企業も変わる』(産労総合研究所出版部経営書院、2018年)。

・辻村みよ子『ジェンダーと法{[}第2版{]}(不磨書房、2010年)』。

・筒井淳也『仕事と家族
なぜ日本は働きづらく、産みにくいのか』(中公新書、2016年)。

・永井暁子「家事と仕事をめぐる夫婦の関係」日本労働研究雑誌719号(2020年)。

・中野円佳『なぜ共働きも専業もしんどいのか 主婦がいないと回らない構造』(PHP、2019年)。

・日本弁護士連合会第58回人権擁護大会シンポジウム第1分科会実行委員会編著『女性と労働
:
貧困を克服し男女ともに人間らしく豊かに生活するために』(~旬報社、2017年)。

・藤井治枝「戦後女性労働の推移と日本的特質」労務理論学会誌16巻(2006年)41-65頁。

・水町勇一郎『労働法{[}第8版{]}』(有斐閣、2020年)。

・三成美保他『ジェンダー法学入門{[}第3版{]}』(法律文化社、2019年)。

・毛利透他『憲法Ⅱ人権{[}第2版{]}』(有斐閣、2017年)。

・男女共同参画白書 令和2年度版。

・総務省 労働力調査詳細集計2019年分。

・総務省統計局「社会生活基本調査」(平成28年)。
