\section{はじめに}

\subsection{背景説明}

DNAが収集され、データベース化されたことに対する訴訟の事例が現在、増加している。

名古屋では、犬を捜すために、チラシを電柱に貼ったことが、市の屋外広告物条例に違反したとして、顔写真を撮影され、指紋を採られ、DNAまでも採られた事例がある。当事者は屋外広告物条例違反について、不起訴処分となったが、取られた個人情報、特にDNAのデータがどうされるのかが不安になり、国を相手としてDNAデータの抹消やプライバシー侵害による精神的な苦痛への補償などを求め、名古屋地裁に提訴した。

同事件で警察庁は当事者のDNA型データは保管していないと主張したが、都道府県警で鑑定したデータは警察庁に送ることになるDNA型記録取扱規則(平成17年国家公安委員会規則第15号)が定められているので、当事者は警察庁に自分のDNAデータが確かに保存していないという証明を求め、争いはまだ続いている。

また、愛知県で立ち入り禁止の場所でバス釣りをしていたとして、軽犯罪法違反の容疑で県警から事情聴取を受け、DNAなどを取られた者もいる。不起訴処分となった当事者はデータの抹消を求めて国や県を相手取り、訴訟を起こした。この裁判では、警察庁が男性のDNA型データを保管していることを認めたのである。

その他に、DNA型データの抹消などを求めたこれまでの訴訟で、裁判所も「適法に採取、登録したものは抹消の理由がない」との判断を示している。警視庁に東京都迷惑防止条例違反の疑いで調べを受けた男性が抹消などを求めた訴訟で、東京地裁は2016年6月、データの採取などは適法に行われ人格権の侵害などにあたらず、目的を逸脱してデータが運用されているとも認められないとして男性の訴えを退けた。判決はその後、東京高裁で確定したという\footnote{朝日新聞「犬探して貼り紙\ldots{}それだけでDNA『一生、容疑者扱い』」(2020年8月24日記事)。}。

以上述べた事例によると、顔写真や指紋、DNAなど個人識別情報を採取することに罪種の限定がなく、微罪でも取られる場合があるということが分かった。そして、同様に不起訴処分になったことと、採取されたDNAがデータ化されたかどうかということは連動するわけではないということも示した。更に、どのような場合であればDNAなど個人識別情報がデータベースから抹消されるのかに関する基準はあいまいであり、不起訴処分を受けたとしても、取られたDNAの情報が削除されないままデータベースに残る可能性があるということが明らかになった。

犯罪にかかわった疑いがあるとして警察が逮捕など検挙した容疑者から得たDNA型のデータベースの登録件数が、年間十数万件のペースで増え続け、2019年末時点で約130万件にのぼることが分かった。日本の人口のほぼ100人に1人にあたる数だ。検挙した容疑者の多くからDNAを得ていることになり、対象の罪種は重要犯罪に限らず軽微なものも含まれている\footnote{朝日新聞「容疑者の多くからDNA採取 DBに130万件と判明」(2020年8月23日記事)。}。そうすれば、一度登録されたら、当事者は一生、容疑者扱いされてしまう可能性が否定できないのである。

\subsection{問題提起}

本稿で着目したい問題点は、まず、どのような状況であればDNAサンプルを取ることが許せるか。行為の態様や罪の分類によって比例原則にしたがって分ける必要があるかを検討する。次に、取られたDNA情報の保存、利用、特にデータベース化をし、単一事件で利用されるだけではなく、本人の同意がなくして死亡まで自分の情報を何回でも利用されるかもしれないということが、直接的には国家公安委員会規則である「DNA型記録取扱規則」により規定することについて、その合憲性を考察する。

\subsection{全体予告}

本稿ではまず、DNA情報の現状、つまり収集、利用及びデータベースの構築についての現状を説明する。次に、それと憲法13条により保障されるプライバシー権ないし自己情報コントロール権と35条の令状主義に関する合憲性を検討する。そしてDNA情報をデータベース化の合憲性について、令状主義の精神、比較衡量、構造審査及び指紋データベースとの比較の立場からアプローチして論じる。また、現時点でDNAデータベースを構築することを、法律の根拠がない国家公安委員会規則で行うということができるのかを議論する。最後に、諸国の法制度から、日本のDNAデータベースに関する制度をどう整えるのかを論じる。

\section{DNA情報について}

\subsection{個人情報を収集する根拠}

行政機関の保有する個人情報の保護に関する法律第3条第1項によると、「行政機関は、個人情報を保有するに当たっては、法令の定める所掌事務を遂行するため必要な場合に限り、かつ、その利用の目的をできる限り特定しなければならない」と定めており、2項では、「前項の規定により特定された利用の目的の達成に必要な範囲を超えて、個人情報を保有してはならない」と示している。利用目的が変更する場合があれば、同条3項でそれは「変更前の利用目的と相当の関連性を有すると合理的に認められる範囲を超えて行ってはならない」と規定している。

次に、警察庁は、犯罪鑑識に関する事務を所掌し\footnote{警察法5条第2項第19号及び17条参照。}、当該事務の一環として犯罪捜査に資する目的で
DNA
型データベースを構築・運用するため、都道府県警察からDNA型記録を収集する。警察庁の犯罪鑑識に関する事務は国家公安委員会の管理の対象であり、国家公安委員会は、管理に係る事務を適正に行うために
DNA 型記録等の収集等に関する国家公安委員会規則を定めた\footnote{警察法81条及び警察法施行令13条第1項参照。}。

都道府県警察が DNA 型鑑定を行うために個人からDNA
資料を採取することは、都道府県警察の捜査の必要性の判断に基づき、刑事訴訟法197条の任意捜査及び218条・225条の強制捜査を根拠とする。犯人を特定する目的で得た
DNA
型情報を都道府県警察が警察庁に送信することは、警察法等に基づく国家公安委員会規則にしたがうものである。DNA
型記録等の個人情報に関しては、個人情報にかかわる法律等(公務員法、行政機関個人情報保護法による秘密・個人情報の保護、警察法及び同法に基づく国家公安委員会規則による規律の保持・監察の実施、各都道府県の条例による目的外利用等の制限)により警察組織・職員の活動が律され、個人情報の保護が図られる\footnote{末井誠史「DNA型データベースをめぐる論点」レファレンス
  61(3)(2011年3月)14頁。}。

なお、DNA型記録取扱規則では、DNA分析に関する手続きが十分に明確化されておらず、対象者の範囲の特定、鑑定についての手続きの保障、記録に関わる手続き保障など、様々な面での不備が指摘されている現状である\footnote{玉蟲由樹「警察DNAデータベースの合憲性」日本法学82(2)(2016年10月)665頁。}。

\subsection{目的}

警察庁のデータベースに関する国家公安委員会規則の目的は、DNA
型記録取扱規則第 1
条により、「犯罪捜査に資すること」と規定されている。個人識別の情報が保有されることにより、将来犯罪行為をしたときには当該個人の犯行であることが容易に確認され得ることを認識させることを通じて犯罪の抑止機能を果たすことについて言及されることがあり、立法時には、犯罪捜査以外の目的も設定するかどうかという論点はある\footnote{末井・前掲注5)12頁。}。

日本弁護士連合会「警察庁 DNA
型データベース・システムに関する意見書」によると、データベースは「原則として具体的な捜査の目的のみに利用されるべきである」とし、例外として冤罪を解明する目的の利用を認めることを主張する\footnote{日本弁護士連合会「警察庁
  DNA
  型データベース・システムに関する意見書」(https://www.nichibenren.or.jp/library/ja/opinion/report/data/071221\_000.pdf)(2007年12月)16頁。}。

\subsection{採集種類}

2004年12月、警察庁は犯罪現場などに遺留された血痕のDNA型情報検索システムの運用を開始し、2005年8月に国家公安委員会規則15号「DNA型記録取扱規則」が公布され、9月より施行されて運用が開始され、これにより、警察の遺留資料の他に、被疑者・変死者のDNA型情報もデータベースの対象となった。つまり、収集される資料となるものは、①同意もしくは令状により被疑者の身体から採取された被疑者資料、②犯罪現場その他の場所に被疑者が遺留したと認められる遺留資料、又は③身元が明らかでない変死者などの死体から採取された変死者資料に類型化されている。なお、被疑事実に限定はない\footnote{日弁連・前掲注8)4-5頁。}。

その他に、調査協力を理由として犯罪現場の近所にいる住民の同意に基づいて、任意提出によりDNA情報を採取することもある。このような情報は④一般人DNA型情報と言い、警察によると、このような調査協力によって採取されたDNA情報は犯罪捜査が終了する際に、当該情報を削除すると述べた。

本稿では、自己情報コントロール権とは関係ないの②遺留資料・③変死者資料、及び同意に基づいて取得した④一般人DNA型情報は議論しなく、①被疑者資料のみに注目する。

\subsection{利用現状 }

DNA型鑑定は細胞核にあるDNAを構成している塩基の配列や繰り返しのパターンが個人によって異なることを利用し、個人を識別することができるようになり、日本で警察の捜査に導入されたのは1989年からである。当初、200人に1人ほどだった個人識別の精度は、検査手法の向上で565人に1人にまで向上している。そして、

容疑者のDNA型鑑定で別の犯罪への関与が分かることも多い。2005年から、警察庁が容疑者のDNA型についてデータベースの運用を始めた。登録されるDNA型は毎年十数万件で増え、2019年に130万件まで達し、自白偏重による無罪判決が相次ぎ、客観証拠が重視されるようになったことも、データベース件数が増えた背景の一つである。

DNA型の採取や保管について、直接に規定した法律はなく、行政機関個人情報保護法や国家公安委員会規則が根拠とされているだけである。そして、同規則ではデータの抹消について、それは登録された者が死亡した際に、もしくは記録を保管する必要がなくなった際にしか抹消しないと規定している。実際に保管の必要があるかどうかも、行政的な裁量であり、明確な基準が定めていない。新聞記事により掲載された警察庁幹部の説明によると、DNAの採取は原則的に任意提出であり、データの抹消は事件ごとに各都道府県警の判断に委ねるようで、情報を漏えいすることを避けるように、「厳重に管理されていることは間違いない」と話し、採取した資料自体も鑑定でほぼ全量を消費し、残りは廃棄するのが通常であるという\footnote{日本経済新聞「警察のDNAデータ増大 登録130万件 身元確認の切札 保管・抹消の法整備課題」(2020年11月10日記事)。}。

朝日新聞の記事によると、DNA情報が取られた罪種が7割は刑法犯であり、一番多いのは窃盗罪で、約42万件だった。そのうち重要犯罪は5%にも達していないことが分かった。その他に、刑法犯では詐欺、占有離脱物横領、脅迫、遺失物横領など、特別法犯では道路交通法違反、銃刀法違反、出入国管理法違反や軽犯罪法違反、風俗営業法違反などが含まれるという\footnote{朝日新聞・前掲注2)。}。

\section{具体的な事件を解決するためのDNA型情報取得の合憲性について}

以上見てきたように、DNA型情報はデータベース化されているが、このDNA型情報はもともと具体的な事件の解決のために取得されたものである。そこで第3章では、そもそも、捜査機関が事件を解決するためにDNA型情報を取得することの合憲性について考えていきたい。

\subsection{憲法13条の合憲性について}

捜査機関が事件解決のためにDNAサンプルを採取し、DNA型情報を取得する場合として、①同意により被疑者から取得する場合、②令状により被疑者から取得する場合、③同意により捜査の協力者から取得する場合の3つの場合があり得る。ここでDNA型情報は個人のセンシティブな情報を持たない、単なる個人識別情報にすぎないが、指紋押捺拒否事件(最高裁平成7年12月15日第三小法廷判決)で判示されたように、「性質上万人不同性、終生不変性を持つので、採取された指紋の利用方法次第では個人の私生活あるいはプライバシーが侵害される危険性がある」。そのため、DNA型情報を開示するか否かは、憲法13条の保障するプライバシー権に含まれると解される。そして、同意のある①、③の場合は、プライバシー権は制約されないが、②の場合は被疑者のプライバシー権が制約される。そのため、②の場合は被疑者のプライバシー権の制約を公共の福祉の観点から正当化できるか検討しなければならない。

ここで、DNAサンプルから取得するDNA型情報は個人識別情報にすぎないため、思想・信条・精神・身体に関する基本情報、重大な社会的差別の原因となる情報とは言えない。そして、DNAサンプルの採取が、口腔内組織片や唾液の採取といった比較的侵害態様が軽微である。以上のことから、合理性の基準を採用する。

たしかに、京都府学連事件(最高裁昭和44年12月24日大法廷判決)では、個人識別情報にすぎない顔写真について、厳格な合理性の基準を採用している。しかしながらこれは、令状主義の精神から厳格に審査されているとみることができる\footnote{憲法判例研究会編『判例プラクティス憲法〔増補版〕』(信山社、2016年)44頁〔山本龍彦執筆〕。}。DNAサンプルの採取はそもそも令状に基づいて行われるため、京都府学連事件のような考慮は要しない。以上より、合理性の基準を採用する。

ここで、DNA型情報の取得の目的につき、犯罪の捜査と言えるが、これは警察法2条1項により、警察の責務とされている。そのため、DNAサンプルの採取の目的は正当と言える。DNA型情報の取得の手段についても、DNA型情報により、犯人の特定につながるなど、犯罪の捜査という目的と合理的に関連していると言える。

以上より、犯罪の捜査のためにDNAサンプル採取しDNA型情報を取得することは憲法13条に反しないと言える。

\subsection{憲法35条の合憲性について}

憲法35条は、「何人も、その住居、書類及び所持品について、侵入、捜索及び押収を受けることのない権利は、第三十三条の場合を除いては、正当な理由に基いて発せられ、且つ捜索する場所及び押収する物を明示する令状がなければ、侵されない。」とする。ここで「正当な理由に基づいて」とは、「侵入・捜索・押収を必要とするだけの十分な理由」を意味し、「明示する」とは、「捜索および押収の対象たる場所および物を個別・具体的に特定して」という意味である\footnote{佐藤幸治『日本国憲法論』(成文堂、2020)360頁。}。ここで、事件解決の必要性がある場合、「侵入・捜索・押収を必要とするだけの十分な理由」があると言え、「捜索および押収の対象たる場所および物を個別・具体的に特定」されているかについても、口腔内組織片などであり、特定されていると言える。

以上より、事件解決を目的にDNA型情報の取得のためにDNAサンプル採取することは憲法35条に反しないと言える。

\subsection{DNAサンプル採取は捜索に当たるか}

上記のように、捜査機関がDNA型情報を得るためには、まずはDNAサンプルを採取する必要があるが、このDNAサンプル採取について、現在同意または令状に基づいて行われているため、このDNAサンプル採取が「捜索」に当たるか否かは問題とならない。しかしながら、将来、捜査機関のDNAサンプル採取の心理的ハードルが下がり、無令状でDNAサンプル採取が行われる可能性もないとは言い切れない。DNAサンプルの採取が「捜索」に該当する場合、無令状でDNAサンプルを採取することは、憲法35条に違反する。そのため、以下本節ではDNAサンプルの採取が「捜索」に当たるか否かを検討する。その際、山本龍彦が整理し、紹介するアメリカの判例の議論を参考する\footnote{山本龍彦『遺伝情報と法政策』(成文堂、2007年)113-117頁。}。

DNAサンプルの採取は綿棒により口腔内組織片を採取することにより行われるが、口腔内組織片の採取の身体への侵襲の程度は比較的低いと言える。また、人の口腔内は、会話や欠伸などにより、しばしば公衆に晒しているとも言える。そのため、口腔内を晒すによる羞恥心や社会的当惑の程度は低いと言える。そのため、「捜索」に該当しないと言える可能性がある。

一方で、DNAサンプルの採取は、以下の点で、一般に「捜索」に該当するとされる、採尿と類似していると言える。①対象者の口腔内を擦るという行為は、人間の尊厳に関する利益に係る点、②口腔内は公的領域に属しているとまではいえない点、③口腔内組織片は尿サンプルと同じく、化学分析によって明らかにされうる「個人に関する多くの私的な医学的事実」、「夥しい量の遺伝的アイデンティティ情報」を含んでいるという点。以上3点の類似性により、やはりDNAサンプルの採取は「捜索」に当たるというべきである。

以上より、DNAサンプルを採取すること「捜索」に当たるため、これを無令状で行うとすると、憲法35条に違反する。

\subsection{身体拘束中の被疑者の場合}

第3節より、DNAサンプルを採取することは「捜索」に当たるため、刑事訴訟法上の、強制処分に該当すると言える。一方で刑事訴訟法218条3項は、「身体の拘束を受けている被疑者の指紋若しくは足型を採取し、身長若しくは体重を測定し、又は写真を撮影するには、被疑者を裸にしない限り、第一項の令状によることを要しない。」と規定し、身体拘束中の被疑者について、一定の強制処分を無令状で行うことを認めている。そこで、刑事訴訟法218条3項を類推して、捜査機関が身体拘束中の被疑者から、DNAサンプルを無令状で採取できるか否かについても検討する。

そもそも刑事訴訟法218条3項で一定の強制処分につき無令状で行うことが可能なのは、指紋採取などの行為は、被疑者を特定するうえで必要とされる一方で、人権侵害の程度は比較的小さく、すでに身体の拘束という強制力を加えている以上、この程度の強制は許されるものと考えられるし、更に、身体の拘束という行為は、被疑者にこの程度の強制を加えて証拠を収集することを予定しているものとも考えられるのであって、いわば指紋の採取等のこれらの処分は、身体の拘束という処分のなかに実質的に包含されるものと解されるところから、令状を要しないものとされる。そうすると、指紋の採取などの処分は、身体の拘束に当然付随する程度の検証の例示と見ることができ、刑事訴訟法218条3項を類推適用し、口腔内組織片を採取するなどは許されると解することもできる\footnote{伊藤栄樹ほか『注釈刑事訴訟法(新版)第3巻』(立花書房、1996年)205-206頁。}。

しかしながら、DNAサンプル採取後、いくら実際に使用する部分はセンシティブな情報を含まない非コード領域だけだとしても、センシティブな情報の宝庫である口腔内組織片などのDN
Aサンプルと、センシティブな情報を得ることが不可能な指紋等と同列に扱うことはできない\footnote{水野陽一「刑事手続における強制採血とDNA型鑑定に関する一考察」広島法学36巻2号(2012年)137頁。}。

以上より、身体拘束中の被疑者についても、無令状でDNAサンプルを採取することは許されないとするべきである。

\section{データベース化の合憲性について}

第3章では具体的な事件を解決するために、DNAサンプルを採取し、DNA型情報を取得することの合憲性について検討したが、警察では、取得したDNA型情報をデータベース化することが行われている。そこで第4章では、DNA型情報をデータベース化することの合憲性について考えていきたい。

\subsection{京都府学連事件からのアプローチ}

DNA型情報の取得が同意に基づく場合であっても令状に基づく場合であっても、当該事件限りの使用でしか正当化されない。そのため、DNA型情報のデータベース化をするためには、別の正当化理由が必要である。ここで、DNA型情報のデータベース化は同意にも令状にも基づかない、情報の再収集という側面を有する。そこで、同じく、同意にも令状にも基づかずに顔写真という個人識別情報を収集した京都府学連事件(最高裁昭和44年12月24日大法廷判決)の判断枠組みを検討する。

京都府学連事件では、憲法13条により、「何人も、その承諾なしに、みだりにその容ぼう・姿態(以下「容ぼう等」という。)を撮影されない自由を有する」とした上で、「個人の有する右自由も、国家権力の行使から無制限に保護されるわけでなく、公共の福祉のため必要のある場合には相当の制限を受けることは同条の規定に照らして明らかである。」とし、制限が許容される限度について、「身体の拘束を受けている被疑者の写真撮影を規定した刑訴法218条2項のような場合のほか、次のような場合には、撮影される本人の同意がなく、また裁判官の令状がなくても、警察官による個人の容ぼう等の撮影が許容されるものと解すべきである。すなわち、現に犯罪が行なわれもしくは行なわれたのち間がないと認められる場合であつて、しかも証拠保全の必要性および緊急性があり、かつその撮影が一般的に許容される限度をこえない相当な方法をもつて行なわれるときである。」と判断している。

つまり、本人の同意もなく、裁判官の令状もなくても警察官による個人の容ぼう等の撮影が許される場合として、①現行犯ないし準現行犯的状況、②証拠保全の必要性および緊急性、③手段の相当性という3つの要件を要求している。ここで京都府学連事件は、捜索・押収等に司法官憲の発する各別の令状を求めることによって、公権力の典型としての刑事警察権を統制し、国家の恣意性を抑え込もうとする憲法35条の令状主義の精神から厳しい要件を立てたとも見ることができる\textsuperscript{1}。そのため、犯罪の予防を目的とした撮影の場合はこの3要件が適用されないとも思われる。そこで高裁判決であるが、犯罪の予防を目的とした警察による撮影が許容される基準を示した、山谷テレビカメラ監視事件についても検討する。

山谷テレビカメラ監視事件では、「当該現場において犯罪が発生する相当高度の蓋然性が認められる場合であり、あらかじめ証拠保全の手段、方法をとっておく必要性及び緊急性があり、かつ、その撮影、録画が社会通念に照らして相当と認められる方法でもって行われるときには、現に犯罪が行われる時点以前から犯罪の発生が予測される場所を継続的、自動的に撮影、録画することも許されると解すべき」と判断している。ここで京都府学連事件の3要件のうちの①の要件を「当該現場において犯罪が発生する相当高度の蓋然性」に緩和している。つまり、犯罪の予防を目的とした警察による撮影が行われた事案において、①当該現場において犯罪が発生する相当高度の蓋然性、②証拠保全の必要性および緊急性、③手段の相当性という3つの要件を満たせば許容されるとしている。

京都府学連事件、山谷テレビカメラ監視事件のどちらの判断基準をとっても、現状すべての犯罪につきデータベース化することは、問題があり、当該対象者のDNA型情報をデータベースに入れるには、再犯率の高さなどの要件を要求すべきである。

一方で、以上の議論は指紋データベースにもあてはまるが、現状指紋データベースについては問題視されていない。そのことから推察されるに、京都府学連事件や山谷テレビカメラ監視事件の要件は、監視国家に転落しないために、裁判所が形成してきた予防的ルールに過ぎず、必ずしも憲法上の要請でない可能性がある\footnote{山本・前掲注14)143頁。}。実際に最高裁平成20年4月15日決定は、被侵害利益の重大性と撮影の必要性等を考慮する比較衡量で判断している。

しかしながら、そもそも憲法が31条以下で刑事手続に関する規定を置いているのは、自由権が国家権力によりいたずらに剥奪されないようにするためである。つまり、国家権力が個人の自由を剥奪してよい例外の手続条件を特記し、国家権力を手続的に拘束することで、個人の自由を確保しているのである。そして、このような観点からすると、憲法31条以下で保障される権利は公共の福祉との調整のために制限されることはあってはならず、比較衡量とはなじまない\footnote{奥平康弘『憲法Ⅲ 憲法が保障する権利』(有斐閣、1993年)297-301頁。}。そうすると個人識別符号であるDNA型情報について、憲法35条の適用があるかどうかは怪しいが、令状主義の精神から、京都府学連事件のような3要件を立てるのは、憲法の考えに合致していると考えられる。よって、DNA型情報をデータベース化することが許容されるためには、再犯率の高さなどの厳しい要件が必要と考える。

\subsection{比例原則からのアプローチ}

比例原則とは、個人の自由に対する公権力の介入が比例的であることを要求するもの\footnote{玉蟲由樹「平等取扱原則と比例性」日本法学85巻2号(2019年9月)。}であり、この時、目的と手段は不釣り合いであってはならず、規制は最小限である必要があるとされている。

次に、警察のDNAデータベースに関する規則であるDNA型記録取扱規則を参照してみる。DNA型記録取扱規則では第1条に「この規則は、被疑者DNA型記録等を組織的に作成し、管理し、及び運用するために必要な事項を定め、もって犯罪捜査に資することを目的とする」という目的規定をおいている。そして第7条1号と2号には「被疑者DNA型記録に係るものが死亡したとき」「被疑者DNA型を保管する必要がなくなったとき」には当該DNA型を抹消しなけらばならないという削除規定がおかれている。

これを比例原則に当てはめると、目的は犯罪捜査に資することであり、この目的を達成する手段としてDNA型記録のデータベース化がとられていることが分かる。そして、この時制約される権利は自己情報コントロール権となる。ここでは、自己情報コントロール権とは「自己の存在に関わる情報を開示する範囲を選択できる権利」\footnote{佐藤・前掲注13)203頁。}を示す。

比例原則からアプローチをするにあたっては、簡易的にDNAが採取されるケースを大きく4つに分類する。1つ目は重要犯罪(殺人、強盗、性犯罪など)の場合、2つ目は重要犯罪に該当しない刑法犯(暴行、傷害、窃盗など)の場合、3つ目は微罪の場合、4つ目は無罪の場合である。以下、詳しく検討していく。

\subsubsection{重要犯罪}

対象となる人物が重要犯罪を犯していた場合、削除を求める権利または何年保管したのちに削除するなどの削除規定があれば良いと考えられる。例えば、重要犯罪のうちの一つである性犯罪を例にとってみると、その再犯率は13.9\%\footnote{法務省(2015)『平成27年度犯罪白書』。}であり、全再犯者のうちの67.4%\footnote{法務省・前掲注21)。}を占めていた。DNAデータベースが持つ再犯防止効果によりこの数値を抑えることは可能であると考えられる。次に重要犯罪を犯した者の持つべき削除を求めることのできる権利の程度について考えてみる。参考のために海外に目を向けてみると、フランスは性犯罪その他生命犯で有罪が確定した者と当該罪を犯したことを推定される重大な帳票がある被疑者のDNA型をファイルに保存できると規定している。削除に関しては、被疑者について登録目的に照らして保存が不必要となれば、もしくは検察官か当人の求めにより可能となっており、被疑者本人が削除を求めることが権利として規定されている。再犯防止効果と犯罪の重さの程度などを考えると、フランスのように本人が削除を求めることのできる権利の規定は最低限必要であると考えられる。

\subsubsection{重要犯罪に該当しない刑法犯}

重要犯罪に該当しない刑法犯に関しては、保管期間や削除規定を定める必要があると考える。窃盗犯を例として考えてみる。窃盗犯は5年以内累積再犯率が最も高い犯罪の一つであり令和2年度の再犯率は48.8%となっている\footnote{法務省(2020)『令和2年年度犯罪白書』。}。よって、その再犯を防止するための策としてDNAのデータベース化はかなり有効がある可能性が高い。なぜなら、DNAデータを保管されている人物は少しでもDNAを採取できるような証拠物を現場に残すだけで、警察によって特定され、逮捕される可能性が高まるからである。一方で、現状の規定だけでは自己情報コントロール権に対する制約が強すぎると考えられるので、一定の保管期間しかり削除規定を定めることは必須であると考えられる。

\subsubsection{微罪}

ここでいう微罪とは刑事訴訟法246条但書に規定されているものである。検察官があらかじめ指定した犯情の特に軽微な成人による事件について、司法警察官が検察に装置しない手続きを取ることを示す。つまり、前歴はつくものの、前科はつかないため、微罪になることによって相当不利益は小さくなる。

微罪と認定されるためには犯罪が軽微であること、被害回復がされていること、犯行態様が悪質でないこと、被害者が処罰を望んでいないこと、初犯であること、監督者がいること、反省していることなどの項目を全て満たす必要がある。よって、DNAを採取される人物が再犯をする可能性は低いということができる。微罪の例としては被害が低額である万引きなどが挙げられる。

犯罪捜査において、DNAデータベースを利用することは効率の向上につながり、DNAデータベースの目的である犯罪捜査に資する。それなので、万引きのような微罪であっても事件の捜査に必要な限りでの利用であれば許容されるべきと考えることができる。しかし、微罪を重要犯罪や重要犯罪に該当しない刑法犯と比べたとき、自己情報コントロール権は再犯防止目的が少なくなるので比例的に高く保護されるべきだと考えられる。よって、保管期間などを定めた削除規定または削除を求める権利の規定は最低限規定されるべきで、その期間は上記二つの類型よりも短くあるべきと考えることができる。

\subsubsection{無罪}

DNAを採取された人物が無罪だった場合、無罪が確定した時点でDNAデータベースに保管されたデータを削除する旨の規定が必要であると考えられる。無罪の場合を考えるにあたっては米国で2013年に下されたMaryland
v. Kingの判決が参考となる\footnote{森本直子「被逮捕者のDNA採取と修正4条」比較法学48巻2号。}。この判決は、重大犯罪の非逮捕者からのスワブによるDNA採取は合衆国憲法修正4条に違反するかどうかについて争うものであった。よって、主な争点はスワブによるDNAサンプルの採取が修正4条にいう「不合理な捜索」に当たるか、であったのだが、これを判断するにあたってまず修正4条が合憲であるかが問われた。そして本判決においてマリーランド州は合憲であると判断された。マリーランド州法は採取対象者として逮捕・起訴された暴力犯罪または不法目的侵入罪あるいはそれらの未遂罪による被疑者であることと明記しており、データベース入力のタイミングが罪状認否手続後となっており、被疑者が無罪となった場合、DNA型は自動的に削除されることとなっていることが理由となっている。

これを現行のDNA型記録取扱規則と比べた時、最大の違いはやはり無罪の場合の削除規定の存在の有無であると考えられる。自己情報コントロール権保護の観点から考えたとき、自動的とまではいかなくても、少なくとも無罪と認められた場合はデータベースから削除される旨の規定は最低限必要ではないだろうか。

以上で検討した通り、犯罪の程度や種類にかかわらず、保管の可否や保管期間などが再犯性などを加味した上で規定されていないという問題点が現行のDNA型記録取扱規則には存在しており、これらの要素を踏まえずに全犯罪の被疑者のDNA型記録をデータベース化することは比例原則に反すると考えられる。

\subsection{構造審査的観点からのアプローチ}

構造審査は、従来の自己情報コントロール権という発想からは個人情報の濫用や漏洩のリスクに適切に対処できないことから生まれた考え方である。具体的には、センシティブな情報ではないプライバシー概念情報の収集であっても、表面上掲げられる正当な保有・利用方法を担保する「構造」が備わっていなければ、プライバシー権を不当に侵害するという発想が生まれたということである。今回問題となっているDNAデータベースは情報システムであり、そのシステムやそれを取り巻く環境がしっかりと確立できているのかについての検討が必要となる。つまり、システム自体のセキュリティをはじめとして、データを収集する目的を明確化することやデータを扱う人間の行動に対する罰則を整備することなどが含まれる。この観点から考えるにあたって、現行のDNA型記録取扱規則では以下の3つが問題となる。

\subsubsection{DNAサンプルの破棄について}

現在、警察庁が発表しているリーフレット\footnote{警察庁丁鑑発第906号「DNA型データベースの抜本的拡充に向けた取組について」(2012)。}によると、DNAサンプルは非コード領域情報の取得後に破棄され、警察が保有しているデータはDNAの非コード領域のみとなっている。しかし、正式に発表されている資料が法的拘束力を持たないリーフレットということは問題である。さらに、万が一DNAサンプルが破棄されずに保管され続けている場合、そのサンプルを再び分析することで非コード領域以外のセンシティブな情報を取得できる可能性は非常に高いのだが、このような二次利用がされた場合の罰則規定などの処理が規定されていないことも問題だと考えられる。これに対しては、法的拘束力を持つ規定に記載するなど、DNAサンプルが必ず破棄されることまたは個人とは結びつかない形で保管する旨の規定を定めるべきである。

\subsubsection{目的外利用}

警察によるDNAの非コード領域の取得方法は時代とともに一番個人特定率が高いものに変化してきた。その中には親子関係が分かってしまう方法も存在した。親子関係の特定などの目的外利用は防ぐ必要があるが、現状では目的外利用についての定めはDNA型記録取扱規則では特に設けられていない。行政機関個人情報保護法4条において利用目的が明確であることは求められてはいるが、未だ曖昧さは残る。現在の調査方法から親子関係が分かってしまう可能性は極めて低いが、今後検査方法が変わる可能性、そしてそれによって個人を特定する以外の情報が判明してしまう可能性を考えると、目的外利用を防ぐために現在よりも明確な目的外利用を防ぐための規定が必要だと考えられる。

さらに、DNAについてはまだ不明なことが多く、研究も続けられているため、将来起こりうる新発見についても考慮する必要がある。例えば、現在でこそ非コード領域からはセンシティブな遺伝情報は解明できないとされているが、将来非コード領域からもこれらの情報の取得が可能になってそれが悪用される可能性は存在する。よって、将来的に非コード領域が自己情報コントロール権を大きく侵害するようなセンシティブな情報となった場合、どのような対応が取られるかについてあらかじめ規定しておく必要はあると考えられる。

\section{DNA型記録取扱規則の形式的正当性}

\subsection{問題状況}

ここまでで、DNA型記録取扱規則の実質的正当性、つまりDNA型記録取扱規則はプライバシー権(憲法13条後段)を侵害し違憲なのではないかということについて検討してきた。ここからは、DNA型記録取扱規則の形式的正当性について検討することとする。つまり、DNA型記録取扱規則で規定されている被疑者DNA型記録の作成、管理、運用は国会によって制定された法律ではなく行政機関が制定した規則を直接の根拠として行われているが、これは憲法上問題がないのかについて検討する。以上に記した問題を明らかにするために、以下では、次のような順序で検討を進める。

まず、DNA型記録取扱規則の法的位置づけ、つまりDNA型記録取扱規則は法律の委任を受けて制定されているのか、委任を受けて制定されているとして本規則はどのような委任を受けているのかについて確認し、その後、憲法上の形式的正当性についてどのような問題が生じるのか、本規則は憲法上の形式的正当性を有しているのかについて検討することとする。

\subsection{DNA型記録取扱規則の法的位置づけ}

まず、DNA型記録取扱規則の法的位置づけについて確認する。警察法81条は、「この法律に特別の定がある場合を除く外、この法律の実施のため必要な事項は、政令で定める。」と規定し警察法の実施に必要な事項を警察法施行令に委任している。警察法5条4項20号に「犯罪鑑識施設の維持管理その他犯罪鑑識に関すること。」と規定されていることから、本規則で規定している「被疑者DNA型記録等を組織的に作成し、管理し、及び運用するために必要な事項」(同規則1条)も「この法律の実施のため必要な事項」(警察法81条)に含まれると考えられる。警察法施行令13条1項は、「国家公安委員会が法第5条第4項の規定による管理に係る事務\ldots{}を行うために必要な手続その他の事項は、国家公安委員会規則で定める。」と規定し、警察法81条で警察法施行令に委任した事項をさらに国家公安委員会規則に再委任している。これを受けて、国家公安委員会規則として、DNA型記録取扱規則が定められている。以上より、DNA型記録取扱規則は、警察法の委任を受けた委任命令であることが分かる。

\subsection{警察法5条4項の性質}

まず、本件規則へ警察法施行令13条1項の再委任を通じて委任している警察法5条4項は法律としてどのような性質を有しているのかが問題となる。これについて、法律はその規定している内容の性質から3つに分類されると考えられている。一つ目は、組織規範である。具体例としては、「外務省の設置並びに任務及びこれを達成するため必要となる明確な範囲の所掌事務を定めるとともに、その所掌する行政事務を能率的に遂行するため必要な組織を定めることを目的とする」(外務省設置法1条)外務省設置法などの各省設置法や国家行政組織法などがある。組織規範がなぜ必要なのかということの説明としては二通りの説明が考えられている。一つ目は、憲法学的な説明である。日本国憲法下では、国民が主権者として憲法を制定し、立法権、行政権、司法権を誕生させ、それぞれ国会に立法権、内閣に行政権、裁判所に行政権を委任したことになる\footnote{高田篤「法律事項」小山剛ほか編『論点探求憲法』(弘文堂,2013年)287頁。}。この行政権は、明治憲法下の行政権のように天皇があらかじめ持っていたというものではなく、憲法の制定によって創設されたものである。よって、明治憲法下のように原則的に自由に行政を行い、例外的に法律に制限されるというものではなく、本来的に自由に行うことができないものである。つまり、内閣が行政権を行使する場合には、憲法上明記されている場合(日本国憲法73条各号)を除いて、主権者たる国民の代表者である国会議員の定める法律の授権が原則として必要となる。この法律が組織規範である。二つ目の考え方が、行政法学的な考え方である。行政権は内閣に属しているが、実際に行政活動を行うのは、自然人である\footnote{塩野宏『行政法Ⅰ〔第六版〕行政法総論』(有斐閣,2020年)81頁。}。しかし、自然人が自然人として行動するのでは、行政活動ということはできない。行政活動といえるためには、つまり自然人が行った行為の効果を国に帰属させるためには、自然人が行政主体の機関である行政機関として行動しなければならない。もっとも、実際には自然人は一人ではないことから行政機関も複数存在することになる。よって、国家の行う事務を行政機関に分配する規律が必要であり、その規律の範囲内で行った行為については、行政主体たる国家に帰属すると考えられる。ここでいう規律が組織規範に当たる。この考え方は、民法における代理権の授与に似ているように思われる。実態が存在していない国家が実際に行政活動を行うことができないため、自然人である行政機関を代理人として、組織規範で規定されている内容についてのみ代理権が与えられる。その代理権の範囲内で行われた行為についてのみその行為の効果が国会に帰属するという考え方である。

二つ目は、規制規範である。規制規範というのは、「ある行政活動をある行政機関がなしうることを前提として、その適正を図るため」\footnote{塩野・前掲注27)82頁。}の規律のことをいう。具体例としては、「行政運営における公正の確保と透明性\ldots{}の向上を図」ること(行政手続法1条)を目的とした行政手続法や「補助金等の交付の不正な申請及び補助金等の不正な使用の防止その他補助金等に係る予算の執行並びに補助金等の交付の決定の適正化を図ることを目的とする」(補助金適正化法1条)補助金適正化法が挙げられる。

三つ目は、根拠規範である。根拠規範とは、「ある行政活動を行うのに組織規範が存在するとして、さらにこれに加えて、その行為をするに際して特別に根拠となる規範」\footnote{塩野・前掲注27)82頁。}のことをいう。根拠規範がなぜ必要なのかということについては以下の説明が挙げられる。民主主義が導入された日本国憲法においては、前文で、「国政は、国民の厳粛な信託によるものであつて、その権威は国民に由来し、その権力は国民の代表者がこれを行使し、その福利は国民がこれを享受する」と規定されているように、民主主義的正統性が要請されている。この民主主義的正統性には、二つの意味がある。一つ目は、「人的正統性」である\footnote{高田・前掲注26)287頁。}。人的正統性とは、すべての権力は国民に由来するというものである。これは、国家権力を行使する公務員は直接的に又は間接的に国民によって選ばれている(憲法15条1項、3項、43条1項、67条、68条、6条2項、79条1項、80条1項)ことによって、担保されている。組織規範の制定により行政機関に権限を付与することは、人的正統性の要請によるものである。二つ目は、「内容的正統性」である\footnote{ 高田・前掲注26)288頁。}。つまり、国家権力の行使である行政権の行使については、その内容は国民が決定したものである必要があるということである。この内容的正統性の要請から、原則として、行政権の行使については、その内容は主権者たる国民の代表者である国会議員によって構成された国会で定める法律で決めなければならないと考えられている。ここでいう法律のことを根拠規範という。しかし、この考え方を貫徹すると、すべての行政権の行使の内容に根拠規範が必要となってしまうが、これでは「法律の留保」論における全部留保説という考え方に対する批判が当てはまり、妥当ではないと考えられる。

ここで、本規則に警察法施行令の再委任を通じて委任した警察法5条4項は組織規範、規制規範、根拠規範のいずれであるかを検討する。警察法1条には、警察法の目的が規定されているが、そこには「この法律は、個人の権利と自由を保護し、公共の安全と秩序を維持するため、民主的理念を基調とする警察の管理と運営を保障し、且つ、能率的にその任務を遂行するに足る警察の組織を定めることを目的とする。」と書かれている。これにより、警察法は「警察の組織」を定めた法律であることがわかる。また、警察法5条の見出しは「任務及び所掌事務」であることからも、警察法5条4項が、国家公安委員会に与えられた権限についての規定であることがわかる。よって、警察法5条4項は組織規範であるといえる。

\subsection{被疑者DNA型情報の収集・保管・運用を行うのに根拠規範が必要か}

次に問題となるのは、警察法5条4項が組織規範であるとして、本規則で規定されている被疑者DNA型記録の作成、管理、運用に根拠規範、つまり特別に根拠となる法律が必要であるのかということである。この論点には、二つのアプローチがあると思われる。一つ目は、「実質的な意味での立法」とは何かということに着目する考え方である。憲法41条は、「国会は、国権の最高機関であつて、国の唯一の立法機関である。」と規定し、国会が立法権を有していることを定めている。この立法権とは、言葉の通り、「法律を定立する権限」のことであるが、ここでいう「法律」の意義について議論が蓄積されている。一つ目は、対象に着目したものであり、「法律」とは、「権利を制限し義務を課す規範」とする学説である\footnote{佐藤・前掲注13)473頁。}。二つ目は、性質に着目したものであり、「一般的抽象的法規範」とする学説である\footnote{佐藤・前掲注13)473頁。}。しかし、それぞれには批判もある。一つ目は実態に合っていないという批判である\footnote{佐藤・前掲注13)474頁。}。現在、国会で定められている「法律」には、権利を制限し、義務を課す規範」ではないもののみを内容とする法律が多く制定されており、議員や委員会が作成した法律である議員立法の多くはそのような法律である。また、「一般的抽象的規範」を「法律」とする学説に対しても、成田新法と呼ばれる新東京国際空港の安全確保に関する緊急措置法や各省設置法、日本銀行法などの特別の事柄に対する具体的な内容を規定した法律が存在する。二つ目は、比較法的観点による批判である\footnote{毛利透「23 法律の概念、個別的法律」曽我部真裕ほか編『憲法論点教室 第2版』(日本評論社,2020年)174頁。}。ドイツでは、立法権における「法律」とはどのようなものを指すのかという議論は、第一次世界大戦前のドイツのような立憲君主制下では妥当するものではあるが、民主主義が導入された現在のドイツでは妥当しないと考えられている。つまり、立憲君主制下では、原則、立法権及び司法権以外のすべての権限を有する君主は自由に行政を行うことができ、例外的に議会の定める法律がなければできない行政活動というものが存在するとされていた。そのため、そのような行政活動の範囲を定める「法律」について議論することは有益であった。しかし、民主主義が導入された現在のドイツでは、このような議論は有益ではないとされている。よって、現在のドイツでは「法律」は形式的な意味での法律、つまり議会手続を経て成立した法規範のことを法律と考えればよいとされている。このような批判から、実質的な意味での法律という概念は不要であるとし、法律は憲法に次ぐ効力を有する規範とすればよいという学説も有力であるとされている。現在の法律の制定状況や国家体制からこのように考えるのが妥当であると思われる。よって、「実質的意味での法律」の概念から特別に根拠となる法律が必要な範囲を確定するのは妥当ではないと考えられる。

二つ目は、法律による行政の原理に着目する考え方である。法律による行政の原理とは、言葉通り、行政は法律に従わなければならないという原理\footnote{塩野・前掲注27)77頁。}のことである。この原理は、憲法上明文の規定はないが、権力分立主義の当然の帰結とされており、日本国憲法においても、立法権(憲法41条)、行政権(憲法65条)、司法権(憲法76条1項)のように三権分立がなされていることから妥当する原理であるといえる。さらに、法律による行政の原理は、「法律の法規創造力」、「法律の優位」、「法律の留保」に分けられ、特別に根拠となる法律が必要となる範囲の確定において重要なのは、法律の留保である。

「法律の留保」とは、行政行為の全てにおいて法律に従わなければならないというわけではなく、特定の範囲に属する行政行為のみ法律の根拠を必要とする考え方である\footnote{塩野・前掲注27)80頁。}。この考え方は、先ほどから述べられている通り、立憲君主制下では、君主は原則として自由に行政を行うことができ、例外的に権利保障という目的から特定の範囲に属する行政行為については議会で定める法律によらなければ行うことができないという考えに基づいている。

注意が必要なのは、行政法学で論じられている「法律の留保」と憲法学で論じられている「法律の留保」とは少し意味合いが変わってくるということである。行政法学で論じられている「法律の留保」(Vorbehalt
des
Gesetzes)とは、先ほど述べたように、特定の範囲に属する行政行為には特別の法律の根拠を必要とする考え方であり、その特定の範囲に属する行政行為とは何かということを行政法学では議論していた。これに対して、憲法学で論じられている「法律の留保」(Gesetzesvorbehalt)とは、憲法上の権利保障の内容が憲法自体によってではなく法律の規定によって定まるような権利保障のあり方を意味している\footnote{亘理格「法律の密度と委任命令」法学教室323号 (2007年)58頁。}。このような考え方は、憲法が法律よりも優位性を有すようになったワイマール憲法に「法律の範囲内」で保障する権利について定めた条項が多く存在しており、このような条項では、憲法で保障された権利を、憲法自らが法律で保障される権利に貶めていることを批判したものである。大日本帝国憲法にもこのような条項が多く存在していたことから、このような憲法を批判するために憲法学において言及されてきた。

「法律の留保」によって、法律の根拠が必要な行政行為の範囲が問題となる。「法律の留保」が論じられ始めた時代には、この範囲は、国民の自由と財産を侵害する行為に限られていた。これは、自由主義的思想に基づいているとされている。自由主義において最も重要な自由と財産を侵害する場合には、侵害される当事者である国民の代表者である議員を構成員とする議会が定める法律がなければならないと考えられた\footnote{塩野・前掲注27)84頁。}。また、自由主義において国家は安全保障や治安維持などの必要最小限度の行政行為のみを行い、それ以外の行政行為は積極的に行われなかった。この考え方は、侵害留保説と呼ばれ現在でも立法実務や学説で支配的であるといえる。しかし、日本国憲法は自由主義に加え、民主主義も重要な価値であり、さらに行政活動の態様も多様化してきたことから、自由と財産を侵害する行政活動に対してのみ根拠となる法律が必要であるという考え方は妥当しないのではないかと考えられている\footnote{塩野・前掲注27)85頁。}。

そのため、現在では、法律の根拠が必要な法律の範囲について多くの学説が存在する。一つ目は、民主主義の正統性の観点から、すべての行政活動において、根拠となる法律が必要であると考えるものであり、全部留保説と呼ばれる。この考え方では、行政活動を行うときには常に法律の根拠が必要であることから、状況の変化に柔軟に対応できなかったり、対応しようとすると、行政庁が定める法規である命令に包括的に委任することになったりするため妥当ではないと考えられている。また、民主的なコントロールの観点から侵害留保説では不十分であるという方向に働くのは確かであるが、国会が組織規範を定め行政機関の権限を設定した段階で民主主義の正統性は一定程度担保されているのではないかとも考えられている\footnote{藤田宙靖『行政法総論』(青林書院,2013年)86頁。}。よって、民主主義の正統性から直ちに、すべての行為に組織規範だけでなく根拠規範まで必要であると考えるのは妥当ではないと考えられる。

二つ目は、現在では、国家は必要最小限度の行為を行うのではなく、社会保障等の行政活動をも行っていることから、自由と財産を侵害する行政活動に加え、権利や利益を付与する行為に対しても法律の根拠が必要であるとする考え方であり、社会留保論と呼ばれている。この考え方に対しては、すべての権利や利益を付与する行為について法律の根拠が必要であるとするのか、すべてには必要としないとする場合には、どのように線引きするのかが不明確である\footnote{前掲27)85頁。}という批判があるのに加え、そのような給付行政については、その財源である予算について国会の承認が必要であることから、これにより民主主義によるコントロールは行うことができ、これで十分なのではないかと思われる。

三つ目は、行政権は主権者たる国民に優越することはないということから、権力的行為形式についての全ての行為について根拠となる法律が必要であるという考え方があり、これを権力留保理論という。しかし、これについては、補助金の交付などは規制規範によって根拠づけられているとすると、必ずしも根拠規範を必要としないのではないかという批判がある。さらに、この権力留保理論では、行政指導、行政計画、民法上の契約などの強制力のない形式で行われる行政活動については権力的行為形式で行われる行政行為ではないため、根拠となる法律は必要ではないため、侵害留保説の代替手段にはなりえないのではないかという批判もある。

四つ目は、ドイツにおける判例理論となっている本質性理論と呼ばれるものである。これは、行政活動について、本質的な事項は議会が決定すべきであり、行政に委ねてはいけないという考え方\footnote{前掲注27)88頁。}であり、行政について民主的な統制を図りつつ、本質的でない部分については民主的な統制は必要ではないという考え方である。この考え方は、立憲主義と民主主義という憲法における重要な価値を尊重するものであるように思える。しかし、問題となるのは本質的な事項とは何かということである。侵害留保説においても本質的な事項が自由と財産であると考えると、本質性理論と両立するものであるといえる。これに加え、現代の実情に合わせて、本質的な事項を決定することができるという点では、より妥当な考え方であるとも思える。しかし、その本質的な事項を誰が決定するのかが問題となる。ドイツでは、本質的な事項の決定は憲法裁判所が行っている。もっとも、これを日本にそのまま導入するのは妥当ではないように思われる。そもそも、民主主義の価値を尊重するための考え方であるのに、その本質的な事項の決定を国民から直接選ばれておらず、民主的な人的正統性が国会よりも劣る裁判所が決定することは妥当ではない。また、ドイツでは抽象的違憲審査制を採り積極的に憲法的な事項を判断しているため、本質的な事項の範囲を決定することができているとも思える。しかし、日本の最高裁判所は憲法にかかわる事項についての判断に謙抑的であることから、裁判所がその役割を担ったとしても、あまりうまく機能しないと思われる。

以上のように、行政行為を行うのに特別に根拠となる法律が必要な範囲についての学説を見てきたが、この議論で重要なのは理論的整合性と実務における妥当性のバランスであるように思われる。結論が正しくても理論的根拠が乏しいものは妥当ではないが、理論的に正しくても実際の行政活動がうまく機能しなかったり実際に行政活動を行う行政機関等の判断の基準として明確でなかったりするものも妥当ではない。これらのことを考慮すると、侵害留保説が妥当ではないかと思われる。自由主義の観点から、自由と財産を侵害する行為について法律の根拠を必要としている。また、民主的な正当性の観点からは、それ以外の行為についても組織規範や規制規範が存在していることから、一定程度は民主的正統性が担保されているように思われる。また、民主的コントロールの観点においても、給付金の交付などについては、その財源となる予算について国会の承認があることから、一定程度は民主的統制がとられているように思われる。実務における妥当性については、基準としての明確性という観点において、他の学説よりも妥当であるように思われる。以上より、以下の本規則についての検討においては、侵害留保説に則って検討し、本規則で定められている被疑者DNA型情報の収集、保管、運用が権利、財産を侵害する行為であるかについて検討する。

本規則で定められた事項が「国民の権利、財産の侵害」に当たる場合には、根拠となる法律が必要であるといえ、組織規範で規定されているにすぎない本規則は「法律の留保」に反するといえる。よって、本規則で定められている被疑者DNA型情報の収集、保管、運用が「国民の権利、財産を侵害」しているかについて検討する。前述のとおり、本規則は被疑者DNA型情報の収集と管理運用を分けて考えられる。収集については、類推適用ではあるが、刑事訴訟法から根拠を導き出すことができるため、問題にはならない。問題となるのは、被疑者DNA型情報の保管、運用である。これらの行為はDNA型情報の再収集であると考えられる。ここで、参考となるのは、前述の構造審査である。構造審査は、住基ネット判決の判断基準と言われており、個人情報を保管、利用するデータベースの合憲性について比例原則に加え、システムに適正な保管、利用を担保する「構造」が備わっていなければならないという考え方であり、その「構造」の要件として、①システム上の欠陥がないこと、②目的外利用や秘密の漏洩等に対して、懲戒処分や刑罰によって禁止されていること、③根拠法律等によって、第三者機関等の設置など、本人確認情報の適切な利用を担保するなどの措置が取られていることの全てが満たされなければならないとしている。住基ネットでは、これらの要件が満たされていたため、権利侵害の現実的な危険性がないとされている。しかし、①~③の要件が満たされないシステムを利用して個人情報の収集、保管、運用を行っていた場合には、権利侵害の現実的な危険があるといえるため、このような制度は、「個人に関する情報をみだりに第三者に開示または公表されない自由」を「侵害」するおそれが内在しているといえる。

よって、被疑者DNA型情報の保管、運用も「法律事項」であるといえ、根拠規範が必要である。

\section{制度設計}

\subsection{制度設計の観点}

日本のDNAデータベース制度の問題点として、比例原則の観点からは、対象犯罪の限定がないこと、無罪となった場合の抹消規定がないことが、構造審査の観点からは、DNAサンプルの保存方法、目的外利用の禁止、第三者機関についての定めがないことが挙げられる。そこで、本章では、これらの二つの観点からDNAデータベースの制度設計について考察することとする。その際、ドイツ、アメリカ、カナダ等の諸外国の制度設計やその問題点について概観を得た上で、日本のDNAデータベース制度の現状との比較を行うことを通じて、比較法的観点からもDNAデータベースの制度設計について検討を行う。

\subsection{比例原則の観点から}

\subsubsection{比例原則に適合する制度とは}

比例原則とは、目的の重要性と制約の重さとの間に均衡が保たれていることを要求する原則である。日本のDNAデータベースの現状としては、対象犯罪の規定がなく、無罪となった場合の抹消規定がない\footnote{山本龍彦「日本におけるDNAデータベース法制と憲法」比較法研究70号(2008年)74頁、77頁。}ため、制約が目的の重要性と比べて重く、比例原則に反する状態になっている。そこで、比例原則に適合する制度にするためには、制約の程度を低くする必要がある。

\subsubsection{ドイツの理論}

ドイツのDNAデータベースは、連邦刑事局、連邦警察、州警察、税関を結ぶ形で、総合的に運用がされている。対象犯罪は「重大な犯罪」または性的自己決定を害する犯罪とされており\footnote{ドイツ刑事訴訟法81g条1項参照。}、データベースへの登録に際しては、裁判官が「再犯危険性の推定」を基準に判断するという仕組みが取られている。

DNAデータベースへの登録の対象犯罪として規定されている「重大な犯罪」とは、ドイツの他の刑事訴訟法上の規定でも用いられる概念であり、2000年連邦最高裁判所決定では、「少なくとも中程度の犯罪の領域に属し、法的安定性を著しく阻害し、国民の法的安定についての感情を著しく侵害するに十分なもの」\footnote{BVerfGE
  103,21 (33f.)。}とされている。2003年の改正で、露出行為、ポルノグラフ文書の配布などの「性的自己決定に対する犯罪」も対象犯罪とされたほか、累犯者についても、個々の犯罪が全体として評価され、その不法性が重大とみなされた場合にデータベース登録の対象となるなど、対象犯罪の拡大傾向があることが指摘されている\footnote{玉蟲由樹「ドイツのDNAデータベース法制」比較法研究70号(2008年)50頁。}。

「再犯危険性の推定」とは、特定の事実(犯罪の種類もしくは態様、被疑者・被告人の人格その他の判断)にもとづいた、将来的に重大犯罪を理由とするさらなる刑事訴訟が行われるとの予測のことである\footnote{玉蟲由樹「刑事手続におけるDNA鑑定の利用と人権論(1)」福岡大学法学論叢52巻2・3号(2006年)342頁。}。よって、被告人が無罪判決を受けた場合などは、「再犯危険性の推定」がないと言えるため、記録は抹消されるようになっている。この規定は、個別の事案ごとに裁判官が判断するという仕組みを定めたものであり、制約の程度を低くするものである。

\subsubsection{日本における制度設計}

ここでは、前項で見たドイツの制度設計を踏まえて、日本のDNAデータベースの制度設計のあり方を検討する。

日本の現状は、対象犯罪については、捜査の過程で得られたDNA型情報を登録する形式であるため、実質的に限定がないと言える\footnote{山本・前掲注44)75頁。}。そこで、ドイツでの議論を参考に、「重大な犯罪」といった形で登録の対象となる犯罪に限定をかけることが考えられる。どの範囲の犯罪を対象とするかについては検討の余地があるものの、少なくとも、微罪や無罪の場合にはDNA型情報を登録しない、という制度にすることが望ましいと考えられる。

また、登録要件についても、現状としては、同意による取得の場合は要件なし、令状による取得の場合は裁判官が「当該事件の捜査での必要性」の判断を行うという仕組みになっている\footnote{山本・前掲注44)75頁。}。すなわち、当該事件の捜査で必要であると裁判官が認められれば、その判断をもって、当該事件の捜査とはもはや関係を持たない、DNAデータベースへの登録が認められるようになっているのである。ここで生じうる問題として、裁判官がDNAデータベースへの登録の必要性を判断するという仕組みになっていないため、警察内部で恣意的な登録が行われる可能性があることが指摘できる。よって、ドイツでの制度設計を参考に、裁判官が、将来の再犯危険性を判断してデータベースの登録の必要性を決める仕組みを採用すべきである。

抹消規定として、被疑者DNA型記録に係る者の死亡、被疑者DNA型記録を保管する必要性の消滅しか規定されていないため、無罪となった場合にもDNA型記録が抹消されるか不明であることも、日本の現状の問題点として挙げられる\footnote{山本・前掲注44)77頁。}。この点については、前述のように、将来の再犯危険性を裁判官が判断する制度を構築することで、嫌疑が晴れた場合、あるいは無罪判決が出た場合に、再犯可能性の推定がないとして、DNA型記録のデータベースからの抹消が認められるようになる。

\subsubsection{小括}

ここまでの議論をまとめると、比例原則の観点からは、ドイツでの議論を参考に、登録対象から少なくとも微罪と無罪は外すこと、「再犯危険性の推定」を裁判官が判断する仕組みにすること、嫌疑が晴れるか、無罪判決が出た場合には、データベースからの抹消が認められることが制度設計として考えられる。

\subsection{構造審査の観点から}

\subsubsection{構造審査とは}

構造審査とは、当該システムが情報漏洩や目的外利用等による実害発生を有効に防ぎうる強固な構造を有しているかを問う判断枠組みのことである。住基ネット訴訟判決\footnote{最判平成20年3月6日判タ1268号110頁。}では、①システム上の欠陥がないこと、②目的外利用や秘密の漏洩等に対して、懲戒処分や刑罰によって禁止されていること、③根拠法律等によって、第三者機関の設置など、本人確認情報の適切な利用を担保するなどの措置が取られること、これら①から③までの観点などから構造を審査し、個人に関する情報をみだりに第三者に開示又は公表されない自由は侵害されていないとの結論を導いている。ここでは、これらの①から③の観点から、日本のDNAデータベースの制度設計について考察を行う。

\subsubsection{DNAサンプルの取り扱いについて}

前述のように、日本のDNAデータベースの現状として、DNA型記録取扱規則上、DNAサンプルの保管についての規定がないため、DNAサンプルが分析後も保管されるのか、また、保管されるとすれば、その保管方法はいかなるものか、規則からは不明となっている\footnote{山本・前掲注44)76頁。}。規則とは別に、警察庁が発表しているリーフレットでは、「現場資料と比較対照するために被疑者の身体から採取された血液等の鑑定後に生じた残余については破棄」する胸が記載されている\footnote{警察庁『DNA型情報の活用に向けて』6頁。}が、リーフレットは法的拘束力を持つものではないため、構造審査の①の観点、すなわち「システム上の欠陥がないこと」に照らして、法的拘束力を持つ別途の定めが必要であると言える。

DNAサンプルの取扱いについて、諸外国の規定を見る。アメリカでは、ローランド州、オレゴン州といった一部の州を除いて、多くの州ではサンプルそれ自体の保護や利用について明確な規定は置かれておらず、ウィスコンシン州のみが、DNA分析後のサンプルの破棄についての定めを置いている\footnote{山本龍彦「アメリカにおけるDNAデータベース法制」比較法研究70号(2008年)24頁。}。カナダでは、1988年に成立したDNA鑑定法によって、完全釈放から1年以内、又は仮釈放から3年以内には、DNAサンプルが破棄されることが規定されている\footnote{日弁連・前掲注8)25頁。}。

\subsubsection{目的外利用の防止について}

DNA型記録取扱規則上は、利用目的を限定する条項は存在せず、目的外利用を禁止する規定はない\footnote{山本・前掲注44)76頁。}。行政機関の保有する個人情報の保護に関する法律では、目的外利用、提供の禁止(同法8条1項)、行政機関の職員等による濫用的な保有個人情報の提供、利用等への罰則規定(同法53条から57条まで)が設けられている。DNAデータベースを所管する警察庁は、同法における行政機関に、データベースに保存されるDNA型情報は同法における個人情報に該当するため、同法によって目的外利用の防止が十分に図られているように見える。しかし、DNA型記録取扱規則上の目的は、「犯罪捜査に資すること」(同規則1条)とされており、目的規定として曖昧なものとなっている。そのため、行政機関の保有する個人情報の保護に関する法律によって目的外利用を防止する規定が設けられていても、目的規定が曖昧なために、目的外利用の防止効果が十分に発揮されない可能性がある。そこで、構造審査の②の観点、すなわち「目的外利用や秘密の漏洩等に対して、懲戒処分や刑罰によって禁止されていること」に照らして、目的の範囲を明確に限定することが構造審査に適合した制度と言えるために必要不可欠な要件であると考える。

目的外利用の防止について、諸外国の規定を見る。ドイツでは、「将来の重大犯罪の訴追のための準備」がDNAデータベースの目的となっており\footnote{玉蟲・前掲注47)46頁。}、DNAデータの取得を被疑者・被告人の「再犯危険性の推定」と結びつける規定と密接な関係性を有するものとなっている\footnote{玉蟲・前掲注47)49頁。}。アメリカでは、連邦法上は、捜査、裁判、刑事防御、データベース構築・プロトコル作成・品質保証が、DNA分析の目的として限定されており、このような目的から外れた利用がされた場合には、罰則が設けられている。州レベルでは、例えば、ニューヨーク州において、身元不明の遺体の識別または犯罪捜査上の個人識別、被告人の防御、人口統計データベース維持に目的を限定し、目的外利用に対しては罰則が設けられている\footnote{山本・前掲注55)25頁。}。このように、ドイツやアメリカにおいては、目的規定の具体化と、罰則規定とが相互に連関しあって、目的外利用の防止を実現しようとするものとなっている。

\subsubsection{第三者機関の設置について}

日本におけるDNAデータベースの運用においては、警察内部での運用をチェックする第三者機関が設置されていない。そのため、警察内部での運用が不透明なものとなる危険性がある。構造審査の③の観点、すなわち「根拠法律等によって、第三者機関の設置など、本人確認情報の適切な利用を担保するなどの措置が取られること」に照らして、警察内部での運用が適切になされているかを監査する機関の設置が、構造審査に適合するためには求められる。

ここで、DNAデータベースの運用に関して第三者機関を設置している具体例として、カナダを取り上げる。カナダにおいては、独立の外部評価機関が諮問委員会の形で設置され、データベース運用の中での、サンプル処理や分析方法といった科学的問題や、法的倫理面での問題について検討し、報告するという制度が取られている\footnote{日弁連・前掲注8)25頁。}。

\subsection{小括}

以上より、構造審査的観点から、DNAサンプルの取り扱いについて、吐きの有無、保存方法についての法的根拠が必要であること、目的外利用について、目的を明確に限定することで目的外利用の防止規定の実効性を上げること、第三者機関の設置について、カナダのように、データベースの運用を監査する第三者機関を設けることが必要であると言える。

\section{結び}

本稿では、DNAデータベース制度の合憲性について検討を行った。国家公安委員会規則であるDNA型記録取扱規則に基づいて運営されている現在のDNAデータベースは形式的正当性、実質的正当性ともに憲法上問題があるという結論になった。また、諸外国の制度との比較から憲法に適合する制度についても検討を行った。DNA型記録取扱規則について、平成22年4月27日の法務委員会で千葉景子法相が、「法律で定める必要があるかどうかということは、その必要性も含めて、運用の実情等をかんがみながら、これも検討すべき課題ではあろうかというふうに思います。」と発言していたが、当時と比べて登録件数が増加し、捜査機関においてもその重要性が高まっている現状にかんがみると、今こそ法律で定めるべきか、定めるとしてその内容をいかにすべきかについて検討すべきときではないかと思われる。

\section{参考文献}

・朝日新聞「犬探して貼り紙\ldots{}それだけでDNA『一生、容疑者扱い』」(2020年8月24日記事)。

・朝日新聞「容疑者の多くからDNA採取 DBに130万件と判明」(2020年8月23日記事)。

・日本経済新聞「警察のDNAデータ増大 登録130万件 身元確認の切札 保管・抹消の法整備課題」(2020年11月10日記事)。

・末井誠史「DNA型データベースをめぐる論点」レファレンス
61(3)(2011年3月)。

・玉蟲由樹「警察DNAデータベースの合憲性」日本法学82(2)(2016年10月)。

・日本弁護士連合会「警察庁 DNA
型データベース・システムに関する意見書」(2007年12月)。

・憲法判例研究会編『判例プラクティス憲法〔増補版〕』(信山社、2016年)。

・佐藤幸治『日本国憲法論』(成文堂、2020)。

・山本龍彦『遺伝情報と法政策』(成文堂、2007年)。

・伊藤栄樹ほか『注釈刑事訴訟法(新版)第3巻』(立花書房、1996年)。

・水野陽一「刑事手続における強制採血とDNA型鑑定に関する一考察」広島法学36巻2号(2012年)。

・奥平康弘『憲法Ⅲ 憲法が保障する権利』(有斐閣、1993年)297-301頁。

・玉蟲由樹「平等取扱原則と比例性」日本法学85巻2号(2019年9月)。

・法務省(2015)『平成27年度犯罪白書』。

・法務省(2020)『令和2年年度犯罪白書』。

・森本直子「被逮捕者のDNA採取と修正4条」比較法学48巻2号。

・警察庁丁鑑発第906号「DNA型データベースの抜本的拡充に向けた取組について」(2012)。

・高田篤「法律事項」小山剛ほか編『論点探求憲法』(弘文堂,2013年)。

・塩野宏『行政法Ⅰ〔第六版〕行政法総論』(有斐閣,2020年)。

・亘理格「法律の密度と委任命令」法学教室323号(2007年)。

・藤田宙靖『行政法総論』(青林書院,2013年)。

・山本龍彦「日本におけるDNAデータベース法制と憲法」比較法研究70号(2008年)。

・玉蟲由樹「ドイツのDNAデータベース法制」比較法研究70号(2008年)。

・玉蟲由樹「刑事手続におけるDNA鑑定の利用と人権論(1)」福岡大学法学論叢52巻2・3号(2006年)。

・警察庁『DNA型情報の活用に向けて』。

・山本龍彦「アメリカにおけるDNAデータベース法制」比較法研究70号(2008年)。
