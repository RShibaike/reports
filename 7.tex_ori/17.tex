\section{はじめに}

集団的自衛権は数年前に話題となったが、そのはるか前から議論されてきた。そして今でも議論は続いており、宇宙空間は集団的自衛権の適用範囲とされ\footnote{「宇宙でも集団的自衛権 防衛相が見解」日本経済新聞2019年10月16日。}、サイバー空間でも自衛権を行使できる国際的ルール作りが始まっている\footnote{「自衛権の行使
  サイバー空間も」日本経済新聞2020年6月12日。}。また、2019年4月22日に札幌地裁で、2019年11月7日に東京地裁で、2020年1月28日に大阪地裁で、\href{http://www.asahi.com/topics/word/\%E9\%9B\%86\%E5\%9B\%A3\%E7\%9A\%84\%E8\%87\%AA\%E8\%A1\%9B\%E6\%A8\%A9.html}{集団的自衛権}の行使を認めた\href{http://www.asahi.com/topics/word/\%E5\%AE\%89\%E5\%85\%A8\%E4\%BF\%9D\%E9\%9A\%9C\%E9\%96\%A2\%E9\%80\%A3\%E6\%B3\%95.html}{安全保障関連法}が\href{http://www.asahi.com/topics/word/\%E6\%86\%B2\%E6\%B3\%95.html}{憲法}に違反するか争われた集団訴訟の判決が次々と出されている。

札幌地裁判決では、「憲法前文は、憲法の『崇高な理想と目的』(前文第4項)を示すものであり、憲法における解釈指針や憲法第3章で定める人権規定においてその趣旨が斟酌されることがあっても、前文に上記文言があることから直ちに国民に『平和のうちに生存する権利』が具体的な権利として保障されているものと解することはできない」と述べた上で、「平和とは理念ないし目的としての抽象的概念であって、その具体的に意味するところは、各人の思想や信条、世界観等の主観によって異なるものであり、これを達成ないし確保する手段も、他者との関係を含めて達成し得るものであって、その当時の国際情勢によっても左右されるところが大きいのであるから、『平和的生存権』の具体的な内容について一義的に確定することは困難であるといわざるを得」ず、「平和的生存権は、法律上保護された具体的な権利ないし利益であるとはいえない」。また、「憲法9条は国の統治機構ないし統治活動についての基本的政策を明らかにしたものであるにすぎず、国民の私法上の権利義務を具体的に定めたものと解することはできない」とした\footnote{札幌地裁2019年4月22日。}。東京地裁\footnote{東京地裁2019年11月7日。}や大阪地裁\footnote{大阪地裁2020年1月28日。}の判決でも同様に、平和的生存権と憲法9条は法律上保護される権利又は利益に当たるということはできないと判示し、原告の主張を門前払いしている。

我が国における集団的自衛権とはどういったものか。憲法や国際法といった法律面から、我が国における集団的自衛権の根拠について論じていきたい。

\section{集団的自衛権の基本原理}

\subsection{国際法上認められた集団的自衛権について}

自衛権は、憲法9条だけでなく国連憲章2条4項により違法とされている武力行使の例外であり、国連憲章51条によって固有の権利として認められている。そして自衛権は個別的自衛権と集団的自衛権の2つに分類される。個別的自衛権は、武力攻撃を受けた国が自国を守るために自ら反撃する権利である。一方で、集団的自衛権は、直接攻撃を受けていない第三国が、武力攻撃を受けた国を助けるために反撃に加わる権利である。そして、第三国が集団的自衛権を行使するためには、国連憲章51条に示される「武力攻撃の発生」に加えて、武力攻撃の犠牲国が武力攻撃を受けたことを宣言し、当該第三国に援助を要請することが必要であるとICJニカラグア事件判決で示されている\footnote{Nicaragua
  Case paras. 195,199。}。

しかしながら、集団的自衛権はそのほとんどが軍事大国に援用されているのが実態である。他国を防衛・援助するという本来の趣旨とは異なる動機(アメリカであれば反米勢力の掃討、ソ連であれば自由化の弾圧)をはらんだ上で、これまで集団的自衛権は行使されてきたことに言及しておく\footnote{長谷部恭男・杉田敦編『安保法制の何が問題か』(岩波書店、2015年)102、103頁。}。

\subsection{我が国における従来の集団的自衛権に関する政府解釈}

このように国際法上認められている集団的自衛権であるが、従来、我が国ではその行使が認められてこなかった。1972年9月14日の参議院決算委員会にて、当時の内閣法制局長官の吉國一郎も「憲法が容認するものは、その国土を守るための最小限度の行為だ。したがって、国土を守るというためには、集団的自衛の行動というふうなものは当然許しておるところではない」\footnote{第69回国会参議院決算委員会閉会後第5号(1972年9月14日)11頁(吉國一郎)。}と述べている。また、我が国が集団的自衛権の行使を認めていないことは、この委員会における水口宏三議員(社会党)の要求に応じて1972年10月14日に政府が提出した資料\footnote{1972年10月14日参議院決算委員会資料「集団的自衛権と憲法との関係」

  \href{http://konishi-hiroyuki.jp/wp-content/uploads/04201-4.pdf}{{http://konishi-hiroyuki.jp/wp-content/uploads/04201-4.pdf}}(2020年12月29日確認)。}(以下「72年見解」と略)からも窺える。憲法9条は戦争を放棄し戦力の保持を禁止しているが、前文の平和的生存権や憲法13条の趣旨から、必要な自衛の措置をとることが禁じられているわけではない。しかし、自衛のための措置が無制限に認めているわけではないとしている。その上で、72年見解では例外的に我が国の憲法の下で武力行使が許される要件が示されている。それは、以下の3つである。

①我が国に対する急迫不正の侵害があること、すなわち武力攻撃が発生したこと

②これを排除するために他の適当な手段がないこと

③必要最小限度の実力行使にとどまるべきこと

のちに自衛権発動の3要件と呼ばれるものであり、集団的自衛権は、①の要件を満たさない。なぜなら、我が国以外の第三国が別の国から武力攻撃を加えられても、我が国の国民全体の生命等に危険が及ぶことはあり得ないとされるからである。

また、個別的自衛権と集団的自衛権は、「武力の行使」という点で共通するものの、自国防衛か他国防衛かという点で大きくその様態が異なる。72年見解で示された自衛権発動の3要件は、そもそも個別的自衛権を発動することを想定したものである以上、集団的自衛権をその基準に当てはめて議論すること自体を疑問視する声もある\footnote{大石眞「日本国憲法と集団的自衛権」ジュリスト1343号(2007年)44頁。}。

\subsection{2014年7月1日の閣議決定後の集団的自衛権に関する政府解釈}

ところが、2014年7月1日の閣議決定(以下「7・1決定」と略)により、政府はこれまでの主張を一変させ、集団的自衛権の行使は限定的に合憲であるとした(以下、7・1決定により合憲とされた限定的な集団的自衛権を「集団的自衛権」と略)。我が国を取り巻く安全保障環境が変化し続けている状況を鑑みて、外交努力や法整備などあらゆる必要な対応を採ったうえで、それでも我が国の存立が脅かされる危険がある場合において、自衛のための措置として、憲法上許容されるとしたのである。ここで注意すべきは、7・1決定で合憲とされた「集団的自衛権」の行使は、他国に対する武力攻撃の発生を契機とするものであっても、あくまで我が国を防衛するためのやむを得ない自衛の措置としてなされるものに限られ、国際法上認められる集団的自衛権の行使一般が容認されるわけではないとしていることである。

これにより、憲法9条のもとで許容される自衛権発動の新3要件は、個別的自衛権だけでなく「集団的自衛権」をも想定し、以下のようなものとなった。

\begin{quote}
①我が国に対する武力攻撃が発生したこと、又は我が国と密接な関係にある他国に対する武力攻撃が発生し、これにより我が国の存立が脅かされ、国民の生命、自由及び幸福追求の権利が根底から覆される明白な危険があること(これを存立危機事態と呼ぶ)
\end{quote}

②これを排除し、我が国の存立を全うし、国民を守るために他に適当な手段がないこと

③必要最小限度の実力行使にとどまること

\section{我が国における集団的自衛権の位置づけ}

\subsection{解釈変更の議論の出発点}

2015年6月4日の衆院憲法審査会にて、集団的自衛権の行使容認が憲法違反でないかという質問に対し、参考人として呼ばれていた早稲田大学教授の長谷部恭男は「憲法違反だ。従来の政府見解の基本的な論理の枠内では説明がつかない」と明言し、共に参考人として呼ばれていた慶応大学名誉教授の小林節と早稲田大学教授の笹田栄司もそろって違憲と判断した\footnote{「集団的自衛権行使、全参考人が『違憲』 衆院憲法審」日本経済新聞2015年6月4日。}。これが後に、集団的自衛権の議論を一変させたといわれている。一方、当時の内閣法制局長官であった横畠裕介は、「今般の閣議決定は、憲法第9条の下でも例外的に自衛のための武力の行使が許される場合があるとする1972年の政府見解の基本論理を維持し、その考え方を前提としております」と述べ、論理的整合性が保たれていることを強調した\footnote{第186回国会参議院予算委員会閉会後第1号(2014年7月15日)23頁(横畠裕介)。}。

一方で、まるで正しいことが前提であるかのような72年見解は、そもそも我が国の実情に即していたといえるのだろうか。

\subsection{従来の政府解釈と現実の乖離}

そもそも我が国において集団的自衛権に相当する又は相当しうる行動が一切なかったのか。

日本政府は1981年に「シーレーン1000海里防衛構想」を打ち出したり、2003年にイラク特措法を成立させたりして、後方支援を可能としてきた。武力行使と一体化しない限り、後方支援は集団的自衛権の行使に当たらないとしてきたのである。これらの支援活動は武力行使を伴っていない以上、「武力行使の禁止」の例外にあたる自衛権の発動とみるのは国際法上厳しいように思える。

では、1960年成立した「日本国とアメリカ合衆国との間の相互協力及び安全保障条約」(以下、新安保条約)に基づき、我が国が米軍基地を防衛することはかねてから集団的自衛権の行使にあたるのではないか、かねてから議論されてきた。京都大学教授で国際法を専門としていた田畑茂二郎は、新安保条約5条に基づき我が国が米軍基地防衛のためにとる行動は、アメリカに対する集団的自衛権に基づく措置とみるのが適当な場合も起こりうるであろうと主張した\footnote{田畑茂二郎『安保体制と自衛権』(甲陽社、1969年)97頁。}。日本人に被害が及ばないケースがあるからである。その一方で、同じく京都大学教授で国際法を専門とする浅田正彦は、そもそも日本領土が攻撃されているわけであるから、当然に個別的自衛権が行使できるとして、集団的自衛権の行使の問題にならないとしている\footnote{高橋和之・浅田正彦・安念潤司・五十嵐武士・山内敏弘「{[}座談会{]}憲法9条の過去・現在・未来」ジュリスト1260号(2004年)27頁。}。また、元内閣法制局長官の林修三も「日本の国内にある米軍の基地なら基地に攻撃を加えるということは、まさに私は日本の領土領海に対する侵略なくしてはあり得ないことだと思います」と述べ、集団的自衛権の行使の問題にならないとしている\footnote{第31回国会衆議院予算委員会第16号(1959年3月2日)16頁(林修三)。}。我が国の本土が攻撃されているにもかかわらず、日本人が攻撃されていない等の理由から放置するというのはありえない。個別的自衛権が自国の防衛を目的にする以上、米軍基地であれ日本本土が攻撃された場合に、その後に日本人が巻き込まれる危険性もあることからも、我が国が米軍基地を防衛することは個別的自衛権の行使にあたると考えるのが自然であろう。ただし、新安保条約に基づき米軍基地を防衛するための日本の行動を、アメリカは日本の集団的自衛権行使と理解していると元内閣法制局長官の高辻正巳は語り、日米間で理解に乖離がある\footnote{中村明『戦後政治にゆれた憲法九条〔第2版〕―内閣法制局の自信と強さ/「武力行使と一体化論」の総仕上げ』(中央経済社、2001年)187頁。}。

他方で、新安保条約6条に基づく米軍への基地提供はどうであろうか。国際法の観点から見ると、国連総会で1974年に可決された『侵略の定義に関する決議』(国連総会決議3314)第3条(f)、2002年に発効された『国際刑事裁判所に関するローマ規程』第8条の2第2項(f)は、「他国の使用に供した領域を、当該他国が第三国に対する侵略行為を行うために使用することを許容する国家の行為」を「侵略」と定義づけている\footnote{国連総会決議3314は、その第2条において、該当行為が侵略行為に当たるかどうかは実際には安保理事会が決めるということが示されており、「侵略」の定義は目安程度のものに過ぎないとされる。一方、『国際刑事裁判所に関するローマ規程』は安保理の認定について何らの言及もしておらず、「侵略」の認定を安保理の認定にかからしめていない。この点で、両者には違いがある。}。アメリカが他国を攻撃するために日本が米軍基地の使用許可を出すことはこれらの決議や条約に違反するため、ベトナム戦争などで米軍基地の使用を許可した日本の行為は「侵略」にあたる。日本本土が攻撃されているわけではない以上、この「侵略」を個別的自衛権の行使として解釈するのは無理がある。

ところで実際の政府の解釈はというと、1950年代末から1960年代にかけては、集団的自衛権行使の全面禁止論ではなく部分的容認論をとっており、あくまで「他国防衛」を禁止していた。当時の首相である岸信介は「集団自衛権については、学説上広い考え方もあるし、狭い考え方もある、しかし、日本の憲法で海外派兵はできない、しかし、基地の提供とか、あるいは経済援助とかいうようなことは当然できるのだ、これを、自衛権というものを広く解釈するところの学説をとれば、そういうものは集団的自衛権で解釈できるだろうし、それをしなくても、日本がそういうことはできる権利を持っていることは当然だ」と答弁しており、集団的自衛権の行使の全面禁止を明言していない\footnote{第34回国会衆議院日米安全保障条約等特別委員会第21号(1960年4月20日)39頁(岸信介)。}。しかし1960年代半ばになると、ベトナム戦争の参戦可能性や1969年の沖縄返還時の佐藤・ニクソン共同声に含まれた韓国条項により、自衛隊派兵の可能性が高まり始めた。こうして侵略戦争への加担が現実味を増していき、市民からの反発が巻き起こった。その批判を回避するという政局的な理由により、政府は集団的自衛権の行使を違憲と解釈した。その結実といえるのが72年見解である\footnote{浦田一郎『集団的自衛権限定容認とは何か-憲法的、批判的分析』(日本評論社、2016年)23~25頁。}。ただし留意すべきこととして、7・1決定において「従来から政府が一貫して表明してきた見解の根幹、いわば基本的な論理」として援用された72年見解は、国際法上の集団的自衛権を違憲としただけであって、限定的な集団的自衛権に関して言及していない。

\subsection{従来の解釈と7・1決定後の解釈の接続}

従来集団的自衛権の行使を認めてこなかったのに、認めるといった憲法解釈の変更を行った、というのは正確な理解ではないように思える。実際には、従来国際法上の集団的自衛権は認めていないが、いわば限定的な集団的自衛権には言及せずにいたというべき状態であり、7・1決定により明確に「集団的自衛権」として定義づけ、その行使が可能と認めたのである。

長谷部恭男や横畠裕介のように論理的整合性、法的安定性を重要視する考えは一定程度支持できる。最高行政機関である内閣が職務を執行するにあたり、憲法その他の法令を解釈し、その解釈を変更することは問題ないとはいえ、我が国の憲法において根幹の1つである平和主義については、その解釈を好き勝手に変えることはあってはならない\footnote{朝日新聞のアンケート(「安保法案 学者 アンケート」朝日新聞2015年7月11日)において、関西学院大学大学院司法研究科教授の井上武史も、「従来解釈との論理的整合性や一貫性が欠如していることは、それだけでは憲法違反の理由にはならないと思います。もちろん、新解釈への変更の理由や妥当性について、政府は政治責任を負うことになります」と述べ、解釈変更は可能であるとする一方、説明責任の必要性を説いている。}。一方で、そもそも従来の解釈は疑問が残るものであることからも、まるで従来の解釈が正しいためその法的安定性を神聖視するような考えには納得しがたい。

そこで、従来の解釈に固執しすぎることなく、今回の「集団的自衛権」を考える。自衛権発動の新3要件①を見る限り、他国に自衛隊を派遣するといってもその主目的は自国のためであり、個別的自衛権の本質である自国防衛を引き継いでいる。集団的自衛権の本質である他国防衛の性質を持ち合わせていないことからも、「集団的自衛権」は個別的自衛権の延長上であると理解すべきと思われる。この前提を踏まえて、「集団的自衛権」の根拠についてみていく。

\section{「集団的自衛権」の根拠}

\subsection{個別的自衛権の根拠規定}

72年見解が出るきっかけともなった1972年9月14日の参議院決算委員会にて、当時の内閣法制局長官の吉國一郎は、なぜ憲法9条で自衛権(=個別的自衛権)が認められ、その自衛権を行使して自衛のために必要最小限度の行動をとることを許されているか、次のように説明している。憲法第9条は、まさに国際紛争解決の手段として武力を行使することを放棄しているが、国家の固有の権利としての自衛権を否定したものでないとしたうえで、「その自衛権があるということから、さらに進んで自衛のため必要な行動をとれるかどうかということになりますが、憲法の前文においてもそうでございますし、また、憲法の第13条の規定を見ましても、日本国が、この国土が他国に侵略をせられまして国民が非常な苦しみにおちいるということを放置するというところまで憲法が命じておるものではな」く、「この国土がじゅうりんをせられて国民が苦しむ状態を容認するものではない。したがって、この国土が他国の武力によって侵されて国民が塗炭の苦しみにあえがなければならない。その直前の段階においては、自衛のため必要な行動はとれる」\footnote{ 第69回国会参議院決算委員会閉会後第5号(1972年9月14日)11頁(吉國一郎)。}。つまり、個別的自衛権発動の根拠として、憲法前文における平和的生存権、憲法13条を挙げている。

\subsection{「集団的自衛権」の根拠規定}

7・1決定において、政府は、「憲法第9条はその文言からすると、国際関係における『武力の行使』を一切禁じているように見えるが、憲法前文で確認している『国民の平和的生存権』や憲法第13条が『生命、自由及び幸福追求に対する国民の権利』は国政の上で最大の尊重を必要とする旨定めている趣旨を踏まえて考えると、憲法第9条が、我が国が自国の平和と安全を維持し、その存立を全うするために必要な自衛の措置を採ることを禁じているとは到底解されない」と述べている。すなわち、政府は憲法前文における平和的生存権と憲法13条を根拠に、「集団的自衛権」の行使を容認している。

「集団的自衛権」は個別的自衛権の延長上として捉えることができることからも、その根拠規定が同一であるのは当然である。以下では、「集団的自衛権」の根拠として憲法前文における平和的生存権と憲法13条が妥当なのかみていくこととする。

\subsection{根拠規定① 憲法前文における平和的生存権}

憲法前文における平和的生存権が「集団的自衛権」の行使の根拠となりうるのか。裁判所は、憲法前文が平和を希求するものだと強調している。東京地裁は、「憲法前文は、憲法の基本的精神及び理念を表明したものであって、そこに表明されたものが本文各規定を解釈する指針となり得ることがあるとしても、それ自体が具体的権利の賦与やその保障を定めたものとは解し難い」とするも、「日本国民が平和のうちに生存すべきであるとの理念を備えていることに疑いはない」としている\footnote{東京地裁2019年11月7日。}。砂川事件判決(最大判1959年12月16日判タ99号42頁)では「われら日本国民は、憲法9条2項により、同条項にいわゆる戦力は保持しないけれども、これによつて生ずるわが国の防衛力の不足は、これを憲法前文にいわゆる平和を愛好する諸国民の公正と信義に信頼することによつて補ない、もつてわれらの安全と生存を保持しようと決意したのである」とした。

現代において、平和を希求すれば平和になるのか。中国や北朝鮮、ロシアなどの各国が軍備増強していく中で、平和や中立を唱えるだけでは平和は訪れない。「もし、攻め込まれたら\ldots{}\ldots{}」という不安が付きまとい、真の平和を構築することはできないだろう。その最悪の例として、中立を宣言していたベルギーが二度の大戦にわたってドイツの侵略を受けた歴史がある。もちろん、ヨーロッパとアジアを同様に考えるのは危険だが、平和や中立だけでは真の平和は訪れないことを示しているだろう。

平和的に生存することと自国を防衛することは矛盾しないだろう。したがって、憲法前文の精神を尊重しつつも、自国防衛を可能とすることは問題にないと思われる。

\subsection{根拠規定② 憲法13条}

次に憲法13条である。一般に憲法9条によって武力行使は禁じられるが、憲法13条により例外的に日本国民の安全を守るための武力行使が認められる、と解釈してきた\footnote{木村草太「安保法案のどこに問題があるのか」長谷部恭男(編)『検証・安保法案-どこが憲法違反か』(有斐閣、2015年)13頁。}。これに対し、東京外国語大学教授である篠田英朗は、憲法13条の権利を持つ人民が原理としての「厳粛な信託」(憲法前文)で政府を樹立することからも、憲法13条の権利をより強く守るために平和を求める政策が求められ、その手段として憲法9条の規定が定められている、と反論している\footnote{ 篠田英朗『集団的自衛権の思想史-憲法九条と日米安保』(風行社、2016年)32~36頁。}。後者の考え方は日本国憲法の基礎ともなっている社会契約論とも親和的で非常に説得的である。このように複数の解釈の仕方があるが、結論として、憲法9条と憲法13条という両者の権利が衝突し、その整合的解釈として必要最小限度の範囲における武力行使が認められていることになるだろう。この点に関する議論は、72年見解より前にまで遡ることができる。元内閣法制局長官の佐藤達夫は、憲法13条からいうと、「外敵の侵入によつて国民の安全が害されるような場合、国は、あらゆる手段によつてそれを撃退しなければならないことになる。その方からいうと、防衛のための実力は強ければ強いほどいいわけである。したがつて、その意味では第13条は国の戦闘力をおし上げる方向の要請を含んでいるわけである。しかし、それを手放しにしておいたのでは、もうひとつの憲法の理想となつている平和主義に衝突する」と述べている\footnote{佐藤達夫『憲法講話』(立花書房、1960年)19、20頁。}。憲法13条による要請をどこまで許容するかは、かつてから問題になっていたことがわかる。

より具体的に憲法13条の中身を検討したい。憲法13条は、「すべて国民は、個人として尊重される。生命、自由及び幸福追求に対する国民の権利については、公共の福祉に反しない限り、立法その他の国政の上で、最大の尊重を必要とする」と規定している。つまり、国民の「生命」や「自由」に関して、「立法その他の国政の上で、最大の尊重」をしなければならない。憲法13条は日本国民を守ることを念頭に置いているのであり、外国を防衛することを義務付けていないのである。国際法上認められている集団的自衛権は、外国からの要請に基づき、その外国の防衛を援助する権利であり、文言を素直に読むと、憲法13条によって根拠づけることはできないのは明らかである。

それでは、集団的自衛権の限定行使についてはどうだろうか。日本政府は、憲法13条を非常に意識していることが要件からも分かる。自衛権発動の新3要件①は、「我が国に対する武力攻撃が発生したこと、又は我が国と密接な関係にある他国に対する武力攻撃が発生し、これにより我が国の存立が脅かされ、国民の生命、自由及び幸福追求の権利が根底から覆される明白な危険があること」である。外国からの侵略から日本国民を守るのは、日本政府の非常に重要な役割の1つである。それを前提に、憲法13条との整合性を意識して要件を組み立てている。一見問題はなさそうである。

集団的自衛権の限定行使の容認に憲法13条の幸福追求権を援用することを批判する学者として、憲法を専門とする上智大学教授の高見勝利が挙げられる。憲法13条の幸福追求権は、集団的自衛権の限定行使容認の根拠規定にとどまらず、要件に組み込まれることで集団的自衛権の限定行使の目的にされており、その結果、佐藤の恐れた武力行使の範囲拡大が現実にものになると高見教授は警告している。その上で、2つの理由から憲法13条の援用を否定している。第1に、幸福追求権を含む憲法上の「自由」は「国家からの自由」を指すため、我が国が武力行使に訴えることを正当化することはできない。第2に、仮に憲法13条を援用できたとしても、幸福追求権による保護義務は我が国の主権が及ぶ領域等に限られているはずであるから、ホルムズ海峡といった世界各地にまで幸福追求権の保護が及ぶはずがない\footnote{高見勝利「集団的自衛権行使容認論の非理非道 従来の政府見解との関係で」『世界』2014年12月号185、186頁。

  高見勝利「集団的自衛権『限定行使』の虚実-『保護法益』の視点から」『世界』2015年9月号76、77頁。}。

第1の理由について、京都府学連事件(最大判1969年12月24日判タ242号119頁)で「憲法13条は、\ldots{}\ldots{}国民の私生活上の自由が、警察権等の国家権力の行使に対しても保護されるべきことを規定している」と最高裁が述べていることからも理解できる。しかし、高見教授の考えはいささか「自由」に着目しすぎではないだろうか。「集団的自衛権」行使の容認の根拠として憲法13条を援用していることは、「自由」よりも「生命」に着目していると言えそうである。国民の「生命」を「立法その他の国政の上で、最大の尊重」するのであれば、国家が国民の「生命」を侵害しないだけでなく保護することまで憲法13条が要求していると解するのは曲解といえないのではないか。さらに、「集団的自衛権」は個別的自衛権の延長上ともいうべきでものであるから、個別的自衛権もまた憲法13条を根拠にしている。仮に高見教授の主張を受け入れたとすると、個別的自衛権も認められないことになってしまう。

第2の理由について、「集団的自衛権」による保護の対象は一義的には日本国内にいる日本国民であり、世界各地に幸福追求権の保護が及ぶという考えはそもそも少し的を外しているように思われる。

以上の点からも、高見氏の主張には疑問点が存在し、憲法13条の援用を完全に否定するには説得力が乏しいと思われる。

\subsection{砂川事件判決について}

政府は、「集団的自衛権」の行使を容認する根拠の1つとして、砂川事件判決(最大判1959年12月16日判タ99号42頁)を挙げている\footnote{第189回国会参議院我が国及び国際社会の平和安全法制に関する特別委員会第12号(2015年8月25日)16、17頁。}。判決の「わが国が、自国の平和と安全を維持しその存立を全うするために必要な自衛の措置をとりうることは、国家固有の権能の行使として当然のことといわなければならない。」という箇所に着目し、我が国の自衛のための「集団的自衛権」の行使は容認されると解釈した。その上で、集団的自衛権の行使全般を認めるのは現憲法のもとでは困難であり、その場合には憲法改正が必要になると明言している\footnote{第186回国会衆議院予算委員会第18号(2014年7月14日)3頁。}。

ところで、学者が唱える学説と裁判官の裁判(判決や決定)との間には、その思考回路において大きな違いがあると元最高裁判事である藤田宙靖は述べる。裁判官にとっては、目の前にある当事者間の現実の争いについて、そのいずれかに軍配を挙げるのかこそが究極の課題であり、裁判判決は、その紛争をどう解決するのが最も適正化という見地からなされているに過ぎず、厳密には判決理由は一般論を展開していないのである\footnote{藤田宙靖『裁判と法律学「最高裁回想録」補遺』(有斐閣、2016年)34~37頁。}。

砂川事件は、アメリカ合衆国軍隊の駐留が、憲法9条2項前段に違反するかどうか争った事件である。その文脈をほとんど無視して、一部だけ切り出して用いて都合よく解釈するのは少し無理がある\footnote{朝日新聞による砂川事件の最高裁判決が集団的自衛権行使を認めているかどうかという憲法学者への質問に対して、認めていると回答した学者はいない。}。「集団的自衛権」の行使が可能であることをより説得力をもって説明するために砂川事件判決を挙げたのは政府のミスである。しかし、あくまで「集団的自衛権」は憲法前文における平和的生存権と憲法13条を根拠にしているものなので、この政府のミスをもって「集団的自衛権」が違憲であると唱えるのも無理があるだろう。

\subsection{「集団的自衛権」の根拠規定のまとめ}

憲法9条との関係で、行使できる自衛権とできない自衛権との線引きは非常に難しい。しかし、それは解釈の問題である。ここまで「集団的自衛権」の根拠規定を見てきたが、砂川事件判決を援用することは不適切であるものの、憲法前文における平和的生存権と憲法13条は「集団的自衛権」の根拠になり得るであろう。したがって、「集団的自衛権」が自国の防衛を本質としている限り、憲法に違反すると言い切るのは不可能であるように思われる。

\section{おわりに}

ここまで「集団的自衛権」について述べてきたが、国際法や憲法、裁判の判決に関する言及に終始してきた。最後に政策面の個人的な意見を述べておきたい。

「集団的自衛権」行使の要件の1つでもある存立危機事態の発生について、政府は数少ない例の1つにホルムズ海峡の機雷封鎖を挙げている。「我が国が輸入する原油の八割、天然ガスの三割が通過する、エネルギー安全保障の観点から極めて重要な輸送経路であるホルムズ海峡に機雷が敷設された場合には、我が国に深刻なエネルギー危機が発生するおそれがあ」るというのが理由である\footnote{第189回国会衆議院我が国及び国際社会の平和安全法制に関する特別委員会第3号(2015年5月27日)32頁(安倍晋三)。}。ところで、7・1決定は、旧解釈の内容を全面的原則的に否定したのではなく、例外的に極く絞られたケースにおいては例外も認められるという修正をしたに過ぎない。にもかかわらず、上記の理由で例外的な自衛権の発動を認めるのであれば、厳しく絞ったはずの要件が実はその実質において底抜けではないかという疑念が残ってしまうのではないか。さらに、アメリカによる対日石油輸出禁止といった経済制裁を1つのきっかけとして英米との戦争に至った歴史を日本は有することからも、石油が手に入らないからといって「集団的自衛権」という形で武力行使を行うことは正しいのか。

また、政府が挙げた存立危機事態の例として、邦人輸送中の米輸送艦の防護もある。確かに日本人が乗っているのなら見捨てるべきではない。しかし、早稲田大学教授の水島朝穂によると、米軍は非米国市民民間人の退避に軍艦・軍用機を用いることはなく、アメリカはどの国に対しても非米国市民の退避の援助を保障する協定を締結していないのである\footnote{水島朝穂「九条の政府解釈のゆくえ」水島朝穂(編)『立憲的ダイナミズム』(岩波書店、2014年)27~29頁。}。政府が想定する存立危機事態が適切なのかは甚だ疑問ではある。政府は「集団的自衛権」の行使を容認するも、その具体的なケースに関してはっきりせず、日本国民に胡散臭さを与える結果となっている。

このように、「集団的自衛権」の具体的な適用場面を見ると、非常に疑念を持たざるを得ないのが現実である。7・1決定において自衛権発動の要件を見直すに至った理由でもある世界情勢の変化は政治論的なものであるが、「集団的自衛権」の具体的な適用場面については法律学的に判断できるため、この点を曖昧にしている政府の答弁には不信感を持たざるを得ない。

\section{参考文献}

・麻生多聞「自衛隊の海外派遣(特集
ステップアップ憲法)」法学教室476号(2020年)。

・浦田一郎『集団的自衛権限定容認とは何か-憲法的、批判的分析』(日本評論社、2016年)。

・大石眞「日本国憲法と集団的自衛権」ジュリスト1343号(2007年)。

・奥平康弘・山口二郎(編)『集団的自衛権の何が問題か 解釈改憲批判』(岩波書店、2014年)。

・木村草太「安保法案のどこに問題があるのか」長谷部恭男(編)『検証・安保法案-どこが憲法違反か』(有斐閣、2015年)。

・纐纈厚『集団的自衛権容認の深層-平和憲法をなきものにする狙いは何か』(日本評論社、2014年)。

・阪田雅裕『政府の憲法解釈』(有斐閣、2013年)。

・阪田雅裕『憲法9条と安保法制 政府の新たな憲法解釈の検証』(有斐閣、2016年)。

・佐藤達夫『憲法講話』(立花書房、1960年)。

・篠田英朗『集団的自衛権の思想史-憲法九条と日米安保』(風行社、2016年)。

・高橋和之・浅田正彦・安念潤司・五十嵐武士・山内敏弘「{[}座談会{]}憲法9条の過去・現在・未来」ジュリスト1260号(2004年)。

・高見勝利「集団的自衛権行使容認論の非理非道 従来の政府見解との関係で」『世界』2014年12月

号。

・高見勝利「集団的自衛権『限定行使』の虚実-『保護法益』の視点から」『世界』2015年9月号。

・田畑茂二郎『安保体制と自衛権』(甲陽社、1969年)。

・中村明『戦後政治にゆれた憲法九条〔第2版〕―内閣法制局の自信と強さ/「武力行使と一体化論」の総仕上げ』(中央経済社、2001年)。

・長谷部恭男、石川健治、宍戸常寿(編)『憲法判例百選II(第7版)』(有斐閣、2019年)。

・長谷部恭男・杉田敦編『安保法制の何が問題か』(岩波書店、2015年)。

・樋口陽一「どう読み、どう考えたか
藤田宙靖『覚え書き-集団的自衛権の行使容認を巡る違憲論議について』に接して」『世界』2016年6月号。

・細谷雄一『安保論争』(筑摩書房、2016年)。

・藤田宙靖「覚え書き-集団的自衛権の行使容認を巡る違憲論議について」『自治研究』2016年2月号。

・藤田宙靖『裁判と法律学「最高裁回想録」補遺』(有斐閣、2016年)。

・水島朝穂「九条の政府解釈のゆくえ」水島朝穂(編)『立憲的ダイナミズム』(岩波書店、2014年)。
