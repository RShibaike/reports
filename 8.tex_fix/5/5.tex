\documentclass[twoside]{jsarticle}
\usepackage[dvipdfmx,hidelinks]{hyperref}
\usepackage{pxjahyper}
\usepackage[jis2004]{otf}
\usepackage[dvipdfmx]{graphicx}
\usepackage{url}
\setcounter{page}{1}
\usepackage{fancyhdr}
\begin{document}
\title{難民申請者の人権享有主体性\\\large―マクリーン事件最高裁判決の克服と伝統的性質説の精緻化を念頭に―
}
\author{小西和也}
\date{}
\maketitle
\tableofcontents
\clearpage
\pagestyle{fancy}
\lhead[難民申請者の人権享有主体性―マクリーン事件最高裁判決の克服と伝統的性質説の精緻化を念頭に―]{}
\rhead[]{\leftmark}
\section{はじめに―問題の所在}

日本国憲法は、「われらは、いづれの国家も、自国のことのみに専念して他国を無視してはならない」と前文にて定めるように、国際協調主義を標榜し、諸外国との共存を期する。このような理念に従って、憲法上「国民の権利」と規定されている基本的人権についても、日本国民でない外国人に可能な限り適用させようとする試みが行われてきた。従来の判例・通説は、憲法上の権利について、各々の権利の性質によって、外国人に適用可能なものとそうでないものを区別し、権利の性質の許すかぎり、外国人に保障されるとする「性質説(以下便宜上、伝統的性質説とする)」を採用し、より普遍性のある人権保障を志向している。しかし、今日においても外国人に対する人権問題は多方面に根深く残っているのが実情である。

その一例として挙げられるのが、わが国に庇護を求める難民申請者の問題である。1980年代に難民認定手続が整備されて以降、わが国では難民申請に関わる種々の問題が顕在化している。特に、難民申請者の入管施設内の待遇については、アメリカ国務省の作成する国別人権報告書においても、その課題が指摘されるところである\footnote{ 在日米国大使館・領事館「2019年国別人権報告書―日本に関する部分」https://jp.usembassy.gov/ja/human-rights-report-2019-ja/(2020年8月25日最終閲覧)参照。}。このような人権問題を個別に議論するに先立って、難民申請者に憲法上の権利が保障されるのか、保障されるとすればそれはどの程度にまで及ぶのかという、いわゆる難民申請者の人権享有主体性の問題を検討する必要性はますます大きくなっているものと考える。

本稿では、最初に難民保護に関わる国際法・国内法制度を取り扱った上で、わが国の難民認定制度の問題点を提示する。そして、外国人の人権享有主体性に関する判例・学説の展開を概観し、難民申請者に保障されるべき基本的人権の内容および程度について考察する。

\section{難民をめぐる国際法・国内法制度}

\subsection{国際的難民保護体制の確立}

現在でいう難民問題が初めて国際社会で注目を集めるようになったのは、20世紀初頭のことである。第1次世界大戦後には、ロシア帝国やオスマン帝国の崩壊に伴い、大量のロシア系・アルメニア系難民が出現した。さらに時代が下ると、スペイン内戦やドイツ・イタリアを中心とするファシズム体制の伸張により、新たな難民が発生することとなった。

第2次世界大戦が終結すると、拡大しつつある難民問題について、「いかなる者が難民であるべきか」という一般的な定義を含む国際文書の必要性が認識された\footnote{ 日本弁護士連合会人権擁護委員会編『難民認定実務マニュアル〔第2版〕』(現代人文社、2017年)3頁。}。その結果、難民問題の解決に向けた国際社会の法的合意として、難民の地位に関する1951年条約(「難民条約」)および難民の地位に関する1967年議定書(「難民議定書」)が採択されるに至った\footnote{ 難民議定書は、難民条約における「難民」の範囲につき時間的・地理的制約を置いていた規定を撤廃し、保護対象となる難民の範囲を拡大することを主たる内容としている。また、一般に「難民条約」と呼ぶ場合、1951年の難民条約と1967年の難民議定書を合わせて指すことが多い。そこで、本稿でも特に断りのない限り、「難民条約」という用語をこの用法により用いることとする。}。その内容は難民に該当する者の法的地位について規定するものであるが、とりわけ難民に対する不法滞在を理由とする処罰の禁止(難民条約31条)および迫害のおそれのある国への難民の追放の禁止(同33条、ノン・ルフールマン原則)が重要である。

以上で示した難民条約の定める定義によれば、「難民」とは「人種、宗教、国籍若しくは特定の社会的集団の構成員であること又は政治的意見を理由に迫害を受けるおそれがあるという十分に理由のある恐怖を有するために、国籍国の外にいる者であって、その国籍国の保護を受けることができないもの又はそのような恐怖を有するためにその国籍国の保護を受けることを望まないもの及び常居所を有していた国の外にいる無国籍者であって、当該常居所を有していた国に帰ることができないもの又はそのような恐怖を有するために当該常居所を有していた国に帰ることを望まないもの」であると解される。このように、条約は政治的理由に基づく難民(政治難民)のみを保護の対象としている。

しかしながら、現実の難民は様々な背景事情に基づき生じる、より多義的な性格を有するものである。例えば、政治的理由以外にも、武力的紛争を理由とする難民(紛争難民)、天災を理由とする難民(災害難民)、経済的貧困を理由とする難民(経済難民)などが実際には存在する\footnote{ 大沼保昭は、今日の難民問題の構造的原因として、
  政治的植民地体制の遺制および経済的・文化的植民地体制が依然強固のために、政治的支配から解放された諸民族の一部が、個々人の保護を国民国家体制の形で十分に保障することができないという事情がある点を指摘する。このような原因および特徴から、同人は、伝統的な政治難民概念に必ずしも該当しない難民を「構造難民」と呼称すべきとする。大沼保昭「『外国人の人権』論再構成の試み」法学協会百周年記念論文集『憲法行政法・刑事法』(有斐閣、1983年) 

  372-373頁、378頁。}。これらの者は、条約締結国が保護するよう義務づけられてはいないものの、国連難民高等弁務官事務所(UNHCR)が保護を考慮すべきと示唆している者である\footnote{ 国連難民高等弁務官(UNHCR)駐日事務所「難民認定基準ハンドブック ―難民の地位の認定の基準及び手続に関する手引き―〔改訂版〕」https://www.unhcr.org/jp/wp-content/uploads/sites/34/2017/06/HB\_web.pdf(2020年08月25日最終閲覧)12頁、18頁、44頁参照。}。以上のように、条約上の定義の「難民」には含まれないが、人道上国際的に保護されるべき者が現実として存在することを忘れてはならない。

本稿では、条約上の定義に該当する前者の立場を「狭義の難民」、定義に該当しないが国際的保護が求められる後者の立場を「広義の難民」などと表すこととする。

\subsection{わが国の難民認定制度の概要}

ベトナム戦争終結後、大量のインドシナ難民が小船で日本国内に到着するという事件(ボートピープル)が起きたことをきっかけに、わが国でも難民問題に対処する必要が生じた。1981年には難民条約に、1982年には難民議定書にそれぞれ加入し、難民認定制度を中心とする国内法整備が図られた。

現在の難民認定手続は、主に「出入国管理及び難民認定法(以下、入管法とする)」の規定のもとで行われている。以下では当該手続の流れについて概括する。

わが国に難民として庇護を希望する者は、入管法61条の2第1項に基づく難民認定申請を行う。申請後は、難民調査官によるインタビューを介して、難民に該当するか否かの審査が行われる。また、申請から認定許否の結果が出るまで、仮滞在が認められる場合がある(同61条の2の4参照)\footnote{ もっとも、現実に仮滞在が認められる割合は非常に少ない。2019(令和元)年に仮滞在の許否を審査された733人のうち、仮滞在の許可が認められたのは25人にとどまる。法務省「令和元年における難民認定者数等について」http://www.moj.go.jp/content/001317678.pdf(2020年8月25日最終閲覧)8頁。}。難民認定の処分を受けた者は、法務大臣により難民認定証明書を交付され(同61条の2第2項)、定住者の在留資格が許可される(同61条の2の2第1項)。また、生活保護法に基づく生活保護等、一定の社会福祉を受けることができる。

難民不認定の処分を受けた者は、処分に対する審査請求を行うことができる(同61条の2の9)ほか、処分の取消しを求める訴訟を提起することができる(行政事件訴訟法14条)。

\includegraphics[width=6.26772in,height=6.58333in]{../../media/5/image1.jpg}

(図は、日本弁護士連合会人権擁護委員会・前掲注(2)11頁より抜粋)

\subsection{わが国における難民申請者に関する諸問題}

わが国における難民申請者について、以下のような問題がある。

第1に、難民認定の数そのものが少ないという問題がある\footnote{ わが国の難民認定制度の実体については、関聡介「続・日本の難民認定制度の現状と課題」難民研究ジャーナル2号(2012年)に詳しい。}。2019(令和元)年の難民申請者が10,375人であったのに対して、難民認定を受けた者は44人にとどまる\footnote{ 法務省・前掲注(6)(2020年8月25日最終閲覧)1頁、7頁。}。その原因として、入管法による難民の認定が、専ら難民条約上の「難民」(狭義の難民)に該当するかに限定して判断している点が挙げられる(2条第3号の2参照)。換言すれば、わが国で難民認定がされない難民申請者の中には、広義の難民が一定数含まれている。

第2に、難民申請者の収容に関する問題がある。仮滞在許可が得られない、または仮滞在期間の終期を迎えた難民申請者は、在留資格を有しない非正規滞在者として、収容令状による収容(入管法39条1項)や退去強制令状による収容(同52条5項)の対象となる。難民認定の審査が長期化している現状において\footnote{ 難民認定審査にかかる平均処理期間について、2013(平成25)年は4.7か月~5.7か月であったのに対し、2018(平成30)年は333日(約11.1か月)~492日(約16.4か月)となっている。法務省「難民認定審査の処理期間の公表について」http://www.moj.go.jp/nyuukokukanri/kouhou/nyuukokukanri03\_00029.html(2020年8月25日最終閲覧)参照。}、長期収容や収容待遇・環境に起因する事件が発生している\footnote{ 「外国人収容者、相次ぐ自殺未遂 人権軽視の入管行政 拘束、仮放免 独自で判断」毎日新聞東京夕刊2018年5月28日2頁、「入管に長期収容、抗議のハンスト 茨城、1週間」朝日新聞朝刊2018年11月27日34頁、「入国管理センターで収容者死亡 病状悪化も『放置』なぜ」毎日新聞東京夕刊2019年7月8日2頁、「『入管で手錠、14時間』 ペルー人男性、賠償求め国提訴」朝日新聞夕刊2020年2月20日11頁参照。また、わが国における難民申請者の収容の実体については、児玉晃一「日本における難民申請者と収容」難民研究ジャーナル8号(2018年)に詳しい。}。

第3に、難民申請者の社会保障に関する問題がある。難民申請中は、外務省より支援金として保護費を受給することができるが、通常の生活保護費と比較すると支援金を得るための審査は長期化しやすく、受給額も低い\footnote{ 難民支援協会「日本にいる難民のQ\&A ―難民から見える世界と私たち―」\\https://www.refugee.or.jp/jar/postfile/QA.pdf(2020年8月25日最終閲覧)14-15頁参照。}。

\section{学説・判例の展開}

\subsection{総説}

以上のような難民申請者をめぐる諸問題について憲法上の考察を及ぼそうとする際に、まず考えなければならないのが、難民申請者の人権享有主体性の問題であるという点は先に述べた通りである。まずはこの問題に先立って、より一般的な論点である外国人の人権享有主体性について伝統的に主張されてきた学説の議論および判例の展開について整理する。

\subsection{学説の議論}

外国人に憲法上の権利がどの程度保障されるかについては、戦後直後、いくつかの見解が主張されている。例として、外国人の人権は憲法上一切保障されないとする「無保障説」や\footnote{ 佐々木惣一『改訂日本国憲法論〔補正版(再版)〕』(有斐閣、1954年)468頁。}、憲法の人権規定の中で「何人も」で始まる権利は外国人にも保障されるのに対し、「国民は」で始まる権利は外国人に保障されないとする「文言説」などが唱えられている\footnote{ 稲田正次『憲法提要〔新版〕』(有斐閣、1964年)144頁。}。

しかし近時までは、外国人にも原則憲法上の権利が保障されるとしつつ、その保障の程度については、個々の権利の性質に応じて検討されるべきとする「伝統的性質説」が支配的見解とされてきた。こうした説が採用されるに至った理由について、以下の点が挙げられる。

まず無保障説に立った場合、憲法上の権利が、より前近代的かつ普遍的である「自然権としての人権」を保障するための制度的性格を有するものである点を無視することになる。さらに、日本国憲法前文第3段にて国際協調主義が掲げられている点にも矛盾することになり、妥当ではない。

次に文言説に立った場合、外国人に保障される人権を「何人も」や「国民は」といった憲法上の規定で厳密に区別することの困難が生じる。22条2項の定める国籍離脱の自由は、その性質上日本国民にのみ保障される権利であるにもかかわらず「何人も」という文言で規定されているなど、文言説を貫徹すると不都合な事態が生じることになる。

以上の点から学説の多くは伝統的性質説を採用し、その上で「参政権」「社会権」および「入国の自由」などについては、性質上外国人に保障されない人権であると解する。また、外国人に保障される人権(自由権等)についても、その性質上日本国民と同程度には保障しえないものがあるとされる。例えば、外国人の政治活動の自由については、国政レベルの参政権が保障されていないことなどを理由に、日本国民よりも大きな制約を加えることを容認するのが多数説である\footnote{ 横田耕一「人権の享有主体」芦部信喜ほか編『演習憲法』(青林書院、1984年)143頁、芦部信喜(高橋和之補訂)『憲法〔第7版〕』(岩波書店、2015年)96-97頁、佐藤幸治『日本国憲法論〔第2版〕』(成文堂、2020年)168頁参照。}。

\subsection{判例の展開}

以上のような学説の展開に対して、判例は比較的早期の時点で外国人の人権享有主体性を認める立場をとっていたことが見てとれる。韓国国籍を有する外国人が、不法入国の罪による刑を終了したのちに外国人虚偽登録罪で起訴され、保釈許可の決定を得て拘置所を出所した直後に、今度は不法入国を理由とする退去強制令状により拘束されたという事件において、原審は「不法入国者は其の国に対して後国家的基本的人権の保護を要求する権利を有しないと解すべきであろう」\footnote{ 京都地判1950(昭和25)年10月25日下民集1巻10号1723頁。}としたのに対し、最高裁はこれを斥けて、「いやしくも人たることにより当然享有する人権は不法入国者と雖もこれを有するものと認むべきである」\footnote{ 最二小判1950(昭和25)年12月28日民集4巻12号683頁。}と述べている。

今日、最高裁が外国人の人権享有主体性について伝統的性質説の採用を最初に明示したとされているのが、マクリーン事件判決\footnote{ 最大判1978(昭和54)年10月4日民集32巻7号1223頁。}である。本事案は、アメリカ国籍を有する外国人教師が、在留期間1年として入国し、1年後に、その在留期間の更新を申請したところ、法務大臣が、彼の在留中の無届転職および政治活動を理由として、更新を拒否したため、その可否について争われたものである。この点、最高裁は、外国人の基本的人権の保障について、「権利の性質上日本国民のみを対象としていると解されるものを除き、わが国に在留する外国人に対しても等しく及ぶものと解すべき」とし、政治活動の自由も「わが国の政治的意思決定又はその実施に影響を及ぼす活動等外国人の地位にかんがみこれを認めることが相当でないと解されるものを除き、その保障が及ぶもの」とした。

他方で、最高裁は同時に、以上の外国人の基本的人権の保障が、在留制度を軸とする出入国管理システムに劣後する余地を認めている。すなわち、「外国人の在留の許否は国の裁量にゆだねられ、わが国に在留する外国人は、憲法上わが国に在留する権利ないし引き続き在留することを要求することができる権利を保障されているものではなく、ただ出入国管理令上法務大臣がその裁量により更新を適当と認めるに足りる相当の理由があると判断する場合に限り在留期間の更新を受けることができる地位を与えられているにすぎないものであり、したがって、外国人に対する憲法の基本的人権の保障は、右のような外国人在留制度のわく内で与えられているにすぎない」とするのである。このように最高裁は、一般論として、あたかも外国人の人権を尊重するようなリップサービスをしておきながら、その運用においては、外国人の人権を実質的に否定している\footnote{ 後藤光男『永住市民の人権』(成文堂、2016年)245頁。}。

\subsection{マクリーン事件最高裁判決の検討}

マクリーン事件において最高裁が示した判例法理に対しては、特に外国人の人権を積極的に肯定する見解からの批判が根強い。近藤敦は、あたかも入管法が憲法の上位法であるかのような転倒した思考方法から脱却すべきであると説く\footnote{ 近藤敦「外国人の『人権保障』」自由人権協会編 『憲法の現在』(信山社、2005年)325頁。}。また日比野勤は、安念潤司の指摘を引用し、入管法に基づき在留を認められた外国人に保障される権利を、憲法で保障された基本的人権と呼ぶことは背理ではないかと疑問視し、憲法の法律に対する優位を主張する\footnote{ 日比野勤「外国人の人権(1)」法学教室210号(1998年)40頁、安念潤司「『外国人の人権』再考」芦部信喜先生古稀祝賀『現代立憲主義の展開 上』(有斐閣、1993年)177頁。安念は、「外国人の入国・在留の条件を認めるか否かが国家の完全な自由裁量に任されている結果、入国・在留の条件を付すという意味で在留外国人の人権を自由に制限できることになるのだとすれば、外国人の入国・在留は憲法上の権利ではないという原則を前提としつつ、なお、本邦に在留している外国人の人権を論ずることが、そもそも問の立て方として正当か、という疑問が生じよう」とする。}。国際法的観点から見ても、日本はこの判決の翌年に国際人権規約(社会権規約・自由権規約)を批准しており、判決の例示をもとに外国人の人権保障を在留制度のわく内に押し込めるといったあり方は、このような人権条約の立場から根本的に見直さなければならないという批判が存在する\footnote{ 申\UTF{60E0}\UTF{4E30}『国際人権法からみた外国人の人権』自由と正義62巻2号13頁。 }。

この法理にて批判される問題は、難民申請者を含む非正規滞在者が当事者である場合に、より深刻なものとして受け止められることになる。すなわち、在留資格の与えられない非正規滞在者は文字通り「在留制度の枠外」に置かれているのであり、判例法理を素直に読むならば、非正規滞在者には何ら人権が保障されないという帰結に至りかねない\footnote{ 関聡介「非正規滞在者の権利」近藤敦編『外国人の人権へのアプローチ』(明石書店、2015年)158頁。}。しかし、このような帰結では、立憲主義に基づく人権保障の意義を無に帰せしめることになる\footnote{ 近藤敦『外国人の人権と市民権』(明石書店、2001年)343頁。}。

そもそも難民申請者についていえば、日本国籍でないという理由でその者の人権を不完全な形でしか保障しないとする判例法理の態度は、正当化しえないのではないかという疑義が生じる。この点、長谷部恭男は、ロバート・グッディンの見解を引用して、次のように主張する\footnote{ 長谷部恭男『憲法の理性〔増補新装版〕』(東京大学出版会、2016年)123-126頁参照。}。すべての国家は、それぞれの属する国民に対して権利保障をすべき義務を負うが、この義務は、すべての人に対する権利保障を効果的に実現するために、便宜上認められたものにすぎない。海水浴場にて溺れた者を効果的に救助するためあらかじめ指定されたライフ・セーバーが特定の集団の救助を担当するように、あるいは病院内の患者を効果的に治療するため特定の患者について担当する医師を指定しておくように、各国家が国籍という一応の標識を用いて、その国籍に属する人々の人権を第一次的に保障するよう努めることが、国際社会全体においてより効果的な権利保障をなしうる。この考え方からすれば、外国人の人権は、本来当該外国人の属する国家が第一次的に保障する責任を負うべきである以上、内国民の人権よりも保障の程度は劣後しうることが正当化される(場合によっては、外国人の人権を一切保障しないといった結論を導出することも可能となる)。しかしながら、難民申請者として庇護を希望する者は、およそ自らの属する国家によっては、生存のために必要な保護を期待できないという前提に立つ存在である\footnote{ 大沼・前掲注(4)377頁。}。その者らに対し、日本国籍がない(あるいはそれに見合った在留資格が与えられない)という理由で、一切の人権保障を無視することは許されないというべきだろう。

マクリーン事件最高裁判決の論理は、外国人の権利保障を在留制度の枠内に押しやることで、在留資格の与えられない難民申請者の人権を無視ないし軽視することを正当化づける一つの根拠として働く。しかしながら、避難元の国家が機能不全に陥った結果、およそ十分な権利保障を期待しえないという点で、その者らと狭義の難民との間に大きな差異は存しない。そうだとすれば、難民性が狭義であれ広義であれ認められる、すなわち何らかの形で国際的保護の必要性が認められる庇護希望者は、広く憲法上の権利の享有主体たりうるとするのが相当である。

\section{難民申請者の憲法上の権利に関するアプローチ}

\subsection{総説}

難民申請者が憲法上の権利の享有主体たりうるとした上で、具体的にどのような権利がどの程度保障されるのか。以下では、この問題の解決にあたり有益と考えられる3つのアプローチを挙げて、各々のケースから導出しうる帰結について検討する。いずれのアプローチにおいても共通する点は、先に述べた伝統的性質説を、より精緻化する方向へ再構成しようとする試みであるといえよう。

\subsection{第1のアプローチ―外国人の態様に着目した類型化}

外国人の人権については、個々の権利の性質によって、外国人に適用可能なものとそうでないものを区別し、権利の性質の許すかぎり、すべて保障されるとするのが伝統的性質説の立場であることは、先に述べた。しかしこの見解では、いかなる基準により、いかなる根拠に基づいて外国人を区別して扱うべきかという具体的な判断枠組みを提示する試みが、必ずしも十分にはなされなかった\footnote{ 大沼・前掲注(4)364頁。}。そこで、今日では権利の性質だけでなく、外国人の存在態様に着目して、保障されるべき権利の内容と程度を判断する立場が一般的となっている\footnote{ 横田・前掲注(14)140頁、芦部・前掲注(14)92頁、米沢広一「国際社会と人権」樋口陽一編 『講座憲法学(2) 主権と国際社会』(日本評論社、1994年)175頁。 }。

外国人の態様の類型について、有力・多数説は、\UTF{2460}日本に生活の本拠をもち、しかも永住資格を認められた定住外国人、\UTF{2461}難民、\UTF{2462}その他一時的旅行者などの一般外国人の3つに分類し、各類型で保障される権利の内容と程度に差異を設ける旨を提起する\footnote{ 芦部・前掲注(14)92頁、辻村みよ子『憲法〔第6版〕』(日本評論社、2018年)113頁、藤井俊夫『憲法と人権\ajRoman{1}』(成文堂、2008年)43-45頁参照。}。しかし、このような分類からは、広義の難民や一般外国人の多くが非正規滞在者の状態にあるといった論点が見えにくくなるなどといった批判があげられている\footnote{ 近藤・前掲注(23)353頁。}。そのため、別の方法として、 \UTF{24B6}非正規滞在者、 \UTF{24B7}居住市民(非永住型の正規滞在者)、 \UTF{24B8}永住市民(定住外国人等)、 \UTF{24B9}国民といった類型化を行い、それぞれ正規化、永住許可、帰化という、社会構成員性の3つの承認ルールを経て、段階的に権利保障が強化されると解する説も主張されている\footnote{ 近藤・前掲注(23)343頁、近藤敦『人権法〔第2版〕』(日本評論社、2020年)54-55頁。}。この考え方によれば、 \UTF{24B6}では大半の市民的権利(自由権・受益権・包括的人権)が保障され、 \UTF{24B7}では加えて一部の居住権・職業の自由・社会権が保障される。そして \UTF{24B8}では居住権・職業の自由・社会権がほぼ完全に保障され、 \UTF{24B9}では完全な形での人権保障がなされることになる\footnote{ 近藤・前掲注(30)55頁参照。}。

在留資格が与えられないという、いわば非正規滞在者の状態にある難民申請者の権利保障の在り方を検討する本稿の立場からすれば、後者のような類型化に根差した理解がより整合的であるといえる。この場合、難民申請者は通常\UTF{24B6}に属する地位に基づき、表現の自由などの自由権、裁判を受ける権利などの受益権、および包括的人権が保障されることになる。加えて入管法上の難民認定を受けた者などは、\UTF{24B7}に属する地位に基づき、さらに居住権、職業の自由、社会権といった権利が、一部ながら保障されると考えることになるだろう。

\subsection{第2のアプローチ―居住権を中心とする人権の検討}

第1のアプローチによれば、難民申請者は、たとえ在留資格が認められない場合であっても、精神的自由をはじめとする市民的権利の大半が保障される。さらに現在の学説では、本来在留資格を有する者のみに保障される居住権(居住の自由)を中心とする人権についても、等しく非正規滞在者が享有できるよう理論構成をする試みも見られている。ここではその代表例として、近藤敦の見解と門田孝の見解を提示して検討する。

近藤は、憲法上の権利を外国人に適用するにあたり、次のように主張する\footnote{ 近藤・前掲注(23)274-278頁、同・前掲注(30)55-56頁参照。}。

日本国憲法における基本的人権のうち、条文上、「国民」に保障されると定められているもの(たとえば、憲法13条の幸福追求権や同14条の平等権など)について、権利の性質上、外国人にも広く保障されると解釈することは可能である。他方で、憲法の条文上、「何人も」保障されると定められている権利を外国人に認めないことは、憲法を掲げて個人の権利・自由を公権力の恣意的な侵害から守るという立憲主義の基本に反することになり、許されない。たとえば憲法17条の国家賠償請求権は、「何人も」保障されると定めている以上、例外なくすべての外国人に適用されると解すべきであり、当該権利に相互保証主義を課す国家賠償法6条の規定は、立憲主義の基本に反するといった問題が生じる\footnote{ 従来の性質説の立場からも、違憲であると結論づける見解がある。奥平康弘『憲法\ajRoman{3} 憲法が保障する権利』(有斐閣、1993年)391-393頁参照。}。また憲法22条2項にて、「何人も」保障されるとしている国籍離脱の自由は、「国籍離脱を強制されない自由」という消極的な権利の側面をも含む「国籍自由の原則」を表明するものであると解するのが相当である\footnote{ このように解すると、公務員資格が国民に限られていない性質の公務員について国籍離脱をその地位喪失の事由としたり、すでに公務員に採用されている外国籍の者に在職要件として外国籍の離脱および日本国籍の取得を課したりする行為は憲法22条2項違反となりうる。こうした「国籍離脱を強制されない自由」の侵害が問題となったケースについて、近藤はかつての「司法修習生採用問題」を挙げている。「国籍離脱を強制されない自由」について、近藤・前掲注(23)276-277頁、同・前掲注(30)55頁、法学協会編『注解日本国憲法 上巻〔改訂版〕』(有斐閣、1953年)447頁、「司法修習生採用問題」について、横田・前掲注(14)143頁、田中宏『在日外国人―法の壁、心の溝〔第3版〕』(岩波新書、2013年)143-148頁参照。}。まとめると、「何人も」という条項はすべて日本にいる外国人にも適用され、「国民は」という条項ないし人権の享有主体性が明示されていない条項は、権利の性質上外国人にも適用可能かどうかについて判断すべきである(立憲性質説)\footnote{ 自身の立憲性質説を説明するにあたり、近藤は日比野勤の見解を引用して説明する。すなわち、日比野は、最高裁の思考枠組みがマクリーン事件判決を機に転換したと指摘し、当該判決以前の思考枠組みを「基本的人権優位説」、判決以後の思考枠組みを「出入国システム優位説」と呼び、両者を対比する。近藤は、従来の性質説はいわば出入国システム優位説の立場であり、立憲性質説は基本的人権優位説の立場であるとする。近藤・前掲注(23)338頁。「出入国システム優位説」「基本的人権優位説」について、日比野勤「外国人の人権(2)」法学教室217号(1998年)44頁。}。

以上の立場を示した上で、さらに近藤は、居住権について、次のように理解する\footnote{ 近藤・前掲注(23)278-279頁、338-339頁、同・前掲注(30)275-276頁参照。}。

憲法22条1項の定める居住権は、「住所または居所を自ら決定ないし変更する自由」であり、自由権の核心が国による妨害排除の請求権であることに留意するならば、「恣意的に住居の選択を妨害されない権利」として「何人も」有するものであることを意味する。そうだとすれば、条文上、このような権利が非正規滞在者にも保障されると解することは不当ではない。むしろ、長期にわたり日本に在住し、社会のメンバーとして人間関係のネットワークを形成している者について退去強制することは不合理である(社会構成員性は、政治機関の判断で決まるものではない)。すなわち、非正規滞在者は、受け入れ社会の事実上のメンバーシップの程度に応じて、「恣意的に退去強制されない自由(出身国に安全に帰る権利)」が「22条1項と結びついた13条」により保障されているというべきである。そして以上の権利は、「公共の福祉に反しない限り」保障されるのであり、その制約の合憲性は比例原則を主軸としつつ、自由権規約26条の差別禁止、同6条の恣意的な生命の剥奪、同7条の残虐・非人道的な取扱いの禁止、難民条約33条1項および拷問等禁止条約3条のノン・ルフールマン(追放禁止)原則などを参照して審査することが求められる。

以上の近藤の見解とほぼ同じ結論を、門田は次のような議論から導く\footnote{ 門田孝「在留権」近藤敦編『外国人の法的地位と人権擁護』(明石書店、2002年)60-61頁、67頁参照。}。

憲法22条1項で保障された「居住・移転の自由」は、各人が自己の好むところに居住し移転するについて、公権力による干渉・妨害等がないことを意味する。また、この自由は一応経済的自由に分類されるが、表現の自由を含む精神的自由や人身の自由といった側面をも含む多面的・複合的性格を有するものと解すことができる\footnote{ 芦部信喜『憲法学\ajRoman{3} 人権各論(1)〔増補版〕』(有斐閣、2000年)565頁。}。このような見解を参考に、「居住・移転の自由」とほぼ同義ないしそれと密接な関連を有するものとして、「居住権」を、文字通り「生活の本拠たる住所ないし一時的な滞在地たる居所を定める権利」としてさしあたり理解した場合、この権利の性格からしても、その享有主体から外国人が排除されるという結論は当然には生じない。そして、非正規滞在者こそ、こうした居住権が保障されるか否かについて検討されるべき主体となりうる。たしかに、日本における在留資格を有しない者には居住権も当然認められないものと言えそうではあるが、日本滞在が長期におよび、生活基盤が形成されているような事情の下にある場合には、そうした現在の生活の本拠に居住し続けることを、一個の「権利」として理解することもまた、理由のないことではない。この場合、生活本拠地における衣食住、職業や教育、さらには家族関係や近隣関係など、居住をめぐる物質的および精神的生活条件全般にまで射程が届く広範かつ多面的な権利として、居住権が保障されることになる。

両説は、居住権の範囲(居住権の内包する種々の権利の内容)などにおいて若干の差異が見られるものの、根幹となる解釈指針においては一つの共通点が見いだせる。すなわち、居住権を中心とする人権を、在留資格の有無にかかわらず、実際の生活基盤や人的関係等の事情から、その者に一定程度の社会構成員性があることを根拠に保障される権利であると解し、このような資格が認められる余地のある非正規滞在者について、当該権利が保障されるよう拡充していこうとするものである。難民申請者の人権を考察する本稿において、参考となる。

このようなアプローチの下では、難民申請者の人権は次のように構成することができると考える。

第1に、難民申請者に何らかの社会構成員性のある(=わが国で生活すべき基盤がある)ことが認められる場合には、当該生活を送る上で必要な権利(広義の居住権)が憲法上保障される。代表的なケースとしては、わが国に長期間滞在している者や、家族・親族の状況からわが国に滞在することが望ましいとされる者が、これに該当する。これらの者は、職業の自由や教育を受ける権利、社会権などに基づく公的支援を受ける権利を有する。

第2に、こうした社会構成員性に乏しい難民申請者であっても、政府による恣意的な退去強制がなされない権利(狭義の居住権)は最低限保障される。これは、難民性という庇護希望者に特有の事情を考慮することで導出される帰結である。この結果、退去強制手続の対象となる難民申請者は、自身の出身国によっては本人の保障されるべき権利も保障しえない場合に、当該手続を行わないよう求める権利を有する\footnote{ 狭義の居住権とほぼ同義の内容を指すと考えられるものとして、大沼保昭の「避難の場所の保障(消極的保護)を求める権利」が挙げられる。ただし、大沼は当該権利の根拠を「一時的庇護」「国境における入国拒否の禁止」といった国際法上の規範および憲法上の平和的生存権に求めており、その法的性質は後述する庇護権にむしろ近いものといえる。大沼・前掲注(4)386頁。}。同様の場合に入管施設内に収容されている難民申請者が仮放免(入管法54条)を認めるよう求めることも、当該権利によって正当化されよう。なお、ここでの「難民性」とは、居住権の趣旨にかんがみて、難民条約の定める狭義の難民性にとどまらず、より広義の難民性をも含むと解すべきである。

以上のように考えた場合、難民申請者の中で、入管法上の難民認定を受けた者とそうでない者(特に広義の居住権を認められた者)との間では、少なくとも保障されるべき人権の範囲においてほとんど差異がないこととなる。そこで、両者の区別は、当該人権に、公共の福祉の観点から正当化しうる制約をどの程度課すことができるかという点での差異に求めることになる。この点、入管法に基づく難民の認定は、先述の通り難民条約上の「難民」の定義に該当するか否かを判断して行われるものであるから、広義の難民性を有するにとどまる庇護希望者に対して、入管法上の難民認定および当該認定に基づく一定の法的地位を与えることは原則としてできない\footnote{ 例外として、在留特別許可制度(入管法50条、61条の2の2第2項)に基づき、法務大臣により在留が特別に認められた者。}。その結果、入管法上の難民に認定された者とそうでない者との間で各々の人権にかかる制約の程度が異なるといった事態はありうる。しかし、それでもなお、権利の制約の可能性の高いことと、はじめから人権保障が否定されることとは大きな違いがあるというべきであるし\footnote{ 近藤・前掲注(23)344頁。}、当該制約が憲法に適合するか判断する場合でも、先に述べた社会構成員性および難民性といった指標に従って、その合理性を判断する必要が生じるだろう。

\subsection{第3のアプローチ―庇護権の検討}

前2つのアプローチと若干趣旨の異なるものとなるが、学説上唱えられている庇護権(亡命権)が憲法上認められるか否かも、難民申請者に保障されるべき人権の内容と程度を左右する要素となりうる。庇護(asylum)とは、迫害から逃れてきた者に対し他国が保護することをさすが\footnote{ 内野正幸「日本国憲法下の人権保障と難民保護」法律時報75巻1号(2003年)89頁。なお、庇護にはさらに「領土的(領域内)庇護」「外交的庇護」などといった区別が存在するが、ここでは国際法上原則的とされる「領土的庇護」を念頭に論じることとする。}、ここでは庇護権を、「被迫害者が」他国による庇護を受ける「憲法上の」権利と定義する\footnote{ このような定義を行ったのは、次の2つの理由による。第1に、国際法上の庇護権には「国家が庇護を与える権利」とする考え方と、「個人が庇護を受ける権利」とする考え方が存在するが、前者は国際慣習法上認められるのに対し、後者は当然には認められないとされる。世界人権宣言14条1項は「全ての者は、迫害からの庇護を他国に求め(seek)、かつ享受する(enjoy)権利を有する」とするが、これは個人が他国による庇護を当然に付与される(be
  granted)ことを意味しない。第2に、庇護権には「国際法上の庇護権」と「国内法上の庇護権」といった次元での差異が存在する。そこで、個人が庇護を受ける権利が先述の通り「国際人権として」保障されないとしても、一国の「憲法上の基本的人権として」保障されるか否かを争う余地が残されている。内野・前掲注(42)90-91頁、島田征夫『庇護権の研究』(成文堂、1983年)91頁、93頁、110-112頁、121-122頁参照。}。その内容として重要なのは政治犯罪人(逃亡犯罪人)不引渡の原則とノン・ルフールマン(追放禁止)原則であり\footnote{ 国際法上、「引渡」と「追放」は明確に異なる措置である。引渡は、外国で犯罪を犯し逃亡してきた者を、関係犯罪につき裁判し処罰することを欲する国の管轄権内に送り返すために行われる措置であり、引渡請求国の存在を必要とする。対して追放は、国家がその滞在を欲しない外国人を排除するためにとる措置で、それがたとえ国の内外の犯罪行為を理由にしたとしても、あくまで自国の安全を確保することを目的に単独で行いうるものである。島田・前掲注(43)278頁。
   }、難民申請者(とりわけ広義の難民)と特に関係があるのは後者の方である。

庇護権が認められるかについて、多数説は憲法上の明文規定がないことなどを理由に、否定説をとる\footnote{ 芦部信喜『憲法学\ajRoman{2} 人権総論』(有斐閣、1994年)143頁参照。}。判例も、出身国内で政治的迫害をうけるおそれのある韓国国籍外国人に退去強制令状を発付する処分を行った事件について、「いわゆる政治犯罪人不引渡の原則は未だ確立した一般的な国際慣習法であると認められない」としている\footnote{ 最二小判1975(昭和51)年1月26日訟月22巻2号578頁(尹秀吉事件)。}。これに対して、日本国憲法前文における「われらは、全世界の国民が等しく恐怖と欠乏から免れ、平和のうちに生存する権利を有することを確認する」という平和的生存権の規定を根拠に、外国人の庇護権を広く肯定する見解もある\footnote{ 近藤・前掲注(30)276頁、山内敏弘「外国人の人権と国籍の検討」国際人権8号(1997年)3頁。このうち近藤は、平和的生存権の規定について、難民条約上の難民にとどまらず、内戦その他を理由とする庇護権を導く解釈指針となるとする。}。この場合、当該規定に庇護権を肯定しうる法的性質が認められるか否かが問題となる。この点庇護権を否定する立場は、「全世界の国民」の有する「権利」に関する規定を、大方においてプログラム的・政治的な宣言とみるべきであるとする\footnote{ 内野・前掲注(42)91頁。}。この議論は、平和的生存権の規範性をいかに解すべきかという問題と関連する。

なお、わが国では難民条約加入に伴う立法措置として、一時庇護のための上陸を許可する一時庇護制度が設けられている(入管法18条の2)。

\section{おわりに}

本稿は、特に非正規の滞在を余儀なくされている難民申請者に関して、憲法上の権利の享有主体性という総論的問題に着目した。そのため、各論的問題(難民申請者の収容および入管施設内での待遇、社会保障などの制度的問題)について、十分な考察がなされていない。本稿の議論を前提として、当該問題の論点を検討することが、今後の課題になると考える。

\section{参考文献}

\noindent
・芦部信喜『憲法学\ajRoman{2} 人権総論』(有斐閣、1994年)\\
・芦部信喜『憲法学\ajRoman{3} 人権各論(1)〔増補版〕』(有斐閣、2000年)\\
・芦部信喜(高橋和之補訂)『憲法〔第7版〕』(岩波書店、2015年)\\
・新井誠ほか『憲法\ajRoman{2} 人権』(日本評論社、2017年)\\
・安念潤司「『外国人の人権』再考」芦部信喜先生古稀祝賀『現代立憲主義の展開 上』(有斐閣、1993年)177頁\\
・泉徳治「マクリーン事件最高裁判決の枠組みの再考」自由と正義62巻2号(2011年)19-26頁\\
・稲田正次『憲法提要〔新版〕』(有斐閣、1964年)\\
・内野正幸「日本国憲法下の人権保障と難民保護」法律時報75巻1号(2003年)89-93頁\\
・大沼保昭「『外国人の人権』論再構成の試み」法学協会百周年記念論文集『憲法行政法・刑事法』(有斐閣、1983年)361-417頁\\
・奥平康弘『憲法\ajRoman{3} 憲法が保障する権利』(有斐閣、1993年)\\
・門田孝「在留権」近藤敦編 『外国人の法的地位と人権擁護』(明石書店、2002年)60-67頁\\
・川村真理『難民の国際的保護』(現代人文社、2003年)\\
・国連難民高等弁務官(UNHCR)駐日事務所「難民認定基準ハンドブック ―難民の地位の認定の基準及び手続に関する手引き―〔改訂版〕」https://www.unhcr.org/jp/wp-content/uploads/sites/34/2017/06/HB\_web.pdf(2020年08月25日最終閲覧)\\
・児玉晃一「日本における難民申請者と収容」難民研究ジャーナル8号(2018年)38-48頁\\
・後藤光男『永住市民の人権』(成文堂、2016年)\\
・近藤敦『外国人の人権と市民権』(明石書店、2001年)\\
・近藤敦「外国人の『人権』保障」自由人権協会編『憲法の現在』(信山社、2005年)325頁\\
・近藤敦『人権法〔第2版〕』(日本評論社、2016年)\\
・在日米国大使館・領事館「2019年国別人権報告書―日本に関する部分」https://jp.usembassy.gov/ja/human-rights-report-2019-ja/ (2020年8月25日最終閲覧)\\
・佐々木惣一『改訂日本国憲法論〔補正版(再版)〕』(有斐閣、1954年)\\
・佐藤幸治『憲法〔第3版〕』(青林書院、1995年)\\
・佐藤幸治『日本国憲法論〔第2版〕』(成文堂、2020年)\\
・島田征夫『庇護権の研究』(成文堂、1983年)\\
・申\UTF{60E0}\UTF{4E30}『国際人権法からみた外国人の人権』自由と正義62巻2号11-18頁\\
・関聡介「続・日本の難民認定制度の現状と課題」難民研究ジャーナル2号(2012年)2-23頁\\
・関聡介「非正規滞在者の権利」近藤敦編『外国人の人権へのアプローチ』(明石書店、2015年)155-174頁\\
・田中宏『在日外国人―法の壁、心の溝〔第3版〕』(岩波新書、2013年)\\
・辻村みよ子『憲法〔第6版〕』(日本評論社、2018年)\\
・難民支援協会「日本にいる難民のQ\&A ―難民から見える世界と私たち―」\\https://www.refugee.or.jp/jar/postfile/QA.pdf (2020年8月25日最終閲覧)\\
・日本弁護士連合会人権擁護委員会編『難民認定実務マニュアル〔第2版〕』(現代人文社、2017年)\\
・長谷部恭男『憲法の理性〔増補新装版〕』(東京大学出版会、2016年)\\
・日比野勤「外国人の人権(1)」法学教室210号(1998年)35-44頁\\
・日比野勤「外国人の人権(2)」法学教室217号(1998年)43-55頁\\
・日比野勤「外国人の人権(3)」法学教室218号(1998年)65-82頁\\
・藤井俊夫『憲法と人権\ajRoman{1}』(成文堂、2008年)\\
・法学協会編『注解日本国憲法 上巻〔改訂版〕』(有斐閣、1953年)447頁\\
・法務省「令和元年における難民認定者数等について」http://www.moj.go.jp/content/001317678.pdf(2020年8月25日最終閲覧)\\
・法務省「難民認定審査の処理期間の公表について」\\http://www.moj.go.jp/nyuukokukanri/kouhou/nyuukokukanri03\_00029.html(2020年8月25日最終閲覧)\\
・山内敏弘「外国人の人権と国籍の検討」国際人権8号(1997年)2-7頁\\
・山神進『激変の時代 我が国と難民問題 昨日―今日―明日』(日本加除出版、2007年)\\
・横田耕一「人権の享有主体」芦部信喜ほか編『演習憲法』(青林書院、1984年)137-146頁\\
・米沢広一「国際社会と人権」樋口陽一編『講座憲法学(2) 主権と国際社会』(日本評論社、1994年)171-200頁\\
\end{document}