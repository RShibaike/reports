\documentclass[twoside]{jsarticle}
\usepackage[dvipdfmx,hidelinks]{hyperref}
\usepackage{pxjahyper}
\usepackage[jis2004]{otf}
\usepackage[dvipdfmx]{graphicx}
\usepackage{url}
\setcounter{page}{1}
\usepackage{fancyhdr}
\begin{document}
\title{ヘイトスピーチ規制の在り方と人種差別の構造}
\author{橋本泰樹}
\date{}
\maketitle
\tableofcontents
\clearpage
\pagestyle{fancy}
\lhead[ヘイトスピーチ規制の在り方と人種差別の構造]{}
\rhead[]{\leftmark}
\section{はじめに}

2016年5月24日、衆議院本会議において「本邦外出身者に対する不当な差別的言動の解消に向けた取組みの推進に関する法律案」(ヘイトスピーチ解消法)が可決、成立した。これは近年国内において特定の民族や国籍の集団へ向けられるヘイトスピーチが社会問題化したことを受けて、こうしたヘイトスピーチの解消に向けた取り組みを行う必要性について与野党間での認識が共有された結果であるといえる\footnote{魚住裕一郎ほか『ヘイトスピーチ解消法 成立の経緯と基本的な考え方』(第一法規・2016)3頁。}。本法は、差別的言動を許さないとする宣言をし、不当な差別的言動の解消に向けた取り組みを推進する理念法であり、ヘイトスピーチへの対策に消極的であった日本においては大きな意義を有するものではある。しかし現状では法律において民事、刑事の具体的なヘイトスピーチ規制がなされるには至っておらず、ヘイトスピーチの実効的な抑制や被害者の救済にはいまだ不十分である\footnote{川崎市で2019年12月に成立し、翌年7月には全面施行された「川崎市差別のない人権尊重のまちづくり条例」においては、条例の定義する不当な差別的言動を繰り返し行った者に罰則を科す規定が国内で初めて置かれている(同条例第23条)。}。一方で、ヘイトスピーチの規制は表現の自由との慎重な調整を要するものであり、その性質を精緻に分析して適切な規制の在り方を模索しなければならない。以下では、考えうるヘイトスピーチ規制の類型と課題、代替案等について検討していく。なお、ここでのヘイトスピーチとは不特定多数人に向けられるものを指していうこととする。

\section{個人的法益の保護による類型}

不当な差別的表現行為の規制手段としてまず想起されるのは刑法の名誉毀損罪(第230条)及び侮辱罪(第231条)である。しかし、名誉毀損及び侮辱といった法的類型は人格的法益たる名誉を保護するものであり、それは特定人を権利主体とするものである。したがって、民族や国籍などの集団に対する言論をこれらによって規制することは通常許されないものと考えられてきた。一方で、民族や国籍といったルーツは個人の人格の核心的部分を形成する事柄であり、それらを公然と否定するヘイトスピーチは当該集団に属する人々に深刻なダメージを与えるものであることが通例である。そこで、こうしたヘイトスピーチはその集団に属する個人の人格的法益たる名誉を毀損すると考えることはできないのだろうか。

ここで、ヘイトスピーチ規制の推進国であるといえるドイツにおけるこの規制類型について検討する。ドイツでは、ナチスによる全体主義的政策によりユダヤ人に対する不当な迫害、ひいてはホロコーストが行われた負の歴史が存在することから、ユダヤ人をはじめとする民族的マイノリティに対するヘイトスピーチの悪性が早くから認識されてきた。ドイツにおいても侮辱罪は本来個人の名誉を保護するものであるが、\UTF{2460}侮辱的表現が集団すべての構成員に共通するメルクマールと結びつき、\UTF{2461}名指しされた集団が区画可能で見渡すことのできる比較的小規模の集団であること、という成立要件のもと、ユダヤ人に対する侮辱表現を集団構成員全体への「集合的」侮辱として、侮辱罪に該当するとの判例がみられるようになっている\footnote{師岡康子『ヘイト・スピーチとは何か』(岩波書店・2013年)107頁。}。一方で、集団的侮辱を広く認めることは、政治的言論において集団に対する批判的言論を抑制することにもつながりかねないため、集団的侮辱の成立には個人の名誉との厳密な関連が必要であると解されている。そこでユダヤ人という、上記\UTF{2461}の要件を充足すると明言できるか定かでない民族的集団に対する集団的侮辱が認められるのは、ドイツにおけるユダヤ人の歴史的な特殊性によるもので、ドイツにおいても例外的な場合のみであるとする見解もある\footnote{毛利透「ヘイトスピーチの法的規制について
  -アメリカ・ドイツの比較法的考察-」法学論叢(京都大学)176巻2-3号(2014年)221頁。}。

他方、日本の判例理論が強く影響を受けているとされるアメリカは、思想の自由市場論が根強く、表現行為の規制に消極的な国家として知られている。名誉毀損や喧嘩言葉(暴力的衝突を生じさせる蓋然性の高いような方法で行われる侮辱的表現)といった表現の自由の濫用に当たる行為はアメリカにおいても憲法の保障の範囲外であるが、ヘイトスピーチが名誉毀損や喧嘩言葉との関連で規制可能であるかについては議論が複雑化していた。しかし、1952年のボハネイ事件判決において連邦最高裁は、「個人に対する名誉毀損に刑事罰が科されるのであるから、人種対立の歴史を有する州が、公共の秩序のために集団に対する名誉毀損に刑事罰を科すことも許される」とし、ヘイトスピーチを名誉毀損のカテゴリーの一つとして規制を合憲としている\footnote{師岡・前掲注3)143頁。}。一方この判例においては、ここでの名誉毀損の集団的側面を問題視し、名誉毀損罪がその範囲を個人に対する者に限定しているのは表現に対し処罰を課すにあたって純然たる私的な争い以上のものを含まないための趣旨であるとする裁判官の反対意見があった。これに対しては、ここでの名誉毀損の関心は人の根本的な評価の基盤であって、その場合一つの集団に属する多数の人々を、彼らの根本的な評判に向けられた攻撃から保護することが考えられるのであり、このときは個々人に対する影響を特定しようとすることに意味はないとする批判が唱えられている\footnote{ジェレミー・ウォルドロン(川岸令和・谷澤正嗣訳)『ヘイト・スピーチという危害』(みすず書房、2015)63頁。}。一方で、1969年のブランデンバーグ判決において典型的な差別煽動も表現の自由の保護を受けると具体的に示されて以降、アメリカでは表現の自由の原則を厳格に適用する方向へと向かい、ヘイトスピーチを事実上規制できないという状況が成立していったとされる\footnote{明戸隆浩「アメリカにおけるヘイトスピーチ規制論の歴史的文脈
  ── 90 年代の規制論争における公民権運動の「継承」」大阪経済法科大学
  アジア太平洋レビュー11号(2014年)29頁。}。

日本においても表現の自由は自己統治の価値、自己実現の価値により重要な権利として憲法21条において保障されている一方、名誉毀損や侮辱は表現の自由の濫用として刑法等によって規制されている。しかし京都朝鮮学校事件の裁判例\footnote{京都地判平成25年10月7日判時2208号74頁。}においても「単に人種差別がされたというだけでなく、これにより具体的な損害が発生している場合に初めて」損害賠償を命ずることができる、と判示されている通り、不特定の者に向けられた場合など、特定の者に具体的な権利・利益の侵害が生じていないとみなされる場合には現行法では救済は不可能であるとされている。

\section{公共秩序の維持という類型}

次に想起されるのは、ヘイトスピーチにより民衆煽動が行われ、暴力的事態の発生する可能性が生じることなどを理由とする規制類型である。保護の対象を特定人の個人的法益としない点では、前述の名誉毀損といった類型で検討した問題を克服するもののようにも思われるが、それはすなわち濫用の危険性が高いということも同時に意味する。政治的言論といった表現に抑制効果を生じさせることは避けなければならず、表現の自由の保障根拠や規制の要件などに関して極めて慎重な配慮が必要であるといえる。

ドイツでは、刑法の「公共の秩序を乱す罪」において「民衆煽動罪」が規定され、他人の人間の尊厳を攻撃する行為が規制対象となっている。そこにおいては住民の一部に対し公共の平穏を乱す態様で憎悪を掻き立てる行為が規制対象となっており、ヘイトスピーチの刑事規制の具体化と考えられている。しかしドイツの判例においても、表現の自由の保障の観点から民衆煽動罪の成立範囲を限定的に解する運用がなされており、例えば多義的な表現の規制においては規制の対象外となりうる解釈を排除する必要があると考えられている\footnote{毛利・前掲注4)225頁。}。どの程度の言論が民衆の煽動として規制対象になるのかについて検討すると、まずある表現がその受け手に不安感等も含め何らかの感情を抱かせる可能性があることは当然のことであり、これをもって公共の平穏を害するとして規制対象とすることは許されない。すなわち公共への危険があると認められるためには、受け手を直接に煽動するような訴えかけなど、客観的にみて公共の平穏を脅かす兆候であると認められるようなものでなければならないのが原則であるといえる。この点について、ドイツにおける民衆煽動罪によってヘイトスピーチの規制が可能であるのは、ナチス支配により現実に残虐な行為が行われてきた歴史の存在により、民族的マイノリティ等に対する差別的言動が具体的・客観的な公共秩序の危険の兆候として認められるためと解釈するべきであるとする指摘もある\footnote{毛利・前掲注4)224頁。}。

アメリカにおいては、前述の通り思想の自由市場論が根強く、表現の自由の保障範囲である表現行為をその内容によって規制するためには、当該表現行為が違法行為を引き起こす明白かつ現在の危険があると認められなければならないとする法理が用いられてきている。ここにいう明白かつ現在の基準とは、その表現が害悪を惹起する蓋然性、その害悪の重大性や時間的切迫性、規制手段が必要不可欠であること、という極めて厳格な基準によってのみ制限を正当化するものであり、先に述べたブランデンバーグ事件判決においても意識的に用いられたものである\footnote{芦部信喜(高橋和之補訂)『憲法
  第7版』(岩波書店・2019)217、218頁。}。これは表現内容の自由を極めて厳格に保障するものであるといえ、表現行為が憲法の保障の範囲外のものであると認められない限りは、ヘイトスピーチを公共秩序の維持を目的として規制することが可能であるのはごく限られた場合のみであると考えられる。

また、労働力人口の増加を目的とした移民流入政策等により多民族社会を形成してきたイギリスでは、1936年に公共秩序法が制定されて以来、公共秩序の維持の類型によるヘイトスピーチ規制が行われてきた。しかし厳格な適用要件が設けられたこともあって適用件数は極めて少なく、またヘイトスピーチ以外の言論や、マイノリティ側からの政治的言論に適用される例もみられた。マイノリティへの差別をその悪性の本質とするヘイトスピーチの規制において公共秩序の維持をその目的に掲げることが、悪影響の強いヘイトスピーチが厳格な要件のもとで規制を免れる事態や、あるいは一方では規制法の濫用により不当に表現行為を制限する事態にも繋がりかねないことを示しているものと考えられている\footnote{師岡・前掲注3)90頁、101頁。}。

\section{ヘイトスピーチ規制における個別論点}

\subsection{対抗言論の有用性に関連して}

ヘイトスピーチ規制に反対する論者の多くは、思想の自由市場の観点から、悪質なヘイトスピーチにも言論によって対抗することが表現の自由の趣旨にかなうとする。一方で規制論者の多くは、ヘイトスピーチには対抗言論が困難である場合が多いため、表現の自由の保障における前提が事実上崩壊しており、特別の配慮が必要であると考えるべきであるとする。よって以下ではこの論点について考察を深めていく。

思想の自由市場論は、あらゆる意見・思想は世間に発表され自由市場において評価を得るべきであり、そこで競争原理が働くことによって真実やより良い意見・思想が生き残ることになるという考え方である。すなわちこの考え方においては、意見・思想を発表する自由が必ず保障されなければならず、それはヘイトスピーチのような差別的言論であっても原則は同様である。それに対して規制論者の多くは、ヘイトスピーチのようなマジョリティからマイノリティへと向けられる差別的言論には沈黙効果が生じるため、マイノリティが対抗言論を行うことは極めて困難な場合が多いという。沈黙効果とはすなわち、ヘイトスピーチに直面することで自己喪失感と無力感により言葉を失ったり、あるいは対抗言論を行うことにより更なる過激な攻撃を受けることが予想されることにより表現を萎縮してしまうことをいう\footnote{師岡・前掲注3)58頁。}。これにより言論による対抗という規制消極論者が主張するヘイトスピーチへの対応方法は現実的ではないと考えられ、思想の自由市場論の前提が失われることになり、ヘイトスピーチ規制へと傾く考慮要素となる。

確かにヘイトスピーチにこうした沈黙効果が存在し、対抗言論を困難にしていることは否めないと考えられる。また、マイノリティのための対抗言論を行う場を設けるなどしても、更なる攻撃の誘因が予想されることによる萎縮効果が拭い去られるわけではない。しかし、こうした言論による対抗言論の困難という効果は、その大きさに差異はあれどヘイトスピーチ以外の言論等にも発生しうるものである。確かにヘイトスピーチにおいてはマジョリティからマイノリティへ一方的に行われるという構造が、看過できない沈黙効果を発生させる点で特殊かつ悪質ではあるが、対抗言論の困難さのみを直接の理由として表現規制を是認することが許されるわけではないように思われる。

そもそも思想の自由市場論における、競争原理によってより良い意見・思想が生き残るという考えはどこまで現実的に想定可能なのだろうか。現代社会において情報の発信者としての規模の大きさや把握する情報量は非対照的であり、インターネットの発達を無視することはできないにしても、完全競争市場のようなものを想定することは難しいようにも思われる。市場における競争原理による分配を想定してみると、モデルとなる経済市場においては政府による一定の規制による調整は必要とされているはずである。とは言っても、言論市場の場合には、表現の自由が脆弱な権利であり、政府の表現内容規制による言論の方向の統制を認めることには看過できないリスクが存在することも事実である。この点で経済市場と言論市場は異質であるため、とりわけ政治的言論との境界を定めることに困難の伴うヘイトスピーチにおいて、安易にその規制を認めることはできない。この点で、ヘイトスピーチ規制が(児童ポルノ規制などのように)本来必要なあるいは理にかなった立法上の企てとして理解されているとする議論\footnote{ウォルドロン・前掲注6)185~188頁。}には、必ずしも賛成できない。一方で、情報の自由市場においては正しい情報が生き残るとは限らず、感情に強く訴えかける情報ほど盛んに流布されるため、流言やデマの悪性を教育やメディアを通して広めることの重要性を指摘する議論もある\footnote{高史明『レイシズムを解剖する 在日コリアンへの偏見とインターネット』(勁草書房・2015年)182~185頁。}。この点について、歴史的事実に反する説の流布などにより悪質な差別表現の蔓延に至るような場合には、政府から独立した第三者機関等による検証結果を指針として発表することは許されるべきであるという指摘もある\footnote{櫻庭総「刑法における表現の自由の限界 ヘイト・スピーチの明確性と歴史性の関係」金尚均編『ヘイト・スピーチの法的研究』(法律文化社・2014)122頁以下。}。これらを検討すると、ヘイトスピーチによる沈黙効果という観点からは、言論の規制というよりは、マイノリティ側からの対抗言論となるような真実の情報や、デマ情報の悪性等を第三者的な機関が広く発信するための仕組みが必要だと考えることもできるかもしれない。一方で、こうした仕組みによってもマイノリティ側からの対抗言論が成立しえないといえるような場合には、法規制による調整が必要だと考えるべきだといえる。

\subsection{濫用の危険をどう考えるか}

表現行為の規制を考えるにあたって慎重な検討が必要であるのが、規制法の濫用の危険についてである。個人の人格の本質的部分たる人種や出自を否定するヘイトスピーチの悪性に対処し、人の尊厳を守るために作られる規制法が、ヘイトスピーチとは言えないような言論に適用されることにより不当に言論が統制されたり、あるいは漠然不明確な要件規定により表現行為への萎縮をもたらすようなことはあってはならない。

前述のように、イギリスにおける公共秩序の維持の類型によるヘイトスピーチ規制においては、数少ない適用事例のうちにマイノリティ側からの言論が取り締まられたものが含まれていた。これは、本来マジョリティからマイノリティへ一方的になされるヘイトスピーチから人の尊厳を守ることを目的として作られる規制法が、公共秩序の維持を建前上目的としているために起こる帰結である。このような類型の規制法においてあくまで公共の秩序の維持という目的が掲げられるのは、当該規制は表現がもたらす結果のために行われるものであるとすることによって、表現の内容に基づく規制とすることを回避するためであるといえる。確かに、表現の内容による規制は言論の統制であるともいえ、とりわけ民主政に資する政治的言論がその内容によって統制されることは決してあってはならない。その意味で、表現の結果や態様による規制よりも表現の内容による規制は厳格に検討されなければならず、必要不可欠な価値を保護するための必要最小限度の手段によるものでなければならない。しかし一方で、公共の危険といった表現の態様による規制も表現の規制であることは同様であり、表現の内容による規制でないことを理由にして広汎な規制が行われ、本来規制すべきでない表現まで規制されてしまうようなことはあってはならない。また公共秩序といった概念は曖昧であり、危険が生じる程度の判断が恣意的になってしまうことにより規制法が濫用される可能性に常に配慮しなければならない。そうすると、政治的言論との厳密な区別が必要となるヘイトスピーチについて、専ら公共秩序の維持という範疇で規制の在り方を考えることは適切ではない。以上のことを考慮すると、ヘイトスピーチ規制を考えるにあたっては、マイノリティの個人的法益、あるいは人間の尊厳という観点を何らかの形で含める必要があると考えるべきである。

また萎縮効果については、ヘイトスピーチの定義を厳格に行うなどによって萎縮効果を最小限にする工夫を行うことが不可欠である。具体的には、ヘイトスピーチの対象となる集団の外縁を限定することなどが重要となる。ヘイトスピーチの定義について、2016年1月18日に施行された大阪市ヘイトスピーチへの対処に関する条例\footnote{大阪市:「大阪市ヘイトスピーチへの対処に関する条例」の運用について\\\href{https://www.city.osaka.lg.jp/shimin/page/0000339043.html}{{https://www.city.osaka.lg.jp/shimin/page/0000339043.html}}。}における定義をみると、この条例においては集団に対するヘイトスピーチについて、「人種若しくは民族に係る特定の属性を有する個人又は当該個人により構成される集団」を特定人等と定義したうえで、特定人等に対し侮辱又は誹謗中傷するものであること、又は脅威を感じさせるものであることという条件のもと、\UTF{2460}特定人等を社会から排除すること、\UTF{2461}特定人等の権利または自由を制限すること、\UTF{2462}特定人等に対する憎悪若しくは差別の意識又は暴力をあおること、のいずれかを目的とするもの、という形式でヘイトスピーチを定義している。これは、集団を対象とする表現行為に関して、「人種・民族に係る特定の属性を有する個人により構成される集団」としてその外縁を限定しており、政治的言論を定義に含めない工夫が各所になされているものと考えることができる。一方で、ヘイトスピーチが刑事規制されることを想定するようなときは、この定義では権力により外縁を広く解釈することによる濫用の危険が否定できないとも考えられる。規制法においてはマジョリティからマイノリティへ一方的になされるヘイトスピーチの悪性を厳密に定義するため、規制の対象を民族的マイノリティに対する言論に限定することが重要であるとする指摘もあり\footnote{師岡・前掲注3)209頁。}、平等原則を念頭に置きつつ検討していかなければならない点である。

\section{代替手段の有効性の検討}

ヘイトスピーチという表現行為を規制するにあたって検討が必要なのが、当該表現行為を法的に規制することが必要最小限度の手段なのかという点である。規制を行うという前提に立ったときにはその内部においても問題になる議論であるが、そもそもヘイトスピーチからマイノリティの尊厳を守るという規制法の目的が表現行為の規制以外の手段によって達成可能なものではないのかが問題になる。とりわけヘイトスピーチ規制消極論者の多くは、ヘイトスピーチ規制の代替手段として、人権教育の徹底や、政府が人種差別を許さない姿勢を明らかにする、といった事柄を挙げ、表現の自由を制限するのではなくそれらによってヘイトスピーチを抑制することに力を注ぐべきであるとする。しかし、ヘイトスピーチによって現実に被害が発生している現状や、それら被害が切実なものであることを考えると、そうした代替手段は決して絵に描いた餅であってはならないはずである。もちろん、人権教育といった代替手段の有効性を厳密に調査することは困難であり、ヘイトスピーチ抑制のためにとることのできる手段の一つとして一定の効果をあげると推定できるものではあるが、ここでは人種差別がどのように生まれるのかについての考察を通して、それらの代替手段がどういったものであるべきかや、それらの有効性について考えてみたい。

人種差別に関して言えば、その構造には古典的レイシズムと現代的レイシズムが存在するといわれる。前者はすなわち、ある人種は別の人種より劣っている、といったようなプロトタイプ的なレイシズムである。それに対し後者は、ある人種に対する差別は解消されているにも関わらず、その人種の人々は自己責任による自らの処遇を差別に転嫁して抗議し、不当な特権を得ている、というような発想に基づくレイシズムである。これらはアメリカにおける黒人差別の分析において用いられてきた概念であるが、在日コリアンに対する差別においてもこの2つの概念において区別可能であるとされており\footnote{高史明・雨宮有里「在日コリアンに対する古典的/現代的レイシズムについての基礎的検討」社会心理学研究
  28巻2号(2013年)68頁以下。}、日本において在日コリアンに対し「在日特権」といった言説が流布することによって生じるレイシズムも現代的レイシズムとの親和性の高いものと思われる。これについて考えると、そもそも現代的レイシズムは差別であると自覚されにくいため抑制が難しく、人種間の平等といった単純な人権教育によって発生が防がれるものではないのではないだろうか。特に情報発信の容易なインターネット、SNS上ではデマ情報の拡散がなされやすく、インターネットやSNSの利用によってそうした情報には日常的に触れることになる。インターネットの使用時間に現代的レイシズムを強める効果があり、インターネットでそうした情報に接触しているうちに個人の信念に基づくレイシズムが強まるのではないかと考察する研究も存在する\footnote{高・前掲注15)137頁。}。そうすると、現代においてレイシズムないしはヘイトスピーチを抑制するための人権教育においては、人種間の平等といった観点のみならず、ネットやSNSを使用する上でのデマ情報等に惑わされないための情報リテラシー教育や、個人は情報源の選択において自己の信念に合致するものを選んでしまいがちであるということを自覚させるような道徳教育がより重要となってくるのではないだろうか。

また、外交問題や外国人による犯罪の報道等により、無関係な在日外国人らに差別意識が向けられる現象にも問題があるように思われる。例えば、2012年8月には韓国の李明博大統領が竹島に上陸し、日本政府も強硬姿勢を示すなど領土問題を中心に緊張した外交関係にあったが、当時はそうした状況を背景として排外主義デモが活発化したといわれる。また同年12月には朝鮮学校が高校無償化制度から除外する方針が発表されているが、これは拉致問題に進展がなく国民の理解が得られない、といった理由によるものであった。朝鮮学校の生徒には拉致問題に責任がないことは明らかであるにもかかわらず、拉致問題と朝鮮学校への政策を結び付けることが政府によって行われていると解釈できるものでもあり、こうした政府の態度やそれらの報道のされ方が国内における排外主義的な思想の涵養に繋がっているとみることもできるのではないだろうか。社会的な不安感によって排外主義的になるという「草の根のレイシズム」が、そうした国家やマスコミによる政策や報道などによる「上から作られたレイシズム」によって方向づけられ発展していく、とする指摘もあるところである\footnote{森千香子「ヘイト・スピーチとレイシズムの関係性 なぜ、今それを問わなければならないのか」金・前掲注16)11頁。}。その意味では、前述の「政府が人種差別を許さない姿勢を明らかにする」といった方策にも有効性があると考えられ、理念法に留まるとはいえ冒頭に述べたヘイトスピーチ解消法が成立したことには意義があると考えることができる。

\section{考察など}

以上では、ヘイトスピーチの規制類型ごとのアプローチや個別論点、代替手段等について検討してきた。それらを踏まえ、日本においてはヘイトスピーチに対してどのような対処が考えられるであろうか。

前述の大阪市条例ではヘイトスピーチの拡散防止措置や実行者の氏名等公表が定められていた。氏名等公表が公表された者の名誉を害するなど、抑止効果を包含する罰則的な意味合いを持ちうることについては検討の余地があるが、基本的には刑罰を科すことのような規定はおかれていない。ある表現行為が行われたことに対して刑罰が科される条項は、厳格な要件を定めたとしても少しでも規制の対象となる言論の外縁が不明確であれば権力による濫用の危険性は高くなるといえ、立法事実に基づく厳密な要件を定義しない限りは刑事規制は許されるべきではないだろう。この点、表現規制による表現の萎縮や濫用の懸念の問題は、公務員による場合に限定すればほとんど生じないとし、まず公務員によるヘイトスピーチに限定して刑事規制を行うべきであるとする見解\footnote{師岡・前掲注3)211頁。}もあるが、いずれにせよ慎重な検討が必要である。

なお、大阪市条例の合憲性が争われた裁判例\footnote{大阪地判2020年1月17日裁判所ウェブサイト\href{https://www.courts.go.jp/app/files/hanrei_jp/318/089318_hanrei.pdf}{{089318\_hanrei.pdf
  (courts.go.jp)}}。}において大阪地裁は、条例の定める特定集団に対するヘイトスピーチの拡散防止措置等について、ヘイトスピーチにより当該人種又は当該民族に対する偏見、差別意識、憎悪等の感情が助長、増幅等されることやこれらの感情が当該人種又は当該民族に属する特定人に対する当該人種又は当該民族に関する侮蔑又は誹謗中傷や暴力行為へと進展することを抑止することを目的としているものであるとし、特定の人種や民族に属することが個人の人格の核心を形成するものであること等を考慮したうえでこの目的を正当とした。また手段の合理性についても、処罰規定が設けられていないことや、当該措置に先立って学識経験者などによる附属機関による諮問が予定されていることなどから表現行為者への制約は大きいものではないとして合理的であるとされた。ヘイトスピーチの定義を限定しつつ、民衆煽動のみならず個人の人格的利益を考慮して条例の目的を認定し、また慎重かつ表現者の制約の少ない手段がとられていることから本条例が合憲であるとされたものであり、判決の示す通り本条例は悪質なヘイトスピーチへの現実的な規制方法として評価できるものであるといえる。一方、条例の規定する氏名等公表は処罰規定でないとはいえ一定の表現抑止効果を有するものであることは否めないものであるとも思われ、萎縮効果についてはより慎重な議論が必要であったともいえる。

一方で、川崎市で2020年7月に全面施行された「川崎市差別のない人権尊重のまちづくり条例」は、規制の対象となる差別的言動を限定的に定義したうえで、それを繰り返し行った者に罰則を科す規定を設けている。同条例第12条は、条例の対象とする「不当な差別的言動」を場所、手段、類型の要件から明文で規定しており、それによると、市の区域内の公共の場所において(場所)、拡声器の使用、看板・プラカード等の掲示、ビラ・パンフレット等の配布により(手段)、本邦外出身者をその居住する地域から退去させることを煽動又は告知するもの、本邦外出身者の生命・身体・自由・名誉・財産に危害を加えることを煽動又は告知するもの、本邦外出身者を人以外の者にたとえるなど著しく侮辱するもの(類型)が禁止される。このうち特に類型については、同条例の解釈指針\footnote{川崎市:「川崎市差別のない人権尊重のまちづくり条例」解釈指針\\\href{http://www.city.kawasaki.jp/250/cmsfiles/contents/0000113/113041/jyourei1.pdf}{{http://www.city.kawasaki.jp/250/cmsfiles/contents/0000113/113041/jyourei1.pdf}}。}において具体的な言動例が挙げられているが、個別具体の言動が規制対象となるかは文脈や趣旨等の諸事情を勘案する必要があるとされ、審査会の意見を聴いたうえで市長が最終判断を下すとされている。また、同解釈指針によると本条の保護法益は「居住する地域において平穏に生活する権利」である。不当な差別的言動を繰り返し行うなどして同条例第14条1項の命令に違反した者に対する罰則を定めた第23条は、その解釈指針によると、前述のヘイトスピーチ解消法の立法の契機となったデモ等が川崎市において行われてきたという地域の実情をふまえた施策であるとされている。

同条例のいう「本邦外出身者」についてはヘイトスピーチ解消法第2条の定義が参照されており、必ずしも民族的マイノリティに対する言論に限定されるわけではなく、その意味で外縁は不明確だともいいうる。一方、表現の自由への配慮から規制対象の表現をその手段や類型の点から具体的に限定する工夫がなされており、罰則までに勧告、命令や審査会による審査などの手続的な配慮がなされている点が注目される。また、当条例の罰則規定が、川崎市内において悪質なヘイトスピーチが行われてきた立法事実を踏まえた規定であることも運用において注意されるべきだと考えられる。

以上の点を考慮すると今後の国内におけるヘイトスピーチ規制は、少なくとも刑事規制については法律レベルにおいてではなく、ヘイトスピーチ解消法第4条第2項の規定を受けた地方条例の形で、各地域ごとの具体的な立法事実に基づいた規制の在り方が現実的であると考えるべきである。規制対象となる表現行為の具体的な限定、手続的配慮や運用の指針となるべき立法事実など条例レベルでの対応を各自治体で検討していく必要があり、その意味でも川崎市条例の今後の運用が注目される。

ヘイトスピーチに対する法的規制は、ヘイトスピーチによる被害が重大であるにもかかわらず、表現行為の規制法として表現の自由との調整に大きな問題があり、政治的言論との区別という難点を孕んでいる。上述のような各論点に配慮した法律・条例レベルでの立法措置の構想は必要であるが、表現の自由に配慮しながら十分な規制の網を敷いていくこと、ひいてはヘイトスピーチの根本としての人々の無自覚なものも含めた差別意識を変容させていくことには自ずから限界がある。人種・民族差別を許さないという意識のみならず、無自覚な差別意識の存在に向き合うことやネットリテラシーの涵養といった、現代のヘイトスピーチの根源としての人々の人種差別意識を変革していくための教育の重要性は、今後一層高まっていくと考えられる。

\section{参考文献}

\noindent
・魚住裕一郎ほか『ヘイトスピーチ解消法 成立の経緯と基本的な考え方』(第一法規・2016年)。\\
・師岡康子『ヘイト・スピーチとは何か』(岩波書店・2013年)。\\
・石崎学・遠藤比呂通編『沈黙する人権』(法律文化社・2012年)。\\
・金尚均編『ヘイト・スピーチの法的研究』(法律文化社・2014年)。\\
・芦部信喜(高橋和之補訂)『憲法 第7版』(岩波書店・2019年)。\\
・高史明『レイシズムを解剖する 在日コリアンへの偏見とインターネット』(勁草書房・2015年)。\\
・ジェレミー・ウォルドロン(川岸令和・谷澤正嗣訳)『ヘイト・スピーチという危害』(みすず書房、2015年)。\\
・奈須祐治「ヘイト・スピーチと理論
日本の学説の整理と検討(1)(2・完)」西南学院大 学法学論集51巻2号、3・4
号(2018、2019年)。\\
・毛利透「ヘイトスピーチの法的規制について
-アメリカ・ドイツの比較法的考察-」法学論叢(京都大学)176巻2-3号(2014年)。\\
・明戸隆浩「アメリカにおけるヘイトスピーチ規制論の歴史的文脈─90年代の規制論争における公民権運動の「継承」」大阪経済法科大学
アジア太平洋レビュー11号(2014年)。\\
・高史明・雨宮有里「在日コリアンに対する古典的/現代的レイシズムについての基礎的検討」社会心理学研究
28巻2号(2013年)。
\end{document}