\documentclass[twoside]{jsarticle}
\usepackage[dvipdfmx,hidelinks]{hyperref}
\usepackage{pxjahyper}
\usepackage[jis2004]{otf}
\usepackage[dvipdfmx]{graphicx}
\usepackage{url}
\setcounter{page}{1}
\usepackage{fancyhdr}
\begin{document}
\title{同性婚、同性パートナーシップ制度と憲法}
\author{河村和賀子}
\date{}
\maketitle
\tableofcontents
\clearpage
\pagestyle{fancy}
\lhead[同性婚、同性パートナーシップ制度と憲法]{}
\rhead[]{\leftmark}
\section{婚姻とは}

\subsection{制度としての婚姻}

日本では、法律上の方式により婚姻を成立させる法律婚主義を採用しており、その方式とは婚姻届の提出である。戸籍法15条より、身分関係の変動は、原則として、当事者の届出に基づいて戸籍に記載されるため、婚姻はその届出によってはじめて、法律効果を生ずる。

家族関係の公示は本人にとって必要なだけではなく、それと関わりを持つ第三者にとってもまた、債権の存在や相続に関わる事項であるため、重要な意味を持つからだ\footnote{中川淳・小川富之編『家族法』(法律文化社、2013年)13,14,24頁}。

つまり、婚姻とは、家族関係を戸籍に反映することで、それから派生して家族関係によって生じる他の様々な権利・義務の根幹となる制度である。

\subsection{婚姻の要件}

婚姻の要件は実質的要件として\UTF{2460}婚姻意思の合致\UTF{2461}婚姻障害がないことあり、形式的要件として届出が必要とされる。つまり、憲法24条1項の「両性の合意」があること、民法731条~741条までの要件を満たすことであり、その主なものとしては婚姻適齢に達していること、重婚でないこと、女性は再婚禁止期間を経過していること、以上の条件を満たした上で民法739条の婚姻届の提出をすることで効力が生じる。

現行制度下では、生殖機能がなくても、男女間であれば婚姻できる。なぜなら、生殖機能を婚姻の要件とする法律の規定もなければ、実際に子どものいない、できない状態の夫婦もいる。また、生殖行為が客観的に不可能な状態であるといえる獄中婚や臨終婚も可能である。さらに、性同一性障害の診断を受けると、自らの生殖腺を失うことが性自認に合致する婚姻の要件となる。そのため、明らかに生殖できない状態でなら、戸籍上の異性と婚姻することができる\footnote{同性婚人権救済弁護団編『同性婚 だれもが自由に結婚する権利』(明石書店,2016年)162頁〔三輪晃義〕}。以上より、明示的に示されてはいないが、制度上の取り扱いから、戸籍上異性であることも婚姻の要件だということができる。

\subsection{婚姻の効果}

主な婚姻制度の効果としては、人格的効果・一般的効果として民法750条の氏の共同や752条の同居協力扶助義務がある。また、財産的効果として755条にあるように夫婦財産契約をしていない限り、760条の婚姻費用の分担や761条の日常家事債務、762条の夫婦間における財産の帰属が適用される。

これを整理すると、婚姻により協力や貞操の義務が生じ、一方的に解消できないことで婚姻関係を実質化・強化する、また、社会の構成単位としての法律婚家族として戸籍上にその関係性が反映されるという法的・社会的経済的利益と、2人の関係性について社会からの承認・認知を得ることという心理的・社会的利益の2種類の利益が法律婚によって得られることになる\footnote{同性婚人権救済弁護団編『同性婚 だれもが自由に結婚する権利』(明石書店,2016年)150頁〔永野靖・岸本英嗣〕}。

\section{日本社会における同性愛(者)}

\subsection{同性愛者とは}

同性愛者は自らの性自認は生物学上の性と一致しているが、性的指向が自らと同じ性を対象としている人のことである。

性自認とは、その人が自分自身の性別をどう思っているかに関する、ある程度持続的な自己意識、アイデンティティのことであり、性的指向とは、魅力を感じる性別の方向性のことである\footnote{石田仁『はじめて学ぶLGBT 基礎からトレンドまで』(ナツメ社、2019年)18,19頁}。

\subsection{同性愛者への歴史的差別}

同性愛は歴史的に、自然で正当なセクシュアリティとしての異性愛の対として、不自然で異常なものとされてきた。また、それは身体の異常であり、本人の意識次第で治るものである精神病だとされ、異性愛とは単なる対ではなく、上下関係・優劣関係の中で捉えられた\footnote{中里見博「日本における同性愛抑圧の実態と特徴」三成美保編『同性愛をめぐる歴史と法 尊厳としてのセクシュアリティ』(明石書店、2015年)76-81頁}。   

そもそも自分の性的な興味・関心について理屈で説明することは難しく、性とは衝動的なものであり、意識では変えられないものである。そのため、本来このような優劣関係はなく、ただ生物学上の性と同一の性自認をもつ異性愛者が多かったために、それ以外の性的マイノリティへの理解がされて来なかっただけである。

\subsection{日本での同性愛者}

日本の性教育は戦後に学習指導要領に組み込まれ、社会教育の観点から性的純白を守らせる目的で、禁欲主義・純潔主義的で固定的性差観の強調を特徴とした。そのため、性的欲求や性そのものに対してネガティブなイメージを与えうるものであった。\footnote{石田・前掲注4)60,61頁}

性的マイノリティについての情報が少ない中で、テレビの表現などでオネエタレントがあからさまに性的誘惑をするさまを強調したり面白がったりして同性愛者への偏見が助長されている面も否めない。

その人数的な多さに加え、後ほど述べるように制度的にも現在の日本社会は異性愛者中心社会といえる。その中で、異性愛者は無自覚に与えられる性的指向をなんの障害もなくそのまま受け入れ、異性愛者であるという性的アイデンティティを持つことすらなく、自然に異性愛者となる。一方、同性愛者には、異性を愛することは自然なことではなく、自らは自然と同性を愛しているということに気づき、自らがマジョリティとは違うことを認識する契機が必ずある。異性愛者であることはその人たちのアイデンティティであり、カムアウトできない状況のせいでその権利は主張されづらいが、確かに存在している、保護すべき尊厳である\footnote{二宮周平「家族法 同性婚への道のりと課題」三成美保編『同性愛をめぐる歴史と法 尊厳としてのセクシュアリティ』(明石書店、2015年)99頁}。

自分の性的興味・関心と相容れない性行為や普段意識することのない性的トピックへの違和感や拒否感があるのは仕方のないことだ。しかし、日本の社会では未だに性的マイノリティへの無理解や知識の欠如から、理解することでなくすことのできる、理屈抜きの嫌悪感などの表現である偏見や差別がある。

この日本型ホモフォビア(同性愛への恐怖)では、明らかな差別の対象とはしないが、社会的にはその存在を認めず、抹消・無効化するという抑圧をしている。このような社会の在り方によって原初的孤立と自己否定へつながる。原初的孤立とは自らの性向に気づいた当初に、家族であっても共通する性質ではないために、身近に自らと似た存在を見つけられず、打ち明けることも難しくなり、孤立することである。また、自己否定は社会的に異性愛者であることが自然なこととされていることはすでに理解しているために、「みんな」とは違う自らの性向を受け入れられなくて自己を否定することである。\footnote{中里見・前掲注5)70,76-81頁}同性愛者には自己否定から自殺を考えたことがある、自殺未遂の経験がある人の割合がそうでない人に比べて高い。\footnote{日高
  庸晴 『LGBT 当事者の意識調査 ~いじめ問題と職場環境等の課題~』

   \href{https://www.health-issue.jp/reach_online2016_report.pdf}{{https://www.health-issue.jp/reach\_online2016\_report.pdf}} }このように生きづらさを感じている同性愛者が社会の中でアイデンティティを保ちながら存在できるようにすることが差別の解消である。

日本では、同性愛それ自体を差別したり、同性間の性行為を処罰の対象としたりする立法は行われておらず、積極的に同性愛者が弾圧されているわけではない。しかし、婚姻・家族制度の中で、同性婚を認めないことは、同性愛を逸脱視する法であるといえる。なぜなら、婚姻からの排除は行政サービスからの排除に波及するからだ。同性愛を逸脱視する社会文化規範と、こうした法制度の在り方が、相互補完的に働くことで、日本での同性愛の抑圧は、法的根拠のある、堅固なものとして経験されることになる。\footnote{中里見・前掲注5)81-83頁}つまり、同性愛者を法制度上で抑圧はしていないものの、その存在がないことを前提とし、無視することによって、実際上の不利益や制度からの排除が生じている。また、これにより、異性愛者にとって同性愛者や同性カップルの存在がいつまでも可視化されないため、抹消効果が上塗りされることにもなるはずだ。

\section{同性婚}

\subsection{同性婚が認められない現状}

現在、異性カップルと同様の法律婚の適用は、同性カップルには認められていない。

2014年に青森市で同性カップルが婚姻届を提出したが、青森市役所は青森地方法務局の助言のもと、憲法24条1項を根拠として不受理とした。\footnote{\href{http://www.city.aomori.aomori.jp/inquiry/detail?sheet-no=6219}{{http://www.city.aomori.aomori.jp/inquiry/detail?sheet-no=6219}} 青森市役所ホームページ 「市民の声 同性の婚姻届の不受理事由について」より\\
  また、本人たちは問題提起、同性カップルの存在を公権力や世間に知らせるために行い、受理されることは期待していなかったこともあり、不受理の合憲性などについて裁判では争われていない。\\ \href{https://news.yahoo.co.jp/articles/e80e5f5779a8553a7fc64adc868fc9a6a188a9a7?page=3}{{https://news.yahoo.co.jp/articles/e80e5f5779a8553a7fc64adc868fc9a6a188a9a7?page=3}} }そして、現在も全国で「婚姻の自由をすべての人に」をスローガンに同性婚を認めない現状を違憲状態だとして争う裁判が行われている。

国会での議論においては、制定当時は想定していなかった同性婚と憲法との関係に関する質問\footnote{\href{http://www.shugiin.go.jp/internet/itdb_shitsumon_pdf_s.nsf/html/shitsumon/pdfS/a201034.pdf/$File/a201034.pdf}{{http://www.shugiin.go.jp/internet/itdb\_shitsumon\_pdf\_s.nsf/html/shitsumon/pdfS/a201034.pdf/\$File/a201034.pdf}} }に対して、2020年の6月23日の衆議院での答弁において安倍晋三首相は、「民法(明治二十九年法律第八十九号)において、婚姻の当事者である「夫婦」とは、男である夫と女である妻を意味しており、同性婚は認められておらず、同性婚をしようとする者の婚姻の届出を受理することはできない。」、「民法において、同性婚をしようとする者の婚姻の届出を受理することはできないことから、国勢調査において、「世帯主」と性別が同一である者については、婚姻関係にある者と同様のものとして取り扱っていない。」と述べた\footnote{\href{http://www.shugiin.go.jp/internet/itdb_shitsumon.nsf/html/shitsumon/b201250.htm}{{http://www.shugiin.go.jp/internet/itdb\_shitsumon.nsf/html/shitsumon/b201250.htm}} }。制定当時に同性婚を想定していなかったため、現行の規定では同性婚は認められないと主張し、異性愛を前提とする家族関係しか認めようとしない古風な姿勢を貫いている。

\subsection{同性婚が認められないことによる非侵害利益}

婚姻は家族関係の根幹であり、そこから派生する権利や義務が多いことから、同性カップルが婚姻制度を利用できないことで侵害される権利や利益は多い。現在係属中の「婚姻の自由をすべての人に」訴訟事件の原告の主張の中では、民法上の権利として扶助協力義務、相続権、財産共有推定・財産分与、共同親権が、税制上の権利として所得税・住民税・相続税の控除、その他にも在留資格、遺族年金・給付金、公営住宅への入居、DV防止法に基づく保護、事実上の権利として本人を代理する医療同意、民間住宅への入居、住宅購入のペアローンの適用が受けられないことを挙げている。

これらの権利の中で、公営住宅への入居が認められないことは自由権規約委員会から日本に対して自由権規約 条違反だとして問題が指摘されており、第6回政府報告を受けて公営住宅法の改正が行われた。それにもかかわらず、実際には同性カップルの入居が認められないままであることに対して2016年12月の政府回答は「公営住宅法の改正により、いわゆる同居親族要件は撤廃したところであるから、法制度上、同性カップルは公営住宅制度から排除されているわけではない。同性カップルを含めていかなる者を公営住宅に入居させるかについては、各地方公共団体の判断にゆだねられている。」というものであった。\footnote{\href{https://www.mofa.go.jp/mofaj/files/000031106.pdf}{{https://www.mofa.go.jp/mofaj/files/000031106.pdf}} }

その後、あらゆる差別的取扱いを包括的に禁じる規約に対して第7回政府報告では、「我が国は,性的指向・性自認を理由とした人権侵害に反対するとの立場をとっている。」と明言し\footnote{\href{https://www.mofa.go.jp/mofaj/files/100045760.pdf}{{https://www.mofa.go.jp/mofaj/files/100045760.pdf}} }、政府の人権侵害の事情についての理解が進んでいることがうかがえる。しかし、同性愛者の公営住宅への入居については各地方公共団体に委任していると繰り返し、実際には地方自治体任せであるということもできる。

法内容の平等が達成されても、その適用や実務において差別的取扱いが続いているこの状況は、アメリカの公民権運動において、ブラウン判決で黒人差別は違憲とされても、社会の理解が進んでいなかったために差別が残ったことを彷彿とさせる。社会の理解・承認がなければ、差別的取扱いを違法としても、その実効性はない。同性カップルが暮らしやすい社会にするためには、ただ同性婚を認めるだけでなく、その社会的土台を整えることも必要である。

\section{同性婚と憲法}

婚姻についての憲法上の規定は憲法24条にある。

憲法24条は13条の個人の尊重と14条の両性の本質的平等を家庭生活の局面で具体化されなければならないことを示している。\footnote{長谷部恭男編『注釈日本国憲法(2)』(有斐閣、2017年)495頁〔川岸令和〕}

\subsection{憲法13条}

前段は家制度を否定した個人主義の立場から、一人ひとりの人間が人格的自律の存在として最大限尊重されるべきであると示している。後段は包括的基本的人権として幸福追求権を認め、個別の人権規定で例示されていない人権を憲法上で根拠づける際に用いられる。

幸福追求権からどのような具体的権利が導き出され、それが憲法上の権利といえるかどうかは、特定の行為が個人の人格的尊重に必要不可欠であることのほか、その行為を社会が伝統的に個人の自律的決定の委ねられたものと考えているか、その行為は多数の国民が行おうと思えば行うことができるか、行っても他人の基本権を侵害するおそれがないかなど、種々の要素を考慮して慎重に決定しなければならない。\footnote{芦部信喜著 高橋和之補訂『憲法 第七版』(岩波書店、2019年)122.123頁}つまり、人格的自律にとって必要不可欠でなくても、個別の行為ごとに慣習や他者加害原則、比例原則・平等原則などから日常生活における一般的な行為をする自由も、憲法上の人権として保障されうる。

幸福追求権から憲法上の権利として解されるもののなかに自己決定権がある。自己決定権は個人の人格的生存に関わる重要な私的事項を公権力の介入・干渉なしに各自が自律的に決定できる自由を保障する。ここには、子どもを持つかどうかなど家族のあり方を決める自由も含まれる。\footnote{芦部・前掲注17)128頁}実体的憲法観からも、家族制度を社会制度の根幹とするなら、この家族形成権は基本的人権だといえる。\footnote{長谷部・前掲注16)
  100頁}

\subsection{憲法14条}

「法の下の」平等として、法適用の平等だけではなく法内容の平等もその内容として含む。これは憲法と法律を質的に区別して裁判所による法律の違憲審査を認めていること、適用が平等でもその法内容が平等でなければ平等の保障は実現されないことに対応している。

「平等」とは、常に同じ取り扱いをする絶対的平等ではなく同一の事情と条件のもとでは均等に取り扱う相対的平等であり、合理的な理由があれば区別した取り扱いも正当化される。

この異なる取り扱いが違憲かどうかを審査する際には、その区別が関わる権利・義務の内容とその区別の指標を考慮して、審査の厳格度を決定する。

そこで、14条の後段列挙事由は、判例ではこれを理由とする差別的取扱いを憲法が禁止している区別の指標の例示であるとされ、通説では、個別の事情がこれにあたる場合には厳格な審査の基準が適用されるべきなので、それぞれの事情の解釈が重要になる。

後段列挙事由のうち、性別とは社会的・文化的な性ではなく、生物学上の性であり、主に男女差別を是正するための例示である。身体的性差から導かれる差別的取扱いや、性的役割分業観に根ざした差別的取扱いを対象とする。\footnote{辻村みよ子『憲法と家族』(日本加除出版、2016年)118頁}そのため、同性愛者を異性愛者から区別する指標である性的指向は含まれない。

一方、社会的身分とは、人が社会において占めている地位のことである。芦部説によると、人が社会において一時的にではなく占めている地位で、自分の力ではそれらから脱却できず、それについて事実上ある種の社会的評価が伴っているものであるとされる。\footnote{芦部・前掲注17)139,140頁}

判例によると、人が社会において占める継続的な地位とし、平成20年6月4日大法廷判決国籍法違憲判決では嫡出子としての地位を自らの意思や努力によっては変えることのできない父母の身分行為にかかる事柄であり、区別を生じさせることに合理的な理由があるか否かについて慎重に検討する必要があるとした。

青年の家事件判決\footnote{東京高裁1997年9月6日判決の東京都青年の家事件では、同性愛者団体の公共施設の宿泊使用の不承認に正当な理由があるのかが問題となった。異性愛者を前提とした男女別室の原則を同性愛者にもそのまま適用し、「一定の条件を付するなど、より制限的でない方法により、同性愛者の利用権との調整を図ろう」ともせず、「一般的に性的行為に及ぶ可能性があることのみを重視して、同性愛者の宿泊利用を一切拒否した」ことは「不当な差別的取扱い」であると判事された。
  地方自治法244条2項に違反するとされたが、憲法14条違反とは明言されなかった。}では同性愛者にそのまま異性愛者と同様の決まりを適用して、結果的に差別的取り扱いをすることが憲法14条に反するとは明示されなかったが、立法事実の変化も考慮する国籍法違憲判決の姿勢から、同性愛者への理解が進んできている現状も考慮するべきである。同性愛の性向は本人の意思や努力で変えられるものではなく、性向に基づく社会的評価も生じうるために社会的地位だということができる。\footnote{長谷部・前掲注16)190頁〔川岸令和〕}そのため、社会的身分に性的指向を含むことは可能だろう。

また、「差別されない」ということは、政治的には選挙や公務就任において、経済的には財産や労働において、社会的には政治・経済的以外の社会生活において、法的に平等に取り扱われなければならないことを意味する。立法・処分だけではなく、事実的行為においても適用される。\footnote{長谷部・前掲注16)212頁〔川岸令和〕}

よって、社会生活において、性的思考に基づき同性愛者だから異性愛者と平等に扱われないことは、憲法14条違反だといえるはずだ。社会的身分だとしない場合でも、差別的取扱いをする合理的な理由が必要である。

\subsection{憲法24条}

\subsubsection{1項 }

現行の婚姻制度は、かつての明治憲法下の男尊女卑思想に基づく家父長制度、家制度の解体と夫婦間の平等を保障することを目的として1項を定めた。家制度の下では、戸主がその家族の身分上の監督権を有し、婚姻・離婚・住所の選定などに戸主の同意が必要であった。\footnote{長谷部・前掲注16)497頁〔川岸令和〕}家父長の権限ではなく、当事者の合意を要件とすることで家父長の介入を防ぎ、成立した夫婦間での平等を定めることで家庭内での女性の地位を保障した。\footnote{衆議院憲法審査会事務局 『衆憲資94号』(平成29年5月)『新しい人権等に関する資料』76-83頁}

\subsubsection{2項 }

一方、2項では、憲法13条に規定される個人の尊重と14条に規定される法の下の平等からも導かれるように、家族に関する事項を個人の尊厳と両性の本質的平等に立脚して制定することを義務付けている。戦前の民法下では、女性は財産取引において無能力者として扱われていたが、この規定のもと、現行の民法において相続、親権の行使、財産の行使に関して男女差別は禁止されている。

\subsubsection{憲法24条}

最高裁は、平成27年12月16日大法廷判決同氏制違憲訴訟で、憲法24条1項は「婚姻するかどうか、いつ誰と婚姻するかについては、当事者間の自由かつ平等な意思決定に委ねられるべきであるという趣旨を明らかにするもの」であり、同条2項は「具体的な制度の構築を第一次的には国会の立法裁量に委ねるとともに、その立法に当たっては、同条1項も前提としつつ、個人の尊厳と両性の本質的平等に立脚すべきであるという要請、指針を示すことによって、その裁量の限界を画したもの」であるとしている。

本条文が特に婚姻の規定から始められていることは、家族の構成が夫婦関係を基礎とすることを示している。\footnote{長谷部・前掲注16)498頁〔川岸令和〕}文言を重視した解釈をすると、憲法24条は同性婚を容認していないことになり、その合憲性は憲法13条の幸福追求権の解釈に家族を形成する権利が含まれるかに委ねられる。一方、制定経緯を重視した解釈をすると、「両性の合意」の強調は家制度の否定のためであり、当時想定されていなかった同性婚についてことさらに排除する趣旨ではないということもできる。\footnote{長谷部・前掲注16)509頁〔川岸令和〕\\
また、2019年9月18日の宇都宮地裁判決において、同性カップルに事実婚制度の適用を認めた際にも、「憲法制定当時は同性婚が想定されていなかったからにすぎず,およそ同性婚を否定する趣旨とまでは解されないから,前記のとおり解することが憲法に反するとも認められない。」として、法的保護の必要性を認めた。}

憲法24条の想定する家族像が、異性愛者である夫婦とその嫡出子からなる近代家族に限定されるのだろうか。憲法上の権利の侵害として、憲法が介入できるのは、婚姻制度が特定の範疇に属する者または集団を正当な理由なく排除しているときであり、憲法が何らかの婚姻像を想定しているとするには何らかの根拠が求められる。\footnote{白水隆『平等権解釈の新展開 同性婚の保障と間接差別の是正に向けて』(三省堂、2020年)103頁}

\subsection{同性婚をめぐる憲法上の論点}

同性カップルに対して、婚姻の自由を認めず、同性婚を認めないことは、同性愛者の自己決定権を侵害しないかを13条との関係で考える。

侵害しないとする見解には、法律婚は法的権利義務が伴うため、婚姻制度への参加は自己決定できるものではない。また、婚姻することを個人の自律・尊厳に必要不可欠なものであるとすると、していない人は自律していないかのようになってしまうので、婚姻の意義を個人の自律性から説くほどに、多様性が失われ、13条の趣旨に反する可能性がある\footnote{白水・前掲注29)99,100頁}というものがある。

一方、侵害するという見解には、法律上の婚姻は、純粋な幸福と結びつくものであり、法律上承認された婚姻関係から得られる相続や接見あるいは扶養や子どもの養育等の権利や義務は、家族として社会生活を送る上で必要不可欠な権利・義務であるから、自己決定権の侵害であると言える\footnote{南和行『同性婚 私たち弁護士夫夫です』(祥伝社、2015年)171,172頁}ものがある。さらに、人が一人ひとり固有のアイデンティティを確立し、自分にとっての幸福を追求することは、基本的人権を保障する日本国憲法の根幹をなすものである。そのため、同性愛者への社会的な偏見や差別が強く、それによってライフスタイルを選択できないことは権利侵害にあたる\footnote{二宮・前掲注7)137頁}というものもある。

私は、婚姻は人格的自律に不可欠でないと思うが、婚姻制度によってしか保障されない生活上の権利があり、それが重大なものであるといえる以上、家族形成権は自己決定権に含まれる憲法上の権利といえることから、同性婚を認めないことは自己決定権を侵害しているといえると考える。同性婚は異性婚と両立するものであり、婚姻をしない者とした者とに優劣関係を生じさせるものでもないからだ。

憲法24条1項は婚姻の自由の根拠規定であるが、これは同性愛者には保障されないのかを、24条との関係で考える。

再婚禁止期間一部違憲判決では、立法裁量は個人の尊厳と両性の本質的平等に制限されることから、合理的な期間を超える部分について憲法14条1項、24条2項に違反するとされた。また、この判決では、24条1項の婚姻の自由は、その趣旨に照らし、十分尊重に値するものだとされた。これにより、婚姻の自由は立法裁量を統制するにあたり、考慮すべき利益として尊重されるものだが、権利ではないと理解できる。

同氏制違憲判決では、24条1項の婚姻の自由が実質的に制約されているので、24条2項を踏まえて立法裁量の逸脱がないか検討され、家族・婚姻関係については総合的な判断によって定められるものだと示された。

婚姻の法制度において同性カップルをその対象としないことに、同性カップルの婚姻の自由の尊重を超える事由がない限り、婚姻の自由は保障されるべきである。

また、婚姻の意義・目的に照らして同性愛者をその制度から排除する理由はあるのか、各侵害権利や利益を付与しない根拠はあるのかを憲法14条との関係・差別的取扱いの合理性から考える。

14条の社会的地位に性的指向も含むとすると厳格な審査が行われるが、当初想定されていなかったからといって同性愛者に婚姻を認めないことは戸籍制度や社会秩序を維持するために必要不可欠だとは言えない。また、婚姻制度を認めないこと以外に、パートナーシップ制度を導入するなど代替手段が存在することは他国の状況からわかることである。同性婚と異性婚は両立可能な制度であるため、特定の性自認・性的指向を理由として、婚姻制度の独占もしくは排除があっていいはずがない。

また、共同生活契約の公証制度がないこと、共同親子関係の形成資格がないことは自己決定権に家族形成権が含まれるとする通説から、憲法上の権利侵害になりうる。共同生活契約の公証制度がないことは特定のパートナーと共同生活するのは同じであり、異性でなくとも生殖前提ではないため、合理的な理由はない。また、共同親子関係の形成資格がないことは異性カップルなら子の福祉のための嫡出推定や特別養子縁組に適合し、同性カップルなら不適合だといえず、いずれにも適不適があるはずである。\footnote{白水・前掲注29)101頁}

よって、差別されない権利としての平等権のもと、同性婚は認められるべきだ。

以上より、婚姻の自由は同性愛者にも認められるべきであり、同性婚を認めないことは、憲法13条・14条に反することだといえる。

そのため、憲法24条2項での政治的な裁量上の合理的な理由がない限り、憲法24条上も違憲であるといえる。合理的な理由として挙げられうるのが、同性カップルへの社会的承認が進んでいないため同性婚導入によって混乱が生じる社会情勢だ。

\section{同性パートナーシップ制度}

ヨーロッパで、婚姻が認められていなかった同性カップルのために、婚姻より保障の内容が減じられた、婚姻によく似た制度としてパートナーシップ制度がつくられ、異性カップルにも拡充されている。\footnote{石田・前掲注4)107,108頁}

\subsection{日本での同性パートナーシップ制度}

日本では、地方自治体において性的マイノリティとつながりのある区議や自らが性的マイノリティである区議など、性的マイノリティの人権問題について問題意識のある政治家が発起人となって各地で導入が進んでいる。

渋谷区では、「渋谷区男女平等及び多様性を尊重する社会を推進する条例」に基づき、任意後見制度と公正証書という手続き的負担を課した上で、パートナーシップの存続期間や財産など二人の関係性について契約で定め、それを区が証明書という形で認める。

世田谷区パートナーシップの宣誓の取り扱いに関する要項では、市長権限で定められる要項に基づき、世田谷区在住であることと成人していることを運転免許証などで示したのち、二人の関係性を区長から認められる
。

規範の形式は主に条例と要綱\footnote{条例は議会の議決によって制定されるが、要綱は行政の内部規範であり、市長の権限で定めることができる。(棚村政行・中川重徳 『同性パートナーシップ制度 世界の動向・日本の自治体における導入の実際と展望』(日本加除出版、2018年)150頁-)}があり、認定の手段は証明と宣誓、登録\footnote{証明は当事者の関係性について公正証書による確認がなされるが、宣誓は確認不要で当事者の意思表示によってなされる。登録は当事者に対して自治体が登録証明書を発行するなど当事者の関係性に継続的に関わる。(棚村政行・中川重徳 『同性パートナーシップ制度 世界の動向・日本の自治体における導入の実際と展望』(日本加除出版、2018年)150頁-、登録制度を導入している那覇市のパートナーシップ登録の取り扱い要領\href{https://www.city.naha.okinawa.jp/kurasitetuduki/collabo/dannjyosankaku/center/jigyou/partnershipregistrat.files/partnershipyoukou.pdf}{{https://www.city.naha.okinawa.jp/kurasitetuduki/collabo/dannjyosankaku/center/jigyou/partnershipregistrat.files/partnershipyoukou.pdf}} )}があるが、いずれにも相続や配偶者控除などの直接的な効力はない。\footnote{藤戸敬貴「同性カップルの法的保護をめぐる国内外の動向―2013年8月~2017年12月、同性婚を中心に~」『レファレンス』805号,国立国会図書館 調査及び立法考査局編,(2018年)84,85頁}

\subsection{憲法との関係}

同性パートナーシップ制度は同性婚を認めるわけではないため異性婚を前提とする現行の婚姻制度の根拠となっている憲法24条に抵触しない。また、同性カップルの永続的な関係を公的に証明するだけであり、直接的・具体的な権利や義務を認めるものではない。ただ憲法13条の自己決定権に含まれる家族形成権という憲法上保障される権利を具体化するものであり、制度の存在によって憲法14条の保障する平等をかなえようとするものである。

法律婚ではなく法的な拘束力もないことから、異性カップルとの平等の保障をする制度とはいえないが、公権力が現行の婚姻制度では保護されない人権を保障する必要性を認めていること自体に大きな意義がある。

\subsection{同性婚と同性パートナーシップ制度の違い}

同性カップルの生活保障類型には3つの段階がある。\UTF{2460}事実婚としての法的保障を同性カップルにも適用すること、\UTF{2461}婚姻とは別の生活パートナーとしての登録・生活のための契約の登録を認め、婚姻に近似した法的権利を保障すること、\UTF{2462}同性婚を認めることの3つだ。\UTF{2461}(→\UTF{2460})→\UTF{2462}の流れが諸外国では一般的である。\footnote{二宮・前掲注7)126頁}

現在の日本では異性カップルへの婚姻制度と一部の地方自治体では同性カップルへの同性パートナーシップ制度がある。異性カップルと同性カップルが共通して利用できる制度はない。また、日本の自治体で導入されているパートナーシップ制度は2人の関係性を登録するだけでその他の法的な拘束力はないことから、同性カップルの生活保障類型における\UTF{2461}よりも生活保障の度合いは低い。婚姻の効果と比較すると、法的経済的利益と心理的社会的利益があったが、日本の同性パートナーシップ制度には後者の利益しかない。

\section{日本での同性婚導入に向けて}

\subsection{同性婚導入国との比較}

同性婚導入国では、選挙において同性婚を認めるかどうかが政治的スタンスの分かれ目となり、多数派が認める立場であったこと、平等原則から合理的理由のない差別的取扱いが絶対的に禁止されていること、その国に共存共栄を目指す社会的アイデンティティがあること、パートナーシップ制度の導入から社会的承認が進んでいたこと、宗教婚と民事婚との区別をすることで宗教上の問題と分離したことが、導入を進める要因となった。\footnote{二宮・前掲注7)126頁}

それに対して、日本は文化・生活において宗教色が薄く、同性婚を承認することによる宗教的な問題はないが、島国であり、他国に比べて人種や民族の多様性が問題になることは歴史的にも少なく、マイノリティの存在に気づきにくい社会である。近年では人権意識の高まりや性的マイノリティの人権を主張する団体の活動の活発化から次第に注目されるようになってきている。しかし、経済政策が主な選挙や国会での論点であり、同性婚を認めるか否かが主要な論点となることは少ない。

\subsection{登録拡大と家族の在り方の多様性}

一方で、同性パートナーシップの登録制度は着実に日本全体に広まってきている。性的マイノリティが働きやすい環境作りを行うNPO法人である虹色ダイバーシティの調査によると、2020年6/30時点で、人口26.4%をカバーする51自治体で同性パートナーシップの登録制度が導入され、1052組の同性カップルが登録している。\footnote{\href{https://nijiirodiversity.jp/partner20200630/}{{https://nijiirodiversity.jp/partner20200630/}} 認定NPO法人虹色ダイバーシティホームページ}

また、近年の同性パートナーシップ登録制度のなかには適用対象を同性カップルだけでなく、性的マイノリティ全体や異性カップルをも含むすべてのカップルへ拡充する自治体もある\footnote{藤戸敬貴「性の在り方と法制度―同性婚、性別変更、第三の性―」『レファレンス』819号,国立国会図書館 調査及び立法考査局編,(2019年)45-52頁}。もし、今の日本で異性婚と同様に同性婚を認めると、確かに同性愛者の地位は高まるが、社会における家族の形は婚姻ベースのままになる。一方、パートナーシップ制度では拘束力の面で婚姻には劣るが、社会が多様な家族の在り方を認めるきっかけになりうる。そのため、家族の在り方に多様性を求める観点からも適用対象の拡充は推奨される。

\subsection{今後の展望}

婚姻制度が異性婚と地方自治体による法的拘束力のない同性パートナーシップ制度である現状から、国による法的拘束力のあるパートナーシップ制度の導入、最後には同性婚の導入へと進み、性的指向にかかわらず、婚姻とパートナーシップ制度のいずれかを選択し、その生活を望む形で保障されるようになることが今後の日本の展開として理想的だと思う。

このように、法的な拘束力はないパートナーシップ制度でも、行政という公権力が性的マイノリティの存在と生活、偏見にさらされている実情を認め、その打破には当事者の努力や行動だけではなく、社会の意識が変わる必要があることを示している。\footnote{南・前掲注31)173頁}

また、同性カップルの存在や同性カップルが婚姻制度を求めることに対して、社会的承認が得られることによって、婚姻制度において同性カップルの生活を保障しないことの政治的な事情がなくなる。これにより、同性婚を設けないことに合理的な理由は全くなくなり、憲法上の権利に基づいて同性婚を求める者の存在を認識しながらその保障に必要な制度を作らなかったことへの立法不作為が問題となる。現在行われている同性婚を求める裁判でも、社会的承認が得られているといえるかどうかという立法事実の変化が認められるかが問題だと思う。

このままパートナーシップ登録制度が日本全国に広まり、ゆくゆくは当たり前のものになり、同性カップルへの社会的承認が進んでいけば、国民や政治家の問題意識から国会での議論もより活発に行われ、法的な生活保障を通じた人権保障につながるはずだ。

\section{参考文献}

・芦部信幸著 高橋和之補訂『憲法 第七版』(岩波書店、2019年)\\
・井上眞理子編『家族社会学を学ぶ人のために』(世界思想社、2010年)\\
・横藤田誠・中坂恵美子著『人権入門(第3版)』(法律文化社、2017年)\\
・窪田充見『家族法(第4版)』(有斐閣、2019年)\\
・現代憲法教育研究会『憲法とそれぞれの人権』(法律文化社、2017年)\\
・三成美保編『世界人権問題叢書94 同性愛をめぐる歴史と法 尊厳としてのセクシュアリティ』(明石書店、2015年)\\
・新・アジア家族法三国会議編『同性婚や同性パートナーシップ制度の可能性と課題』(日・本加除出版、2017年)\\
・新井誠・曽我部真裕・佐々木くみ・横大道聡『憲法\ajRoman{2} 人権』(日本評論社、2018年)\\
・石田仁『はじめて学ぶLGBT 基礎からトレンドまで』(ナツメ社、2019年)\\
・大屋雄裕「正義・同一性・差異」『法学教室』423号(有斐閣、2015年)\\
・大林啓吾・見平典『憲法用語の源泉を読む』(三省堂、2016年)\\
・大林啓吾「同性婚にピリオド?-アメリカの同性婚禁止違憲判決を読む」『法学教室』423号(有斐閣、2015年)\\
・棚村政行・中川重徳 『同性パートナーシップ制度 世界の動向・日本の自治体における導入の実際と展望』(日本加除出版、2018年)\\
・中川淳・小川富之編『家族法』(法律文化社、2013年)\\
・長谷部恭男、石川健治、宍戸常寿(編)『憲法判例百選 II(第 7
版)』(有斐閣、2019 年)\\
・長谷部恭男編『注釈日本国憲法(2)』(有斐閣、2017年)\\
・辻村みよ子『概説ジェンダーと法(第2版)』(信山社、2016年)\\
・棟居快行・松井茂記・赤坂正浩・笹田栄司・常本照樹『基本的人権の事件簿 :
憲法の世界へ』(有斐閣、2019年)\\
・藤戸敬貴「性の在り方と法制度―同性婚、性別変更、第三の性―」『レファレンス』819号,国立国会図書館 調査及び立法考査局編(2019年)45-52頁\\
・藤戸敬貴「同性カップルの法的保護をめぐる国内外の動向―2013年8月~2017年12月、同性婚を中心に~」『レファレンス』805号,国立国会図書館 調査及び立法考査局編(2018年)83-92頁\\
・同性婚人権救済弁護団編『同性婚 だれもが自由に結婚する権利』(明石書店,2016年)\\
・南和之『同性婚 私たち弁護士夫夫です』(祥伝社、2015年)\\
・二宮周平『18歳から考える家族と法』(法律文化社、2018年)\\
・白水隆『平等権解釈の新展開 同性婚の保障と間接差別の是正に向けて』(三省堂、2020年)\\
・福島みずほ 『みんなの憲法二十四条』(明石書店、2005年)\\
\end{document}