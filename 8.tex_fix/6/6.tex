\documentclass[twoside]{jsarticle}
\usepackage[dvipdfmx,hidelinks]{hyperref}
\usepackage{pxjahyper}
\usepackage[jis2004]{otf}
\usepackage[dvipdfmx]{graphicx}
\usepackage{url}
\setcounter{page}{1}
\usepackage{fancyhdr}
\begin{document}
\title{メディア環境の変化から考える公共放送の役割と制度}
\author{紺田雄平}
\date{}
\maketitle
\tableofcontents
\clearpage
\pagestyle{fancy}
\lhead[メディア環境の変化から考える公共放送の役割と制度]{}
\rhead[]{\leftmark}
\section{序}

日本におけるテレビ放送では、日本放送協会(以下NHK)と民放の二元体制が採用されてきた。しかし近年、AbemaTVなどのインターネットテレビ局やYouTubeなどの動画配信サイト、ビデオオンデマンド事業者(以下、インターネット情報提供事業者という)の台頭によって、従来型のテレビ放送をめぐる環境は目まぐるしく変化している。

さらに、NHKのスクランブル放送化を単一の政策とする「NHKから国民を守る党」が国会で議席を獲得したことから、国民の公共放送の捉え方の変化が表面化していると言える。

このような状況下で、公共放送としてのNHKが果たすべき役割とは何なのか。また、NHKがその役割を果たすためにはどのような制度を構築する必要があるのか。判例や学説の放送の自由論をもとに、憲法の観点から
公共放送の存在意義を明らかにする。

\section{日本の放送の現状}

 

\subsection{NHKと民放の二元体制}

2019年に改正された放送法では、放送とは「公衆によって直接受信されることを目的とする電気通信の送信」(放送法2条1号)と定義され、有線によるケーブルテレビ等も放送の定義に含められる一方、インターネット番組の多くは放送には当てはまらないとされている\footnote{新井誠=曽我部真裕=佐々木くみ=横大道聡『憲法\ajRoman{2} 人権』(日本評論社、2017年)151頁〔曽我部真裕執筆〕。}。この放送概念は「基幹放送」と「一般放送」に分けられ、基幹放送の中核をなすのが、受信料を財源とするNHKと、広告料を財源とする民放の二元体制である。

放送における二元体制の成立の過程を見ると、以下のような経緯を辿っている。戦前に政府による統制下にあった社団法人日本放送協会が、戦後の民主化により特殊法人化され、放送事業への民間の参入機会が保障されるようになった。その中で、民間放送事業は急速な発展を遂げ、まさに「二元的」な放送秩序が生まれた。そして現在では、衛星放送、ケーブルテレビなどの技術開発などにより、有料専門番組やビデオオンデマンドなどの双方向性サービスを民間事業者が提供しているなどの進展も見られる\footnote{西土彰一郎『放送の自由の基層』(信山社、2011年)19頁。}。

\subsection{NHK}

放送法によると、NHKの設立目的は、「公共の福祉のために、あまねく日本全国において受信できるように豊かで、かつ、良い放送番組による国内基幹放送を行うとともに、放送及びその受信の進歩発達に必要な業務を行い、合わせて国際放送及び協会国際衛星放送を行うこと」(放送法15条)と定められており、放送の全国普及義務を法的に負っている点にNHKと民放との違いが認められる\footnote{鈴木秀美=山田健太編『放送制度概論』(商事法務、2017年)153頁〔鈴木秀美執筆〕。}。また、NHKは放送法4条1項の番組編集準則に加えて、\UTF{2460}豊かで、かつ、良い放送番組の放送を行うことによって、公衆の要望を満たすとともに文化水準の向上に寄与するように、最大の努力を払うことや、\UTF{2461}全国向けの放送番組のほか、地方向けの放送番組を放送することに加えて、\UTF{2462}日本の過去の優れた文化の保存、新たな文化の育成・普及に役立つことを求められている(同法81条1項)。

また、NHKは受信料を主な財源としており、広告放送を禁止されている(同法83条1項)。これは、私的なイニシアチブによっては十分に供給されない多様で質の良い番組の供給を実現するために、財源を国民に幅広く求める受信料制度が、最も適しているために採用されているとされる\footnote{長谷部恭男「公共放送の役割と財源」舟田正之=長谷部恭男編『放送制度の現代的展開』(有斐閣、2001年)201頁。}。

\section{公共放送の意義に関する判例}

\subsection{概要}

NHKの存在意義について重要な示唆を与えるのが、受信料負担の仕組みを定めた放送法64条1項についての判例である、最高裁平成29年12月6日大法廷判決\footnote{最大判平成29年12月6日民集71巻10号1817頁。}である。

この事案は、NHKが、NHKの放送を受信することができる放送設備を設置していながら受信契約を結んでいなかった被告に対し、受信料の支払いを求めた事案であり、放送法64条1項が、受信設備の設置者に対し受信契約の締結を強制する規定であるかなどが争点となった。判決は、放送法64条1項は受信契約の締結を強制する規定であると示し、憲法13条、21条、29条に反しないとして、上告を棄却した。

\subsection{公共放送の意義についての判示}

この判決では、放送法64条1項が定める規定の意義についての判示で、「放送は、憲法21条が規定する表現の自由の保障の下で、国民の知る権利を実質的に充足し、健全な民主主義の発達に寄与するものとして、国民に広く普及されるべきもの」と示している。この点を反映して放送法が制定されていると考えられるため、この判示は憲法上の放送の意義が示されたと捉えられ\footnote{曽我部真裕「受信料制度の合憲性」別冊ジュリストメディア判例百選(第二版、2018年)201頁。}、重要であると評価できる。

また、同判決では、「放送法は、\ldots{}\ldots{}公共放送事業者と民間放送事業者とが、各々その長所を発揮するとともに、互いに他を啓もうし、各々その欠点を補い、放送により国民が十分福祉を享受することができるように図るべく、二本立て体制を採ることとしたものである。」と、NHKと民法の二本立て体制を採用する意義についても判示しており、放送法制定時の政府の説明に沿って、全国放送の使命を持つ公共放送、民放における番組製作者個人の創意工夫に二元体制の意義を見出している。

しかし、民放ネットワークもほとんど全国で放送されていること、NHKの番組制作にかける個人の創意・工夫が民放に比べて劣るともいえないという批判があり\footnote{長谷部恭男『テレビの憲法理論』(弘文堂、1992年)150頁。}、二元体制を採用する意義は「基本的情報」説とそれを前提とした部分規制論によって説明されるべきである。その上で、メディア環境が変化している現代においても、なぜ公共放送が国民の知る権利に資するといえるのか、公共放送が機能するためにどのような規制をかけるべきかについては、さらなる検討が必要となる。

\section{放送の自由論}

\subsection{検討の視座}

平成29年判決で参考にされたとされる\footnote{西土彰一郎「メディアと憲法公共放送の財源------NHK受信料訴訟大法廷判決をうけて」論究ジュリスト25号(2018年)39頁。}のが、長谷部恭男が提唱する、マスメディアの自由論としての「基本的情報」説である。ここでは、「基本的情報」説、そして「基本的情報」説をもとに、放送に対する規制を正当化する部分規制論を見ることで、放送の二元体制がなぜ国民の知る権利に資すると言えるのかを検討する。

\subsection{従来の議論}

部分規制論は、放送に対して、新聞など他のメディアにはない特別の規制を課す根拠として提示された理論であるから、規制根拠についての伝統的な学説の検討を行う。

従来の議論では、有限希少説と社会的影響力説が有力であった。有限希少説とは、放送用の電波は有限であるため、放送に利用できるチャンネル数には限界があるので、混信を防止しつつ希少な電波を有効に利用するためには、免許制を採ることや一定の規律を課すことが必要であるという説\footnote{芦部信喜「放送の自由」法学教室180号(1995年)74頁。}である。また、社会的影響力説とは、放送は直接お茶の間に侵入し、動画や音声を伴う生の映像を通して視聴される点で、受け手に大きな影響を与えるために、公的規制が必要であるという説\footnote{芦部・前掲注(9)75頁。}である。

しかし、有限希少説や社会的影響力説が当初は説得力を持ちえたとしても、ネット時代の視点からは、多チャンネル化に伴い、電波以外を用いて放送を行う道が開けていることから、電波の希少性が減少しているのではないか、また、そもそも放送の持つ影響力の大きさを論証することができないことや、仮に特別な影響力を持つとしても、動画や音声を用いたコンテンツは他にもあるため、影響力が大きいことを理由に放送にのみ規制をかけることは合理的でない、といった批判が向けられる。

これらの説は、新聞などの他の媒体も含めたメディア全体の中で、放送にのみ規制を課す根拠を探る議論ではない点に難点がある。

\subsection{マスメディアの自由論と部分規制論}

長谷部説によれば、表現の自由には、個人の人権として保障される側面と社会全体の利益を促進するがゆえに保障される側面があるという。マスメディアに表現の自由が保障されるのは、大量の情報の中から社会生活を送る上で誰もが必要とする基本的情報を社会全体に提供し、民主的政治過程を支えるとともに、寛容な社会を再生産する機能を果たすからであり、そうであるからこそ、マスメディアには、情報源へアクセスする便宜や取材源の秘匿が認められ、逆に、個人には認められない制約を加える余地が生じるとされる\footnote{長谷部恭男『憲法学のフロンティア』(岩波書店、2013年)169頁。}。

以上で見たように、マスメディアの自由は「公共の福祉に基づく権利」であるために、基本的情報の社会全体への公平な提供を実現するための政策的規制が課され\footnote{長谷部・前掲注(11)171頁。}、この規制の正当化根拠として部分規制論が出てくる。すなわち、マスメディアが特権的な立場を有することにより、情報のボトルネックが生じ、この危険に対処するために放送だけに対する特別の規制が課されると説明される。

ここでは、放送に対する規制により、社会の中の多様な意見が番組内容に反映される一方で、規制を受けないメディアは放送に対する政府の規制の行き過ぎをチェックし、自由なメディアとしての姿を示す、という相互のやりとりによって、マスメディア全体として、基本的情報の社会全体への公平な提供を行うということが期待される。ここから、放送内部において規制の異なる公共放送と民間放送を併存させることも、基本的情報の平等な提供という、全体としての放送制度を実現することから正当化される\footnote{長谷部・前掲注(7)153頁。}。

\subsection{二つのシナリオ  }

ここまで、「基本的情報」説、部分規制論によって公共放送と民放の二元体制の果たす役割を明らかにしてきたが、デジタル化に伴う多チャンネル化が進んだ場合にも、公共放送はこれまでと変わらない役割を果たすことができるのであろうか。

この点について、長谷部は、「暗いシナリオ」、「もう一つのシナリオ」という、二つのシナリオを提示している。

「暗いシナリオ」で提示される状況では、多チャンネル化が視聴の分散を招き、「狭い範囲の趣向とものの見方を幅の狭い文化的な触れ合いの中で再生産するバルカン化現象」が生じると示されている。また、多数の事業者が新規参入することで、自主規制というカルテルの実効性が低下し、民間放送が公共放送を手本とすることがなくなる、あるいは、公共放送自体が放送市場への影響力を持ちうるほどの視聴率を維持できなくなるという現象が生じ、公共放送が事実上その役目を終えると指摘する\footnote{長谷部・前掲注(4)205頁。}。

「もう一つのシナリオ」は、放送事業における規模の利益、範囲の利益はデジタル時代の下でも継続するため、事業者の統合が進むとの予測を立てる。このシナリオの下では、ある程度の視聴の分散化が生じたとしても、規模の利益、範囲の利益の活用を通じて、メディアはこの状況に対応することが可能になる。メディアが融合し、巨大化する中でも、番組の多様性、品質の良さを保つために、放送市場への影響力を有する公共放送の併存が必要となる\footnote{長谷部・前掲注(4)208頁。}。

\section{放送の現状分析}

\subsection{分析の視座}

現在の日本において、放送を取り巻く状況はこれら二つのシナリオのうちどちらにより近いものであろうか、あるいは全く異なる道を進むものであろうか。この点について、二つのシナリオと現在の状況との異同を考察する。

\subsection{「もう一つのシナリオ」との比較}

「もう一つのシナリオ」の下で、放送事業における規模の利益、範囲の利益が持ち出される場合に、公共放送、民間放送に対抗する新規事業者として想定されていたのは、ハードへの巨額の設備投資が必要となる放送事業者であった。さらに、ハード面への設備投資が少なくて済むインターネットを通じた情報提供事業であっても、品質が高く、多くの視聴者をひきつけるソフトを製作するためには巨額の投資が必要となると想定され、既存の放送事業者の規模の利益は維持されるとの前提があった。

しかし、現在の状況下では、コンテンツ制作に巨額の投資を必要としない、YouTubeなどの動画配信サイトや、ソフト面、すなわちコンテンツへ巨額の投資をするビデオオンデマンド事業者が、公共放送や民間放送に対抗する新規事業者となっている。例えば、ビデオオンデマンド事業者の代表格であるネットフリックスが、2019年にコンテンツに投じた費用は150億ドルであり、NHKの年間制作費の5倍である\footnote{篠原英樹「タブーは鉱脈
  NETFLIXのものづくり」『日本経済新聞』、2020年1月5日、電子版\\(\href{https://www.nikkei.com/article/DGXMZO53764840U9A221C1000000}{{https://www.nikkei.com/article/DGXMZO53764840U9A221C1000000}}
  閲覧日:2020年12月26日)}。

以上のように、ハード面への設備投資が比較的少額であり、ソフト面への巨額の投資を行うインターネットを通じた情報提供事業者の台頭によって、もはや既存の放送事業者の規模の利益は維持されえない、あるいは、消滅しつつあると言える。

\subsection{「暗いシナリオ」との比較}

では、現在の状況は「暗いシナリオ」に沿って進んでおり、公共放送の役割は終焉に向かっているのか。

確かに、インターネット情報提供事業も含めた多チャンネル化が進み、既存の放送事業の視聴時間は減少している。NHK放送文化研究所の「全国個人視聴率調査」(2019年11月)\footnote{中山準之助=伊藤文=保高隆之=内堀諒太「テレビ・ラジオ視聴の現況」放送研究と調査2020年3月号(NHK出版、2020年)90頁。}によると、2019年の日本人のテレビ視聴時間は、1日あたり3時間31分と、2018年の3時間43分という数字から減少しており、近年横ばいで維持されてきた状況に変化の兆しが見られる。また、2011年の研究によると、高年層でテレビの視聴時間が増加する一方、若年層では視聴時間が減少していることが指摘されている\footnote{関根智江「年層による差がさらに広がるテレビ視聴」放送研究と調査2011年12月号(2012年)27頁。}。

また、誰しもが情報発信の主体となることができるようになり、偏った情報や極端な情報が流れやすくなり、バルカン化現象が生じつつあるという評価も可能かもしれない。

\subsection{第3のシナリオ}

しかし、誰しもが情報発信の主体になることができるようになったからこそ、情報発信の主体に対する「信頼」が重要となっていると言えるため、インターネット情報提供事業の拡大による視聴の分散の影響を、放送事業者がどの程度受けるのかということについては、NHKと民間放送とに分けて、それぞれの視聴者からの信頼の程度の差に目を向けて考える必要がある。

2019年に新聞通信調査会が行なった、メディアに関する全国世論調査によると、メディアの信頼度では、NHKテレビが新聞と僅差の数値で2位になっているのに対し、民放テレビは少し離されて3位となっている\footnote{公益財団法人新聞通信調査会「メディアに関する全国世論調査」(2019年)。}。

また、民放テレビとインターネット情報提供事業との間には、広告料、あるいは視聴料といった市場原理に左右される財源を元にコンテンツを製作するという収益構造の点で、類似性が認められる。

民放とNHKの信頼度の違い、財源の違いを考慮すると、インターネット情報提供事業の拡大による視聴の分散の影響を大きく受けるのは、民放であって、NHKに関しては、なお相当の視聴率を維持できるという予測を立てることができる。

よって、長谷部自身が「NHKがなお相当の視聴率が確保できるという想定が成り立つなら」という仮定の下で言及している、「視聴者のバルカン現象が生じた時にこそ、公共放送の役割が重要性を増すという考え方」\footnote{長谷部・前掲注(4)207頁。}が説得的である。

\section{ネット時代における公共放送の役割}

\subsection{二元体制からの転換}

以上のように、ネット時代における視聴者のバルカン現象が生じつつある状況において、公共放送の役割は重要性を増すといえるが、重要性を増した公共放送の役割はどのようなものであろうか。

前述のように、ネット時代においては民間放送とネットワーク情報提供事業が同質性を持つようになるので、公共放送と民間放送の二元体制と、それ以外のメディアという従来の区分ではなく、公共放送とそれ以外のメディアという区分で考える必要がある。

\subsection{ジャーナリズムの同志に向けた公共放送の役割}

この点に関して参考になるのが、西土彰一郎が唱える「ネットワークの結節点としての公共放送」\footnote{西土・前掲注(8)43頁。}である。西土は、ソーシャルメディアの言論領域を、公的なものと私的なものが不分明な「ネットワーク公共圏」と定義し、このネットワークの中での結びつきは、社会的な出来事に対して特定の目的を持って「群がる形」での共有体験であるとする。そのため、総合編成番組により、視聴者の選好と一定の距離を置き、社会全体で共通の認識となる出来事を伝え、連帯感を生み出すという二元体制の構想は瓦解すると主張する。その上で、公共放送は、NPO報道機関との合同調査を行うなどして、ジャーナリズム全体を深化・拡充させる役割、ジャーナリズム・ネットワークの構築を促す役割を担うべきであると提言している。

西土の提言は、ジャーナリズムを担う同志に向けた公共放送の役割を示す点で重要な意義を有する。

\subsection{一般市民に向けた公共放送の役割}

前述のマスメディアの自由と部分規制論の段落で述べたように、公共放送は、一般市民の知る権利に資するものとして、規制メディアとして、民間放送との二元体制の一角を担うものとしての役割を果たしてきた。

この役割については、たとえ民間放送との二元体制が成立しなくなった場合においても継続するものである。すなわち、ネット社会においても、公共放送は、規制を受けないメディアとの相互作用のもとで、規制によって社会の中の多様な意見を番組内容に反映し、社会生活を送る上で誰もが必要とする基本的情報を社会前提に提供することで、様々な政策の選択肢を示し、より良い政治のあり方についての議論を巻き起こしたり、異なる世界観との共存の必要性を自覚させたりするという役割を果たす。

その大前提として、視聴率が一定程度存在すること、すなわち、公共放送への信頼が重要であることは前述した通りである。公共放送への信頼が落ちてしまうと、公共放送が持つ模範的役割や、基本的情報の提供機能が作用しなくなってしまうからである。

\section{放送制度の検討}

\subsection{信頼の維持のために}

国民の知る権利に資する役割を果たす者としての公共放送が、その役割を十分に果たすためには、ネット時代においてどのような放送制度の構築が必要であろうか。

前述のように、公共放送がその役割を果たすための大前提となっているのが、一般市民の公共放送への信頼である。

ここで、公共放送への信頼を維持するための手段としては、経営体制、受信料制度に関する制度構築など、様々な方策が考えられうるが、ここでは、公共放送の役割が、規制によって社会の中の多様な意見を番組内容に反映させることに求められていることから、番組編集準則(放送法4条1項)に着目して、公共放送が国民の知る権利に資するための制度についての考察を巡らす。

\subsection{番組編集準則}

放送法では、放送事業者に対して、\UTF{2460}公安及び善良な風俗を害しないこと、\UTF{2461}政治的に公平であること、\UTF{2462}報道は事実を曲げないですること、\UTF{2463}意見が対立している問題については、できるだけ多くの角度から論点を明らかにすること、という番組編集にあたっての準則が設けられている(放送法4条1項)。この準則については、法的拘束力を持たない倫理的規定と解する場合には合憲であるという、一種の合憲限定解釈をとる立場が学説上多数であり、総務大臣が個々の番組を番組編集準則によって審査する場合には、番組編集準則を違憲とみる見解が多数を占めている\footnote{鈴木秀美ほか編・前掲注(3)放送制度概論 90頁。}。また、通説は、これら4つの原則は個々の番組ごとに判断されるべきものではなく、その放送事業者が放送する番組全体で判断されるべきものとしている\footnote{長谷部・前掲注(4)19頁。}。この番組編集準則については、表現の自由に対する内容規制に当たるが、先述した部分規制論によって正当化されるとする見解が有力である\footnote{新井誠ほか・前掲注(1)150頁。}。

このうち、政治的公平性と多角的解明の準則に対しては、ある放送局がある見解を主張しても、他の放送局が異なった見解を主張することで、放送全体として多様な見解が伝えられるため、個々の放送局に公正さを要求する必要はない、あるいは、放送に公正さが必要であるとしても、政府による公正さの要求は、政権に批判的な放送が不公正とされる危険性があるため、市場原理に任せる方が適切であるといった批判が松井茂記からなされている\footnote{松井茂記「放送の自由と放送の公正」法律時報67巻8号(1995年)14頁。}。

しかし、ネット時代において民間放送の地位が下がりつつあること、バルカン化が進むことを考えると、むしろ少なくとも一つの放送局に対して公正さを要求しないと、視聴者としてはどの放送局を視聴すれば基本的情報を得ることができるのかが判然としないため、一般市民の知る権利に資するとは言えない。その一方、政府による公正さの要求が持つ危うさの指摘は的を射ており、政府に批判的な放送が不公正なものとして恣意的に扱われる場合には、一般市民からの信頼を得ることはできないであろう。ここから、松井による政府が公正さを要求することの危険性の指摘は、番組編集準則の運用に際しては、国家からの独立性を保つ必要があるという示唆を与えるものであるといえる。

\subsection{番組編集準則の運用についての試論}

以上より、番組編集準則の運用に関する二つの視点を提示することができる。一つは、番組編集準則の遵守による、バルカン化への対応である。二つは、番組編集準則の運用において、国家から独立した機関に判断を委ねる必要性である。

この視点に基づいて、私見を敷衍すると、NHKについては番組編集準則の遵守義務を課す一方で、民間放送については番組編集準則を撤廃すること、また、番組編集準則の判断については、総務省ではなく国家から独立した機関に判断を委ねること、が必要であると考える。

NHKに番組編集準則の遵守義務を課す理由としては、ネット時代におけるバルカン化に対応するため、という理由があげられる。

すなわち、放送だけでなく、インターネット情報提供事業も含めて視聴行動が分散した社会においては、人々が自分の選好に応じたコンテンツだけを選び、視野が狭く、多様な意見に触れることのない状況が生じうる。このような場合に、全ての放送事業者が政治的公平や多角的解明の準則から解放されるとすると、人々はどの放送局、コンテンツを通しても、民主的政治体制を維持していくため、あるいは自己実現のために必要な基本的情報を手に入れることができないという状況が発生する危険性がある。

そのため、少なくとも一つの放送局は、ある種の「防波堤」として、人々が基本的情報を手に入れたいと考えた時に提供できる体制を構築し、市民からの信頼を維持することが必要である。NHKは、財源を受信料という特殊な負担金に求めており、民間放送やインターネット情報提供事業とは異なる原理で、すなわち、市場原理から一定の距離を保った番組制作を行うことができる。このことから、番組編集準則の遵守義務の下で基本的情報の提供に責務を負うのに適するのはNHKであるといえる。

民間放送に対して番組編集準則を撤廃する理由としては、インターネット情報提供事業者の拡大によって生じる、番組内容に関する自主規制の機能不全が挙げられる。

長谷部の提示する「暗いシナリオ」にあるように、番組編集準則とそれを元にした自主規制は、他の事業者も守るという前提があるからこそ意味を保つ、いわば「カルテル」的な性質を持つものであり、多数の事業者が新規に参入した場合には「カルテル」の締結と執行はより困難である\footnote{長谷部・前掲注(4)206頁。}。ネット時代においては、新規参入者であるインターネット情報提供事業者に番組編集準則を遵守させることは困難であり、また、その必要性もない。その中で、既存の民間放送にだけ番組編集準則が適用されるとすると、収益構造において同一性を持つ、民間放送とインターネット情報提供事業者との競争において、民間放送にのみ不利益を課すものとなり合理的ではない。民間放送が、従来の番組編集準則、自主規制を維持し続けたい場合は、自局の独自のルールとして視聴者に提示する形をとることができる。よって、民間放送に対しては番組編集準則を撤廃するべきである。

また、国家から独立した機関に番組編集準則の判断を委ねる理由としては、公共放送が、番組編集準則の存在を盾にした国家からの干渉を受けることを阻止するためである。

国家から不当な干渉を受けない、政治的に公平かつ、多角的に論点を取り上げるメディアであるからこそ、公共放送は市民からの信頼を得ることができるのであり、バルカン化に対する最後の「防波堤」としての役割を果たすことができるのである。

\section{結}

以上をまとめると、日本の放送制度は、民放とNHKの二元体制としての放送秩序が続いてきたが、近年二元体制に揺らぎが生じている。公共放送の意義について判示した平成29年判決は、憲法上の放送の意義として国民の公共の福祉に資することをあげており、重要である。

しかし、二元体制をとる意義については、「基本的情報」説、部分規制論による説明が必要となる。マスメディアの自由は市民に基本的情報を提供するがゆえに保障される自由であり、マスメディアへの情報のボトルネックを解消するための方策として、放送にのみ特別の規制を課すという部分規制論が登場する。

現状の分析としては、視聴者のバルカン現象が生じつつあるからこそ、公共放送の役割が重要性を増していると捉えることができる。

ネット時代においては、公共放送と民間放送の二元体制が崩れうるため、バルカン化に対抗するための「防波堤」として公共放送が重要な意義を持つ。ここにおいて、公共放送が国民から信頼されるものとなるための制度的提案が、番組編集準則を公共放送については遵守義務を課す一方で、民間放送については撤廃するというものである。この際、番組編集準則を運用するのは国家から独立した機関であることが重要である。

公共放送と国家との関係については、番組編集準則だけではなく、NHKの経営体制や受信料制度なども踏まえた議論が必要であるが、筆者の力不足によりここでは検討しきることができなかった。

\section{参考文献}

\noindent
・芦部信喜「放送の自由」法学教室180号(1995年)。\\
・新井誠=曽我部真裕=佐々木くみ=横大道聡『憲法\ajRoman{2} 人権』(日本評論社、2017年)151頁〔曽我部真裕執筆〕。\\
・鈴木秀美=山田健太編『放送制度概論』(商事法務、2017年)。\\
・曽我部真裕「受信料制度の合憲性」別冊ジュリストメディア判例百選(第二版、2018年)。\\
・西土彰一郎『放送の自由の基層』(信山社、2011年)。\\
・西土彰一郎「部分規制論」駒村圭吾=鈴木秀美編『表現の自由\ajRoman{1}状況へ』(2011年)。\\
・西土彰一郎「メディアと憲法
公共放送の財源------NHK受信料訴訟大法廷判決をうけて」論究ジュリスト25号(2018年)。\\
・長谷部恭男『テレビの憲法理論』(弘文堂、1992年)。\\
・長谷部恭男『憲法学のフロンティア』(岩波書店、2013年)169頁。\\
・舟田正之=長谷部恭男編『放送制度の現代的展開』(有斐閣、2001年)。\\
・関根智江「年層による差がさらに広がるテレビ視聴」放送研究と調査2011年12月号(2012年)。\\
・中山準之助=伊藤文=保高隆之=内堀諒太「テレビ・ラジオ視聴の現況」放送研究と調査2020年3月号(NHK出版、2020年)。\\
・公益財団法人 新聞通信調査会「メディアに関する全国世論調査」(2019年)。\\
・篠原英樹「タブーは鉱脈
NETFLIXのものづくり」『日本経済新聞』、2020年1月5日、電子版\\(\href{https://www.nikkei.com/article/DGXMZO53764840U9A221C1000000}{{https://www.nikkei.com/article/DGXMZO53764840U9A221C1000000}}
閲覧日:2020年12月26日)。
\end{document}