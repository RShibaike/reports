\documentclass[twoside]{jsarticle}
\usepackage[dvipdfmx,hidelinks]{hyperref}
\usepackage{pxjahyper}
\usepackage[jis2004]{otf}
\usepackage[dvipdfmx]{graphicx}
\usepackage{url}
\setcounter{page}{1}
\usepackage{fancyhdr}
\begin{document}
\title{インターネット上の誹謗中傷と表現の自由}
\author{C2班\quad 橋本 泰樹・前多 陸}
\date{}
\maketitle
\tableofcontents
\clearpage
\pagestyle{fancy}
\lhead[インターネット上の誹謗中傷と表現の自由]{}
\rhead[]{\leftmark}
\section{はじめに}

近年SNS上での誹謗中傷問題が深刻化している。2020年は特に、芸能人の誹謗中傷を理由とする自殺がメディアで大きく取り上げられ、問題が顕在化してきたといえる。そこで誹謗中傷的な言論に対しては規制が必要となってくる一方、インターネット上の自由な言論空間の形成にも配慮しなければならない。このような問題意識のもと、インターネット上の表現の自由と誹謗中傷の問題について、前半は実体法上の観点から、後半は手続法上の観点から、それぞれ検討していくこととする。

\section{インターネットの特殊媒体性について}

インターネット上の表現の自由を考える上で、インターネットが他のメディアと異なる特徴を持つということを考慮する必要がある。その特徴とは、インターネットにおいてはだれでも容易に表現を行うことができるため、批判を受けたとしても反論が可能であるということと、一般の個人による表現であるから、情報について、必ずしも十分な裏付けがあるとは限らないことである。インターネットはこのような特殊媒体性を持つため、名誉毀損的な表現について、従来の免責法理が適用できるかが、議論されてきた。

ここで言う「従来の免責法理」とは、名誉毀損責任の免責法理として採用されている相当性の法理のことである。相当性の法理とは、\UTF{2460}摘示された事実が公共の利害に関わり、\UTF{2461}その目的が主として公益を図ることであり、\UTF{2462}当該事実が真実であることが証明された、または行為者において当該事実が真実であると信じるについて相当の理由があるとき、名誉毀損罪は成立しないこととする法理であり、これは民事・刑事を通じて採用されている。

\section{ラーメンフランチャイズ事件}

具体的事件についてインターネット上の表現がどのように扱われてきたのかを検討していく。まずは、平和神軍観察会事件\footnote{最高裁平成22年3月15日第一小法廷決定判時2075号160頁。

Yが、インターネット上に開設した「平和神軍観察会」と題するホームページにおいて、ラーメン店チェーンの運営会社Aは、一部マスコミでカルト教団として取り上げられていたBという団体と一体であり、加盟店からAへ、AからBへと資金が流れており、A社は虚偽の内容の求人広告を行っている旨の内容を記載した文章を掲載し、名誉毀損罪に問われた。}(別名ラーメンフランチャイズ事件)である。

この事件では、インターネット上の言論における名誉毀損責任の免責については、上記の相当性の法理を用いるべきか、インターネットの特殊媒体性を考慮した新たな基準を設けるべきかと言うことが争点となった。

一審では、インターネットの特殊媒体性に注目し、個人利用者によるインターネット上の名誉毀損罪は、被害者による反論を要求しても不当とは言えないような特段の事情があり、事実の公共性、目的の公益性が認められる場合には、摘示した事実が真実でないことを知りながら発信したか、あるいは、インターネット上の個人利用者に対して要求される水準を満たす調査を行わないで真実かどうかを確かめずに発信したといえる時に限り成立するとして、被告は無罪となった。

この判決は、インターネット上の言論における名誉毀損責任について、新たな免責基準を設けた点と、被害者に一方的に反論の負担を負わせる不当性にも配慮した点で注目に値する。しかし、加害者の果たすべき調査の水準が緩和されていることについての説明が不十分であるという点に問題がある。

現に反論を受け、言論の応酬がおこなわれている場面では、反論を効果的なものにするために迅速に反論をおこなうことが求められるし、その限度でおこないうる調査を果たせば足りると解する。しかしこの判決では、言論の応酬状態が実際に生じていなかったとしても、反論の可能性の存在だけをもって、加害者側の果たすべき調査の水準が緩和されている、ということについて十分に説明がなされていない。

これに対して、二審と最高裁は、個人の利用であっても、おしなべて、閲読者にとって信頼性の低い情報と受け取るとは限らないこと、インターネット上に載せた情報は、不特定多数のインターネット利用者が瞬時に閲覧可能であり、これによる名誉毀損の被害は時として深刻になること、一度損なわれた名誉の回復は容易ではなく、インターネット上の反論によって十分にその回復が図られる保証があるわけでもないことを考慮して、他のメディアによる表現の場合と同様に、相当性の法理を適用すべきであり、より緩やかな用件で同罪の成立を否定すべきものとは解されないとして、被告を有罪とした。

この事件では、一審判決において、インターネット上の表現における名誉棄損責任の免責法理の基準が示されたが、最高裁判決において、その基準は否定され、インターネット上の表現でも、ほかの表現方法の場合と同じように、相当性の法理を適用すべきだとされ、以降の事件でも基準となっている。しかし、インターネットを利用する一般の個人に対してもマスメディアと同等の調査を求めている点は酷であるともいえる。

\section{リツイート訴訟}

次に、SNSにおける表現行為について考えるにあたって、利用者数の多い代表的なSNSであるTwitterを例にして検討していく。Twitterは、1投稿につき140字以内という字数制限の中で個々のユーザーが自由に情報発信(ツイート)をすることのできるSNSサービスであり、ユーザーは原則として公開に設定されている他のアカウントの投稿を自由に閲覧することが可能なほか、アカウントをフォローすることにより、そのユーザーの投稿を自分のタイムラインに表示させて閲覧することができる。また、Twitterにおいては、他人がした投稿を引用する形式で自己のアカウントから投稿すること(リツイート)も可能であり、その場合コメントを付さずに元ツイートをそのまま引用すること(単純リツイート)が可能なほか、元ツイートにコメントを付した形で引用すること(引用リツイート)も可能である。Twitterでアカウントを作成すれば誰でも、以上のリツイートを含む投稿を自由に行うことが可能であり、ユーザーは原則としてそれらの投稿を自由に閲覧することが可能となる。それにより、一般的にユーザーは多数の他ユーザーによる投稿を大量に閲読することになり、また簡単な操作で可能なリツイートにより、一つのツイートがより多数のユーザーに閲読される(拡散される)状態になりうるという特色がある。こうした特色をふまえ、Twitterにおける表現行為(特に、名誉毀損に該当するような他人の権利を侵害する表現)を、表現の自由の観点からどのように理解するべきかが問題となる。

インターネットにおける表現の自由についての前述の最高裁判例に従うと、Twitterにおいてツイート(投稿)により他者の名誉を毀損する行為も、オフラインにおける表現行為と同様の免責法理により処理されることになると考えられる。一方で、他アカウントの投稿を(簡単な操作により)引用して自らのタイムラインに表示させるリツイートについて、どのように考えるべきか問題となる。具体的には、安易になされるリツイート行為をリツイート者の表現行為と考えてよいか(すなわち、そのリツイートされた投稿による権利侵害についてリツイート者に責任があると考えてよいか)や、また、リツイート者による表現であるといえるとすれば、そのリツイートについてどのような意義の表現といえるのかが問題となる。

この点につき、コメントを付さずになされたリツイート(単純リツイート)による投稿が名誉毀損に該当するとされた事例\footnote{大阪地裁令和1年9月12日判時2434号41頁。}がある。この事件は、ジャーナリストであってTwitterのフォロワーが18万人いたとされる被告が、原告(元大阪市長)の名誉を毀損する内容のツイートをコメントを付さずにリツイートしたことが名誉毀損に該当するとされ、不法行為責任(民法709条)に問われた事案である。この判決において裁判所は、リツイートには様々な意図で行われるものがあるとした上で、コメントを付さずに行う単純リツイートについて、「何らのコメントも付加せず元ツイートをそのまま引用するリツイートは、ツイッターを利用する一般の閲読者の普通の注意と読み方を基準とすれば、例えば、前後のツイートの内容から投稿者が当該リツイートをした意図が読み取れる場合など、一般の閲読者をして投稿者が当該リツイートをした意図が理解できるような特段の事情の認められない限り、リツイートの投稿者が、自身のフォロワーに対し、当該元ツイートの内容に賛同する意思を示して行う表現行為と解するのが相当である」と判示した。すなわち、リツイートによる権利侵害についてリツイート者に責任が帰属するとしたうえで、コメントを付さない単純リツイートを原則として元ツイートに賛同する意図の表現行為として解されるものとした。

この事件で問題となったのは、原告の名誉を毀損する内容の訴外ユーザーによる元投稿の、被告による単純リツイートである。前述したTwitterにおける表現の特色などを踏まえ、安易に、また大量になされる単純リツイートを、リツイート者による新たな表現行為として、オフラインにおける表現行為と同等の法理により名誉毀損への該当性を判断することが適切であるかが問題であったが、裁判所はそれを肯定したという形になる。

また、控訴審\footnote{大阪高裁令和2年6月23日裁判所ウェブサイト。}判決においては、追加的に、原告(被控訴人)の元大阪市長は当時フォロワー約214万人のTwitterアカウントを有しており、対抗言論が容易であったこと(それにより、本件投稿に対しては言論による対抗が十分可能であって、被控訴人に精神的損害は生じていないこと)も主張されたが、大阪高裁は、「人の社会的評価を低下させる表現行為が行われれば,それによる精神的苦痛という損害は直ちに発生するというべきであって,その後の被害者の行為により損害が消滅等するものではなく,控訴人の主張するように被害者において当該表現行為に対し言論をもって反論することが十分に可能であるとしても,当該表現行為が被害者の名誉を毀損する不法行為に該当する限りにおいて,その一事をもって被害者が加害者に対し不法行為による損害賠償を求める方法によってその損害の回復を図るみちが閉ざされるものでないことは明らかというべきである」と判示し、控訴人の主張を退けた。ここでは、Twitterという言論空間における前述の特色からして、対抗言論の容易さが損害の発生という時点において考慮されないかが問題となったが、裁判所は本事件においてそれを否定したという形になる。前述のラーメンフランチャイズ事件判決において最高裁は、インターネット上の名誉毀損表現の免責法理を考えるにあたって、対抗言論の容易さからより緩やかな免責法理を採用するという論理を否定したが、本件においても、インターネットの特殊媒体性の一つである対抗言論の容易さについて新たな特別の考慮はされなかったものと捉えることができる。

また、参考として、SNSにおける「イイネ」が不法行為責任に問われうるかが問題となった事案\footnote{東京地判平成26年3月20日裁判所ウェブサイト。}を紹介する。この事案は、SNSのミクシィで原告と被告が相互に誹謗中傷を行ったことについて、互いに訴えを提起したものである。この事件の反訴にて、原告が、第三者がした被告Y1の名誉を毀損し、同被告を侮辱し、脅迫する内容の発言に賛同を示した(「イイネ」した)ことについて、不法行為責任が成立するかが問題となった。

判決では、「ミクシィ上のイイネ機能\footnote{ミクシィのイイネは元投稿に対してイイネしたことが小さく表示されるだけであり、イイネした当人の投稿として表示されるものではない。}は、ミクシィ上のつぶやきなどの発言に対して、賛同の意を示すものにとどまり、上記発言と同視することはできないから、仮に上記つぶやきなどが名誉を毀損する内容であったとしても、このつぶやきに対して「イイネ!」のタグをクリックしたことをもって、いまだそのつぶやきなどの内容について不法行為責任を負うことはないというべきである。」とされた。つまり「イイネ」しただけでは不法行為責任に問われることはないという結論が示された。

\section{配信サービスの抗弁}

インターネット上の表現行為について、オフラインにおける表現行為と同様の免責法理により名誉毀損該当性を判断するとしても、とりわけ前述のリツイートのように、個人のインターネットユーザーにより容易になされることが特色といえる表現行為について、インターネット個人利用者における名誉毀損の免責要件が厳しくなりすぎる(ひいては、萎縮効果につながる)という課題は残る。ここで、一般のインターネット利用者(表現者である一方で取材能力をもたず、従来の相当性の法理を杓子定規的に用いると免責のハードルが非常に高くなる)に対して、特別な対応を考えることはできないのかを検討していきたい。

ここで、名誉毀損表現を行った表現主体の損害賠償責任を否定するための法理として、配信サービスの抗弁について考察していく。配信サービスの抗弁とは、定評ある通信社等からの配信を受けた記事を裏付け取材無しで地方の新聞社等が利用する場合に、掲載しただけの地方新聞社等の損害賠償責任を否定すべきであるとする法理である。これは、全国的な取材網を持ち記者クラブにも加盟する通信社と、取材能力に限界のある地方紙等のメディアが契約を結び、全国的なニュース等について通信社から配信を受けた記事を掲載できるようにしているというような場合が前提として考えられているものである。そのような場合に、取材体制の整備された信頼ある通信社の配信記事の内容について、配信を受けた報道機関は、そこに摘示された事実が真実であると信頼することについて相当な理由があるといえるのではないか、ということがここで取られている考え方であり、アメリカ等では確立した法理となっている。この法理を認めることができれば、取材能力のない表現主体による表現行為の免責法理として、定評ある報道機関の発表に依拠して表現行為を行った個人の損害賠償責任を否定することも類推的に考えられうるのではないかともいえる。以下では、配信サービスの抗弁が認められるかが争点となった判例について紹介していく。

まず、配信サービスの抗弁が認められるかが争点となった判例として、最高裁平成14年1月13日第三小法廷判決があげられる。この事件は、当時タレント等としても活動していた原告が重大な刑事事件に関与しているのではないかという内容のスキャンダル報道がメディアで過熱しており、そうした一連の報道の中で通信社からの配信を受けて原告の名誉を毀損する記事を掲載した地方新聞社が、名誉毀損による損害賠償責任を負うかどうかが問題となったものである。

この事件において最高裁は、「少なくとも、本件配信記事のように、社会の関心と興味をひく私人の犯罪行為やスキャンダルないしこれに関連する事実を内容とする分野における報道については\ldots{}報道が加熱する余り、取材に慎重さを欠いた真実でない内容の報道がまま見られるのであって\ldots{}一定の信頼性を有しているとされる通信社からの配信記事であっても、我が国においては当該配信記事に摘示された事実の真実性について高い信頼性が確立しているということはできない」、「仮に、その他の報道分野においていわゆる配信サービスの抗弁\ldots{}を認める余地があるとしても、私人の犯罪行為等に関する報道分野における記事については、そのような法理を認め得るための、配信記事の信頼性に関する定評という一つの重要な前提が欠けている」と判示した。すなわち、本件においては配信サービスの抗弁の適用を認めなかったものの、一般論としては配信サービスの抗弁の適用の余地を残したものである。

なお、本事件と類似の事案(最高裁平成14年3月8日第二小法廷判決)において、結論としてはこちらでも配信サービスの抗弁は認められなかったものの、「このような報道システムは、地方の報道機関が物理的、経済的及び人的制約を越えて世界的、全国的事件を報道することを可能にするものであって、報道の自由に資するもの」であり、通信社から配信された記事を掲載した報道機関の行為は、特別に憲法21条による正当な行為と見ることができ、違法性を阻却するとして、本件(私人の犯罪行為の報道)においても配信サービスの抗弁を認めるべきである、とする梶谷玄裁判官の反対意見が付されていたことは注目に値する。

一方その後、医師である原告(上告人)が医療ミスにより患者を死亡させたとする内容の記事を、配信を受けて掲載した新聞社が名誉毀損による損害賠償責任を負うかが問題となった最高裁平成23年4月28日第一小法廷判決において、通信社の配信を受けて記事を掲載した報道機関の免責を認める判断枠組みが示された。すなわち、新聞社が通信社からの配信に基づいて新聞に記事を掲載した場合において、「少なくとも、当該通信社と当該新聞社とが、記事の取材、作成、配信及び掲載という一連の過程において、報道主体としての一体性を有すると評価することができるときは、当該新聞社は、当該通信社を取材機関として利用し、取材を代行させたものとして、当該通信社の取材を当該新聞社の取材と同視することが相当であって、当該通信社が当該配信記事に摘示された事実を真実と信ずるについて相当の理由があるのであれば、当該新聞社が当該配信記事に摘示された事実の真実性に疑いを抱くべき事実があるにもかかわらずこれを漫然と掲載したなど特段の事情のない限り、当該新聞社が自己の発行する新聞に掲載した記事に、摘示された事実を真実と信ずるについても相当の理由があるというべきである。」とし、\UTF{2460}報道主体としての一体性があり、\UTF{2461}通信社が相当性の法理による免責要件を満たす場合には、記事を掲載した新聞社にも当該事実を信じるについて相当の理由があるとして免責されうるとしたのである。そして、通信社と新聞社とが報道主体として一体性を有すると評価すべきかは、「通信社と新聞社との関係、通信社から新聞社への記事配信の仕組み、新聞社による記事の内容の実質的変更の可否等の事情を総合考慮して判断するのが相当である」であるとし、通信社と組織上密接なかかわりがあり、また配信を受けた記事が掲載記事の50~60%を占め、一日に1500本程度の配信を受けていた本件の新聞社について、通信社との報道主体としての一体性があり新聞社において新たに取材を行うことは想定されていないものであるとして免責を認めたのである。

配信サービスの抗弁は、十分な取材能力を持たない報道機関について確立されている報道体制において、配信を受ける側である報道機関の免責要件を緩和しようとするものである。この点、インターネットの個人利用者も、表現主体である一方で取材能力をもたないという点において類似の問題があるものと考えられるが、配信サービスの抗弁に関する上記の議論から柔軟な免責法理を検討できないだろうか。最高裁平成23年4月28日第一小法廷判決で認められた判断枠組みは、配信を行った通信社と、配信記事を掲載した新聞社等が報道主体としての一体性があるといえる場合に、通信社において行われた取材行為を新聞社等によるものと同視してそれによる免責を認めるものである。何ら報道機関などと関わりをもたずに情報発信を行うインターネット個人利用者は、報道機関等と報道主体として一体性を持つものとはいえず、この枠組みからは免責されないもののように思われる。

しかし、報道主体としての性格をもたない個人について、一切の例外なく従来の相当性の法理による免責要件を求めることは適切なのであろうか。例えば、ある大手メディアの記事が(事実の摘示による)名誉毀損に該当するとされ、当該記事を閲覧した個人利用者Aが、記事に関するツイートをしていたという場合を想定してみると、元記事を作成・掲載したメディア自身は相当性法理で免責の可能性があるにもかかわらず、ツイートをしたAについても個別に相当性法理が適用され、元メディアと同等の免責要件を満たさない限り法的責任を負うとするのは不均衡だといえるのではないか。

インターネットはあらゆる個人に情報発信能力を与え、自由な言論空間を提供するツールである。インターネットの普及により情報発信主体は多様化し、マスメディアのみが情報発信主体であった時代から、マスメディアに加えてインターネット空間におけるあらゆる私人が情報発信主体である時代になった。だがこのことは、私人が従来のマスメディアと同等のファクトチェック能力を有することを前提とするものではなく、取材能力を有して大きな影響力を持つ従来のマスメディアに加えて、取材能力の想定されない私人にも情報発信主体としての能力が与えられるようになったということに過ぎない。私人の表現行為においても、それが他人の権利を侵害する場合には損害を賠償する責任が発生するものと考えることは当然であるが、表現行為を正当なものとして免責するための要件として常に取材能力に裏付けられた誤信相当性等を要求するのであれば、上記のインターネット言論空間の特質を否定するものといえるのではないか。

確かにインターネット掲示板やSNS上の何ら根拠のない名誉毀損表現やその拡散行為は極めて悪質であり、その被害はとても軽視してよいものではない。例えば、2019年8月に発生した常盤自動車道のあおり運転事件で、加害者の車に同乗していた女性であるとのデマをインターネット上に流され名誉を傷つけられた女性が、Facebook上でデマを拡散した元市議の男性に対して損害賠償を請求した事例\footnote{東京地判令和2年8月17日裁判所ウェブサイト。}などは、悪質な名誉毀損表現による深刻な被害が発生した事例として記憶に新しい。このような、根拠のない名誉毀損表現やその拡散行為は法的責任を問われるべきであることに疑いはない。

しかし一方で、提供する表現について責任主体の明確なマスメディアの報道記事などに依拠して行ったインターネット個人利用者の表現行為・拡散行為は免責されるべきだと考えられるのではないだろうか。少なくとも、依拠した情報の特定の発信者が、従来の相当性の法理により(すなわち、公益性の要件を満たし、その情報が真実であるかまたは十分な裏付け取材を行っており当該情報を誤信することについて相当性があるといえるような場合であるとして)免責される場合には、実際にそのような取材を行っている発信源の情報を信用して行われた表現行為については、私人が物理的制約を越えて自由な言論空間を享受することのできるインターネットの性質が表現の自由に資するものであることに鑑み、憲法21条による正当な行為として違法性を阻却すると考えるべきではないだろうか。

\section{プロバイダ責任制限法について}

インターネット上で権利侵害情報が流通した場合、権利侵害の被害者としては、その情報の削除請求(拡散防止措置)や発信者への責任追及を行うことが考えられる。しかし情報を提供しているプロバイダとしては、削除した場合、或いは削除しなかった場合の責任範囲が明確化されていないと権利侵害情報に適切に対応できない。また、匿名発信者の情報が開示されなければ、被害者は発信者に責任追及することができない。

特定電気通信役務提供者の損害賠償責任の制限及び発信者情報の開示に関する法律(プロバイダ責任制限法)とは、そのような問題に対処するための法律である。3条でプロバイダ等の権利侵害情報の流通や適法情報の削除についての責任が明確にされており、4条で発信者情報開示請求権を定めている。

まず3条では、電子掲示板の管理者、ホスティングプロバイダ等の特定電気通信役務提供者(以下「プロバイダ等」とする)の損害賠償責任を一定の場合に制限する旨を規定している。特定電気通信による情報がインターネット上に流通した場合、一方でプロバイダ等が当該情報について削除等の送信防止措置を講じないときには、情報の流通により権利を侵害された者から損害賠償責任(不作為責任)を追及される可能性がある。他方で、プロバイダ等が当該情報について削除等の送信防止措置を講じた場合には、情報を流通過程においた発信者から損害賠償責任(作為責任)を追求される可能性がある。このようにプロバイダ等は権利を侵害された者と当該情報の発信者の権利保護との間で、いわゆる「板挟み」の状態に置かれることになる。この問題に対処するために、3条1項では、当該情報の送信防止措置を講じなかった場合における、権利を侵害された者に対するプロバイダ等の不作為責任の範囲を明確にされ、削除されるべきでない情報について過度に送信防止措置が講じられ表現の自由が不当に侵害されることを抑止している。そして3条2項では、当該情報について送信防止措置を講じた場合における、発信者に対するプロバイダ等の作為責任の範囲を明確にすることで、削除されるべき情報については迅速に適切な対応がされることを促している\footnote{関原秀行『基本講義プロバイダ責任制限法 インターネット上の違法・有害情報に関する法律実務』(日本加除出版株式会社・
  2016 年)43~59頁。}。

流通に置かれた情報によって権利を侵害された者が発信者に民事責任を追及しようとするためには、発信者を特定する必要がある。しかし特定電気通信による情報の発信は匿名で行われることが多く\footnote{例として、電子掲示板への匿名での書き込みなど。}、利用者がウェブサイトに表示されている情報だけで発信者を特定するのは困難である。4条ではこのような、当該情報の発信者が匿名であるがために責任追及することができないという問題に対処するための、プロバイダ等に対する権利を侵害された者の「発信者情報開示請求権」が定められている。具体的には4条1項で実体法上の請求権としての発信者情報開示請求権を規定し、同条2項以下で発信者情報開示請求がされた場合の発信者の権利保護について定めている\footnote{関原・
  前掲注6)79頁以下。}。

\section{発信者情報開示請求の諸問題}

発信者情報開示請求では、請求権者の被害の救済という法益と、発信者の匿名による表現の自由、プライバシーの権利(憲法13条後段)、通信の秘密\footnote{通信の秘密は、誰にも通信の内容や存在、相手方といった事実を知られずに行うことができるといった内容のものである。憲法21条2項は個人間の通信を想定したものであるが、SNSや掲示板への投稿についても、発信者とプロバイダ等との通信が通信の秘密として保護される。}(憲法21条2項、電気通信事業法4条)といった利益が対立している。そして、このような発信者の利益が重要なものであることから、発信者情報開示請求が認められるための要件は厳格なものとなっている\footnote{プロバイダ責任制限法4条1項1号「\ldots{}権利を侵害されたことが明らかであるとき」。}。そのため、発信者情報開示請求を受けたプロバイダ等としては、要件該当性の判断を行うについて困難がある。そして、請求権者に発信者情報を開示しなかったことによるプロバイダの責任は制限される(4条4項)一方、要件を満たしていないにも関わらず開示した場合にはプロバイダ等の責任について、制限規定はない。

したがって、プロバイダ等としては、裁判外の開示請求に対しては慎重にならざるを得ず、権利侵害が明白であると思われる場合であっても、実務上、発信者情報が裁判外で開示されることは少ない。

上記のように、発信者情報開示請求が認められること自体に困難があるということに加え、開示請求が認められた場合にも、この制度自体があまり実効的な権利救済の手段となっていないという問題がある。一つには、総務省令に定める内容だけでは発信者の特定に至らない可能性があるということである。現行の総務省令では、住所・氏名・メールアドレス・IPアドレス・タイムスタンプをその開示情報の内容として定めている。しかし、近年、投稿時のIPアドレス等を記録・保存していないコンテンツプロバイダ\footnote{掲示板運営者など。}が出現したことにより、投稿時のIPアドレスから通信経路を辿って発信者を特定するという手法がとれない場合が出てきたほか、アクセスプロバイダ\footnote{インターネットの通信業者等。発信者個人とインターネット通信の利用契約を交わしていることが多く、その場合には発信者の氏名・住所等の情報を保持している。}において特定のIPアドレスを割り当てた契約者を特定するために接続先IPアドレス等の付加的な情報を必要とする場合があるなど、現行の省令に定められている内容だけでは発信者を特定することが技術的に困難となる場面が増加しているのである。

もう一つは被害者にとって訴訟手続に多大な負担がかかるということである。これを実際に権利侵害投稿が行われた場面において被害者がとるべき行動を例にとって説明する。インターネット上で権利侵害投稿が行われた場合に、一般的にはコンテンツプロバイダは発信者の氏名・住所等の情報を保有していないことが多い。そこで被害者はまず\UTF{2460}コンテンツプロバイダに当該投稿時のIPアドレスの開示を請求する。そしてそのIPアドレスを元にアクセスプロバイダを特定して、アクセスプロバイダに対して\UTF{2461}発信者の氏名・住所等の開示を請求する。そうして発信者を特定した被害者は、ようやく\UTF{2462}発信者に対して損害賠償請求等を行うことができるのである。\UTF{2460}のコンテンツプロバイダに対する開示請求では、通信履歴(ログ)保全の必要性から仮処分の申立てによって行われるのが一般的である。一方で、\UTF{2461}のアクセスプロバイダに対する開示請求では、開示される情報がそれのみで発信者個人を特定できるものであり、保護の必要性が大きいことから訴訟提起によるのが一般的である。

このように、インターネット上の誹謗中傷等で権利を侵害された被害者は、通常3段階の訴訟手続を踏まなければ自己の利益の救済を図ることができないのである。

また、発信者の特定に必要となり得る通信履歴(ログ)についても、通信の秘密や個人情報保護の観点から、プロバイダ等は業務の遂行の目的に必要な範囲で保存期間を定め、保存期間経過後は速やかに削除する必要があるため、プロバイダ等は通信履歴等の発信者情報を保存する義務は負わず、消去したとしても基本的には請求権者に対して民事上の責任を負うことはない。したがって、実効的な救済のためには請求権者がプロバイダ等に対して発信者情報の保全請求をする必要が生じたり、すでに通信履歴が削除されて発信者を特定できないといった問題が生じたりすることもある。

以上を踏まえて、発信者情報開示請求の諸問題について検討する。

まず、開示請求自体の要件が厳格であるという問題について考える。発信者の利益が重要なものであること、誤って開示されてしまった場合にはプライバシーなどの損害が重大なものとなる上、性質上原状回復が困難なものであることを考慮すると、開示請求の要件自体を緩和することについては、慎重にならざるを得ない。

次に、プロ責法4条自体が被害者にとって実効的な救済方法となっていないという問題についてであるが、これは次章で検討していくこととする。

\section{発信者情報開示の在り方に関する研究会について}

現行の発信者情報開示制度についてその問題点を確認する。\UTF{2460}総務省令に定められている内容だけでは発信者の特定に至らない可能性があること、\UTF{2461}通信履歴等がプロバイダにおいて削除されており、発信者を特定できない場合があること、\UTF{2462}訴訟手続に多大な負担がかかることである。

このような問題に対処するために、「発信者情報開示の在り方に関する研究会」が総務省において発足し、発信者情報開示制度の見直しが図られた。以下ではその内容について紹介\footnote{以下8-a~8-cについて、総務省『発信者情報開示の在り方に関する研究会 中間とりまとめ』及び『発信者情報開示の在り方に関する研究会 最終とりまとめ(案)』参照。}しつつ検討を加えていく。

\subsection{開示情報範囲の拡大}

\UTF{2460}の総務省令に定められている内容だけでは発信者の特定に至らない可能性があるという問題について、開示の対象となる情報の範囲を拡大するという方策が考えられる。そこで新たに加えるべき開示情報として、電話番号とログイン時情報について検討する。

\subsubsection{まず、電話番号についてその必要性、有用性を考える。近年のインターネット上の権利侵害投稿はSNSを媒体としているものが増加しており、SNSを提供する主要なコンテンツプロバイダの中には投稿時のIPアドレス等を保持していないものも多い。また、SNS等ではアカウント作成時において連絡先の登録を行うことや、不正ログイン等を防止するセキュリティ対策を目的とした電話番号の登録が一般化しつつあり、これらのコンテンツプロバイダがユーザの登録者情報として電話番号を保有しているケースが増加している。そのため、電話番号を開示情報に加えることには必要性、有用性がともに認められ、今回の改正において電話番号の項目は追加された。}

\begin{quote}
しかし、電話番号の有用性については、SNSを利用するに当たっては電話番号の登録が義務づけられているものばかりではないこと、一人が複数のアカウントを所持している場合もあり、誹謗中傷をするアカウントで電話番号の登録を行っているとは考えにくいこと等の理由により、疑問が残る。
\end{quote}

\subsubsection{次に、ログイン時情報について検討する。まず、ログイン時情報とはログイン時の通信におけるIPアドレスやタイムスタンプのことである。これに対して投稿時の通信におけるIPアドレスやタイムスタンプを投稿時情報といい、現行法において開示の対象となっている(総務省令に規定されているIPアドレスやタイムスタンプ)のはこの投稿時情報である。しかし、主要なSNSの中には投稿時情報を保持せず、ログイン時情報しか保持していないものがあり、このようなSNSで権利侵害が生じたときには、請求したとしても情報がないために開示されないということが起こりうる。}

したがって、ログイン時情報から発信者を特定することには必要性が認められる。ただし、ログイン時の通信は権利侵害投稿の通信その者ではないことから、ログイン時情報を開示対象とするに当たっては、発信者の同一性と開示の対象とすべきログイン時情報の範囲に留意する必要がある。

\paragraph{発信者の同一性}

\begin{quote}
ログイン時の通信と権利侵害投稿時の通信は違うため、それぞれの通信の発信者が異なる場合がある。仮にログイン時情報として、権利侵害投稿の発信者以外の者の情報が開示されてしまった場合には、当該発信者以外の者の通信の秘密やプライバシー等を侵害することとなる。そのため、両通信の発信者が同一であるときに限り、開示できるものとする必要があるのだが、プロバイダ等において発信者の同一性を確認することは困難であることが予測される。
\end{quote}

\paragraph{開示の対象とすべきログイン時情報の範囲}

開示が認められる場合の要件としては、コンテンツプロバイダが投稿時のログを保有していない場合など、侵害投稿時の通信経路を辿って発信者を特定することができない場合に限定することが必要である。

また、開示の対象となるログイン時情報の範囲が際限なく拡大すれば、権利侵害投稿と関係の薄い通信の秘密やプライバシーを侵害するおそれが高まるため、開示の対象とすべき情報の範囲には一定の限定を付す必要がある。具体的には、権利侵害投稿の前提となる行為としてのログイン時情報のみを対象とするなど、権利侵害投稿と深い関連性が認められる必要最小限のものに限定するのがよい。そして、例外的な事由がある場合に限り、ログイン用のアカウントを取得する際の通信、侵害投稿が発信された後のログアウト時の通信、侵害投稿が発信された後のログイン時の通信に係る情報についても開示対象とすることが認められた。

\subsection{非訟手続}

現行の発信者情報開示手続においては、ログ(通信履歴)等がプロバイダにおいて消去されており発信者を特定できない場合がある、訴訟手続に多大な負担がかかる、などの問題点があり、被害者の実効的な救済への妨げになっていた(前記\UTF{2461}\UTF{2462})。そこで、発信者情報開示の在り方に関する研究会においては、これらの点を改善するための新たな訴訟手続きとして、非訟手続の導入が議論された。

前述したように、プロバイダに対して発信者情報開示請求を行うにあたっては、発信者の通信のログが残存していることが必要であり、とりわけ第二段階の開示請求(アクセスプロバイダに対する開示請求)にあたって実効的な救済のためには、通信のログが消去されてしまう前に、開示請求とは別にログの保全請求を迅速に行うことも必要になる。またそれらの事情も含め、現行制度では請求額と比較して訴訟費用が高額になってしまうことや、訴訟が長期化するなど、被害者にとって負担が大きく、また加害者にとっても和解費用の増大を招くことにもなり、訴訟経済上不利益が大きいものとなっている。これらの点を改善するため、発信者情報開示請求の在り方に関する研究会では、従来の開示手続(発信者情報開示請求権に基づく請求)に加え、コンテンツプロバイダやアクセスプロバイダに対する開示請求、及びログの保全請求を一本化した非訟手続の導入が検討された。非訟手続とは、いわば裁判所による行政作用とも称され、柔軟な制度設計が可能な紛争解決手続であり、比較的簡易な手続により特定の紛争について裁判所による裁量的な判断を求めることができるものである。

発信者情報開示請求の在り方に関する研究会で検討されたのは、\UTF{2460}コンテンツプロバイダ及びアクセスプロバイダ等に対する発信者情報の開示命令、\UTF{2461}コンテンツプロバイダが保有する権利侵害に関係する発信者情報を被害者には秘密にしたまま、アクセスプロバイダに提供するための命令、\UTF{2462}アクセスプロバイダに対して、コンテンツプロバイダから提供された発信者情報を踏まえ権利侵害に関係する発信者情報の消去を禁止する命令、という3つの命令を一体的に非訟手続として位置付ける案である。被害者は\UTF{2460}~\UTF{2462}を裁判所に申し立て、まず\UTF{2461}の命令によりアクセスプロバイダが特定、発信者情報がアクセスプロバイダに提供され、\UTF{2462}の命令により発信者情報の保全がなされ、\UTF{2460}の命令により被害者に発信者情報が開示される、というのが大まかな流れとなる。

非訟手続は、プロバイダ責任制限法に規定された実体法上の請求権に基づく手続としてではなく、被害者からの申立てにより裁判所が発信者情報の開示の適否を判断・決定する仕組みであり、これにより開示請求を一体化できるというだけでなく、発信者情報を被害者には秘密にしたまま、コンテンツプロバイダに迅速に発信者情報を提出させ、アクセスプロバイダにおいて発信者を特定し、当該発信者情報を保全しておくというプロセスを一体的に行うことが可能となる。これにより、開示請求までの間、発信者のプライバシー等の侵害を抑えつつ、迅速に手続きを進めることが可能となり、特に権利侵害が明らかな誹謗中傷などの、争訟性が高いものではない事案について従来のものより迅速な判断が可能となる点が大きな利点である。

一方で、非訟手続には、導入に当たって検討しなければならない課題もある。まず、裁判手続の取下げ等が比較的容易であり、紛争の蒸し返しや手続きの濫用の虞があるのではないかということが指摘されている。この点については、従来の発信者情報開示手続と非訟手続を併置することにより、非訟手続で異議なく開示の可否が判断された場合には既判力を生じさせることができ、濫用的な蒸し返しを防止することができるとされている。

また、非訟手続は対審構造をとることが原則とはされていないことから、開示手続において発信者の意見をどのように反映させるかが問題となる。すなわち、制度設計次第では、発信者側の主張内容が裁判手続に十分に反映されないことにより、適法な情報発信を行う発信者の保護が十分に図られなくなるおそれがある。

この点、発信者情報を保有しているのはプロバイダであるため、新たな非訟手続においてもプロバイダを相手方とすることになると考えられる。現行の訴訟手続においては、発信者にはプロバイダから意見の照会が行われることによって、開示手続に発信者の意見が反映されるようにされている\footnote{プロバイダ責任制限法4条2項参照。}が、発信者の権利利益を確保する制度設計を非訟手続において維持するためにどのような工夫が必要であるかが問題となる。この点については、プロバイダ等の民間事業者の対応と合わせて次章以下で触れていく。

\subsection{発信者の意見照会について}

発信者情報開示手続は、請求の相手方は発信者情報を保有するプロバイダ等になる一方で、その請求について直接の利害関係を有するのは訴外の発信者であるという特殊な当事者構造をとる手続である。非訟手続においても請求の相手方は発信者情報を保有するプロバイダ等の民間事業者になると考えられるため、発信者の権利利益を確保するため非訟手続において民間事業者はどのようにして発信者の意見を手続に反映させるべきかが問題となる。現行制度においてとられている前述の意見照会制度によることも考えられるが、非訟手続において意見照会制度を取るにあたってとりわけ考慮しなければならない点も存在する。

第一は、プロバイダの負担である。開示請求を受けたプロバイダは、裁判手続の中で発信者の利益を適切に擁護することが期待されるが、現行制度の下においても、裁判上の請求に対応する件数の増加等により負担が増し、そうした本来期待されている役割を果たすことが困難になっているなどの課題が指摘されている。こうした課題に対応するため、とりわけ多数の請求が迅速に行われることが期待される非訟手続において、発信者の利益擁護及び手続保障を充分に確保するためどのような仕組みとするべきかについて考える必要がある。すなわち、プロバイダが発信者に対する意見照会を適切に行わないなどの場合においても、発信者の手続保障を確保できるようにする観点から、発信者が(匿名性を保持したままで)裁判手続に関与することを可能とするような措置などについて検討する必要がある。

この点、まずは発信者の意見照会について義務を負っているプロバイダからの発信者照会がなされるべきであるが、一方でそれが適切に行われないような場合においては、発信者が追加的に意見を述べ、それを審理の中で反映させる仕組みが必要であると考えられる。このような仕組みについて、例えば、裁判所が発信者に直接話を聴くような手続など、発信者の匿名での手続関与を認める仕組みも考えられたが、他に例のない制度であり、当該手続に被害者を関与させることができず、発信者の主張等についての攻撃防御の機会の保障の面で問題があるといった点で、法制面及び裁判所の運用面でハードルが高いといった課題があるとされた。また現行制度の意見照会に加えて、裁判所が開示要件を満たすという心証を得た段階で裁判所がプロバイダに意見聴取の嘱託を行うなどの仕組みが提案されたが、全ての請求について2度の照会を行う必要性に疑問があるとされるほか、開示決定までの迅速性も失われるという課題が生じることも指摘された。結局のところ、開示手続の途中で発信者から追加的に意見を述べたい旨の意向が示された場合や、発信者自らが匿名化の責任を負った上で裁判所に書面により意見を提出したいという意向が示された場合に、プロバイダが可能な限り発信者の意向を尊重した上で、個別事案に合わせて具体的な対応がとられることを期待するべきであるとされる。

第二に問題となるのは、濫用的な申し立てについても意見照会が行われることにより、発信者への萎縮効果をもたらす可能性があるという点である。現行の発信者情報開示請求においても、請求の利害関係者である発信者が、自己の情報に関する開示手続きが進行していることを初めに認知するのはプロバイダからの意見照会を受けた段階であり、これにより心理的動揺を受ける発信者が多いとされている。このことを前提に、より簡易な手続きである非訟手続における発信者情報開示手続においても、およそ開示の要件を満たさないような場合も含めすべての場合に発信者に対する意見照会が行われるとすると、適法な情報発信者に対しても容易に心理的動揺を与えることが可能となり、インターネット上の表現の萎縮効果に繋がるのではないかという点がここでの問題となる。

この点、前述のプロバイダの発信者に対する意見照会制度の重要性に鑑み、意見照会制度そのものを廃止することなどは考え難い。一方、とりわけ、請求の濫用(いわゆるスラップ訴訟)のような場合に、発信者への意見照会を行わなくてもよいようにする工夫が必要である。この点、現行の意見照会制度においても、開示要件を満たさないことが一見して明白であるような「特別な事情」のある場合については、意見照会を行わなくてよいものとされているが、この点についての判断をプロバイダにおいて行うことは多くの場合難しいとされている。より簡易な手続きとして迅速な対応が期待される非訟手続において、そうした対応はより一層難しいものとなることが予想されるが、意見照会を行うことを原則としつつも、意見照会を不要とする場合について、ガイドライン等にてなるべく詳細に定めていくことが望ましいというほかない。

\section{まとめ}

インターネットはあらゆる個人に情報発信能力を与え、自由な言論空間を提供する優れたツールである。ここ10数年の間にSNSをはじめとしてインターネットは我々の生活に広く浸透し、個人が匿名で容易に情報発信を行うことが可能となった。一方インターネット上の悪質な情報による権利侵害も、情報の拡散の容易さや、一旦拡散した情報を消去することの困難さ等から、インターネットの普及と比例するように深刻さを増してきたといえる。オフラインで違法な表現はオンラインでも違法である、とする前述のラーメンフランチャイズ事件最高裁判決で示されたように、根拠のない名誉毀損のような表現を不用意に拡散する、メディアリテラシーのない行為は法的責任を問われるべきものであり、インターネット上の表現の自由が一定程度制約されるとしても、投稿が名誉毀損にならないように配慮されることで
、健全な議論が形成されることに資すると考えられる。一方で、厳しすぎる制約はインターネット上の表現の萎縮効果を生むものでもあり、慎重になるべきである。とりわけ、メディアリテラシーをもってインターネット上の情報を信頼し、行った表現行為について、従来の誤信相当性の法理をそのまま用いることについては疑問を呈する余地があると考えられる。

また、インターネット上の表現による権利侵害をうけた被害者の救済のための発信者情報開示請求に関しては、手続をより実効的なものにする必要性は高まってきているが、被害者の被害の救済という法益と、情報発信者のプライバシー、通信の自由といった法益との調整という観点を常に考えて検討する必要がある。従来の裁判手続では費用と時間がかかりすぎることが多く、このことは被害者にとって大きな負担となることから、簡易な手続の導入が必要である。一方、その検討に当たっては、発信者情報開示請求の特殊な当事者構造に配慮する必要がある。また、裁判外での開示や非訟手続におけるアクセスプロバイダへの情報開示、発信者照会手続など、非訟手続においても現行制度においても、プロバイダが円滑に適切な対応を取ることは必要になるといえる。ガイドラインの策定や、あるいは要件該当性を判断する第三者機関の創設などにより、そうしたプロバイダの適切な対応を現実的なものとする働きかけはいずれにせよ重要になるといえる。

\section{参考文献}
\noindent
・小原健「いわゆる「配信サービスの抗弁」最二小判14・3・8判時1785・38。
」小倉京子ほか『名誉・信用毀損プライバシー侵害紛争事例解説集
初版』(新日本法規出版・2016年)。\\
・建部雅「不法行為制度のあり方を考える
複数の者が関与する損害発生における複層性の検討」論究ジュリスト16
号(2016 年)。\\
・前田聡「インターネット上での個人の表現行為と名誉毀損罪の成否」流経法学第9巻第1号。\\
・松井茂記『インターネットの憲法学 新版』(岩波書店・ 2014 年)。\\
・芦部信喜(高橋和之補訂)『憲法 第七版 』(岩波書店・ 2019 年)。\\
・岡田理樹ほか『発信者情報開示・削除請求の実務 インターネット上の権利侵害への対応』(商事法務・2016年)。\\
・佐藤佳弘「ネット中傷対応の現状と課題」都市問題111 号(2020年)。\\
・鈴木秀美「ドイツのSNS対策法と表現の自由」メディア・コミュニケーション
68 号(2018年)。\\
・関原秀行『基本講義プロバイダ責任制限法 インターネット上の違法・有害情報に関する法律実務』(日本加除出版株式会社・
2016 年)。\\
・総務省「通信の秘密、個人情報保護について」(2021年2月18日最終閲覧)\\
\href{https://www.soumu.go.jp/main_sosiki/joho_tsusin/d_faq/5Privacy.htm}{{https://www.soumu.go.jp/main\_sosiki/joho\_tsusin/d\_faq/5Privacy.htm}}。\\
・総務省「発信者情報開示の在り方に関する研究会 最終とりまとめ(案)」\\
\href{https://www.soumu.go.jp/main_content/000724725.pdf}{{https://www.soumu.go.jp/main\_content/000724725.pdf}} 。\\
・総務省「発信者情報開示の在り方に関する研究会 中間とりまとめ」\\
\href{https://www.soumu.go.jp/main_content/000705947.pdf}{{000705947.pdf
(soumu.go.jp)}}。\\
・第二東京弁護士会人権擁護委員会編『インターネットとヘイトスピーチ』(現代人文社・2019年)。\\
・壇俊光「発信者情報開示の現行法の問題点」\\
\href{https://www.soumu.go.jp/main_content/000105852.pdf}{{https://www.soumu.go.jp/main\_content/000105852.pdf}}。\\
・深澤諭史『インターネット権利侵害 削除請求・発信者情報開示請求``後''の法的対応Q\&A』(第一法規・2020年)。\\
・三木浩一ほか『民事訴訟法第3版』(有斐閣・2018年)。\\
・森亮二「誹謗中傷対策とスラップ訴訟」都市問題111号(2020年)。
\end{document}