\documentclass[twoside]{jsarticle}
\usepackage[dvipdfmx,hidelinks]{hyperref}
\usepackage{pxjahyper}
\usepackage[jis2004]{otf}
\usepackage[dvipdfmx]{graphicx}
\usepackage{url}
\setcounter{page}{1}
\usepackage{fancyhdr}
\begin{document}
\title{芸術に関する公的助成の憲法的統制}
\author{菖蒲隆行}
\date{}
\maketitle
\tableofcontents
\clearpage
\pagestyle{fancy}
\lhead[芸術に関する公的助成の憲法的統制]{}
\rhead[]{\leftmark}
\section{はじめに}

\subsection{問題の所在}

2019年の8月1日に開かれたあいちトリエンナーレの企画展である「表現の不自由展・その後」は、その展示されている作品の内容から、多くの政治家の批判や国民の電話・FAX・メールによる抗議、脅迫を受けた。その結果、わずか3日後である8月3日に、あいちトリエンナーレ実行委員会の会長である愛知県知事の大村秀章氏と芸術監督である津田大介氏は合意の上中止を決定した。その後、リスク回避策を講じたうえで、10月8日から展示は再開された。この中止決定について訴訟など法的措置が講じられることは今後ないだろうが、この中止決定は実質的に表現の自由を侵害するものとして大きな問題をはらんでいる。

\subsection{本レポートの構成}

以上に記した問題を明らかにするため、本レポートでは、次のような順にしたがって検討を進める。

まず前提として公的助成に対する公的機関の介入の可否と可能であるとした場合のその憲法的統制について検討したうえで、この中止決定に加えて、この問題に関する個々の事案を検討していきたい。

\section{公的助成に対する公的機関の介入の可否}

公的助成に対する公的機関の介入の可否を論じるうえで重要なのは、公的助成と表現の自由の関係について整理することである。すなわち、表現活動を行う者は表現の自由によって何が保障されているのかを明確にすることで、公的機関の介入が表現の自由に反するのか、反しないのかが見えてくるはずである。

表現の自由の保障について、富山県立近代美術館事件第一審判決\footnote{富山地判平成10年12月16日判時1699号120頁}は以下のように判断している。

「表現の自由の保障とは、情報収集――情報提供――情報受領という情報の流通過程のうち、情報提供の過程においては、情報提供にかかわる国民の諸活動が公権力によって妨げられないことを意味し、公権力に対し、国民が自己の有する情報を提供するための作為を求めることができることまで意味しないものと解するのが相当である。これを芸術上の表現活動の自由についていえば、芸術家が作品を制作して発表することを公権力によって妨げられることはないが、公権力に対し、芸術家が自己の製作した作品を発表するための作為、たとえば、展覧会での展示、美術館による購入等を求めることはできないといわなければならない。」

すなわち、表現活動を行う者は、表現活動に必要な物、たとえば、絵画の製作において、絵の具や紙の用意を表現の自由について定めた憲法21条1項によって公権力に請求することができないのと同様に、その発表の場を公権力に用意してくれるように求めることまでは保障範囲に含まれないということである。

また、公的助成に関して、アメリカの判例では、「権利と特権の二分論(the
rights-privilege
distinction)」という判例法理が20世紀の初頭まで強い影響力を持っていた。この判例法理によれば、表現活動の禁止のような「規制」ではなく、公的助成のような「給付」の場面では、このような行為は憲法上禁止されているわけでも、要求されているわけでもなく、任意で行われていると考える。よって、表現活動を行う者から見ると、公的助成の恩恵を受けることは「権利」ではなく、「特権」であると捉えられる。したがって、一旦公的助成を行った場合に、それに対して公権力が介入し、取消や変更を行うことは、「特権」を撤回したに過ぎないことになる。すなわち、公権力が公的助成に介入したとしても、それは「権利」ではない「特権」を撤回したのであって、憲法上の権利を侵害したのではないという考え方である\footnote{横大道聡『現代国家における表現の自由―言論市場への国家の積極的関与とその憲法的統制』(弘文堂、2013年)40頁}。この判例法理に完全に依拠したとすれば、繰り返しになるが、公的助成に対する公的機関の介入は、表現の自由を侵害したことにはならず、そのような介入は、無制限に行うことが許容されてしまう。

しかしながら、現状にかんがみると、このような考え方に基づいて、無制限に公的助成に介入することを許容することは望ましくない。19世紀の半ばまでは、国家の役割は国防などの安全保障や警察などに限定されており、国家は国民の権利を不当に侵害しないようにすることが求められていた。これが消極国家や夜警国家と言われるものである。このような国家において、表現活動を行う者は、国家に芸術に関する公的助成をそもそも求めておらず、したがって公的助成を特権と捉えることでこれに対する介入を行うことが権利の侵害とはならないという見解を採ったとしても問題は無いように考えられる。しかし、資本主義経済の発展とともに、貧富の格差が生じてきた19世紀の半ば以降は、その貧富の格差を修正するために、国家がこれまでの安全保障や警察作用に加えて、福祉サービスの提供なども行うようになっていった。それに伴って、国家は学問や芸術に対して公的助成などを行うようになっていった。現在の日本では、国公立の美術博物館及びそれに類似する施設の数が全体に占める割合は高く、表現活動を行う者が公的助成の恩恵を受ける場面が多くなっているといえる。このような状況下で、「権利と特権の二分論」のような考え方を採用し、無制限に公的機関が公的助成に介入することが許容されてしまうと、表現の自由を保障した憲法21条1項は形がい化してしまうのではないだろうか。

したがって、このような公的助成に関して、何らかの憲法的統制を及ぼすべきであり、以下でこれについてアメリカの判例法理を中心に検討していく。

\section{介入についての憲法的統制}

\subsection{パブリック・フォーラム論}

パブリック・フォーラム論はもともとアメリカ合衆国最高裁の判例法理から出発した。パブリック・フォーラム論は、前述した「権利と特権の二分論」を克服するものとして提唱され、表現活動を行う「場」に注目した法理である。アメリカのパブリック・フォーラム論は、国の所有する土地を、主に伝統的パブリック・フォーラム、指定されたパブリック・フォーラム、非パブリック・フォーラムに分けて考える。

伝統的パブリック・フォーラムは、「永きにわたる伝統ないし政府の命令により集会及び討論に捧げられてきた場所\footnote{中林暁生「パブリック・フォーラム」駒村圭吾=鈴木秀美編『表現の自由\ajRoman{1} 状況へ』(尚学社、2011年)206頁}」のことであり、道路や公園がこれに該当する。伝統的パブリック・フォーラムにおける表現活動において、国家がこれを規制する場合には、非常に厳格な要件の下で内容中立的な時・場所・方法による規制しか行うことはできない。

指定されたパブリック・フォーラムは、国家が表現活動の場であるとして指定した場所のことであり、主に公立劇場や公民館などがこれに該当する。指定されたパブリック・フォーラムは、国家がこれを維持する義務はないが、維持している限りは、伝統的パブリック・フォーラムと同様に厳格な要件を満たした場合のみ、そこで行われる表現活動を規制することができる。

非パブリック・フォーラムは、伝統的パブリック・フォーラムと指定されたパブリック・フォーラム以外の国が所有する土地のことであり、そのような場所は伝統的に表現活動が行われているわけではなく、国家がその場所を国民の表現活動のために用意する意思を有しているわけでもないことから、その場所で行われる表現活動は、比較的緩やかな基準で規制することができる。

日本では、泉佐野市民会館事件最高裁判決では初めてこの法理が念頭に置かれ判断されている\footnote{近藤崇晴「判解」最高裁判所判例解説民事篇平成7年度(上)295頁}。泉佐野市民会館事件は、集会を開催するために市民会館の使用許可を申請した者に対して、本件集会の実質的主催者は過激派と言われる団体であり、対立する過激派との衝突などによって市民の安全を確保することができないとして、不許可とする処分の違憲・違法を争った事件である。最高裁は以下のように判示している\footnote{最判平成7年3月7日民集49巻3号687頁}。

「地方自治法244条にいう普通地方公共団体の公の施設として、本件会館のように集会の用に供する施設が設けられている場合、住民は、その施設の設置目的に反しない限りその利用を原則的に認められることになるので、管理者が正当な理由なくその利用を拒否するときは、憲法の保障する集会の自由の不当な制限につながるおそれが生ずることになる。」

この判決では、市民会館は泉佐野市によって、集会の用に供するために設置されていることから指定されたパブリック・フォーラムであるといえる。本判決は、拒否事由が「正当な理由」に当たるかどうかにおいて、利益衡量論及び「明らかに差し迫った危険」基準という二段階で判断しており\footnote{近藤崇晴「判解」最高裁判所判例解説民事篇平成7年度(上)289頁}、アメリカのパブリック・フォーラム論とは細かい点で異なる。この判例以後は、同様の事件では、この判例が引用されていることから、日本においてもパブリック・フォーラム論はある程度の支持は得ているといえる。

その後、日本ではパブリック・フォーラム論は集会の自由が問題となる場面を越えて、公的助成の場面においても、そのような考え方が念頭に置かれているのではないかと思わせる判例が登場している。

船橋市立西図書館事件は、図書館の司書が蔵書を内容に対する反感から廃棄したことに対してその蔵書の著作者が国家賠償請求した事案である。最高裁は以下のように判示している\footnote{最判平成17年7月14日民集59巻6号1569頁}。

「公立図書館が,上記のとおり,住民に図書館資料を提供するための公的な場であるということは,そこで閲覧に供された図書の著作者にとって,その思想,意見等を公衆に伝達する公的な場でもあるということができる。したがって,公立図書館の図書館職員が閲覧に供されている図書を著作者の思想や信条を理由とするなど不公正な取扱いによって廃棄することは,当該著作者が著作物によって
その思想,意見等を公衆に伝達する利益を不当に損なうものといわなければならない。そして,著作者の思想の自由,表現の自由が憲法により保障された基本的人権であることにもかんがみると,公立図書館において,その著作物が閲覧に供されている著作者が有する上記利益は,法的保護に値する人格的利益である(以下省略)」

本判決では、著作者はその著作物を図書館で所蔵し、公衆に閲覧してもらうように求める権利は有していないが、図書館が住民だけでなく、そこに収蔵されている図書の著作者にとっても「公的な場」であることから、司書にはそこに収蔵されている図書を公正に取り扱う義務が生じると判示している。この判例は、図書館を住民に図書館資料を提供するための「公的な場」であると同時に、図書の著作者にとってもその思想、意見を伝達する「公的な場」であるとしていることから、住民の知る権利と図書の著作者の表現活動の自由に供される限定的なパブリック・フォーラムであると考えることができる。さらに、この論理は公的助成とパブリック・フォーラム論の相性の悪さを克服するものであると考えることができる。そもそも、パブリック・フォーラム論においては、伝統的パブリック・フォーラムや指定されたパブリック・フォーラムでの表現活動に対する公的機関の規制は、時・場所・方法・に関する内容中立的な規制しか行うことができないという厳格なものであった。しかし、公的助成のような給付の場面においては、公的機関の財源や場所に限界があることから、給付をするかしないかの判断においては、内容に立ち入ってその価値に基づいて判断せざるを得ないとも考えられてきた。しかし、本判決においては、図書館を限定的なパブリック・フォーラムと捉えたとしても、その図書の公正な取り扱いを専門職である司書の義務としたことで、その蔵書の内容に関する判断を図書館という公的機関から自律した司書に委ね、公的機関には、パブリック・フォーラム論の要件の下でしか規制することができないと考えることができる。このように考えることで、パブリック・フォーラムにおける公的機関による内容に基づく介入ではなく、専門職である司書の義務に関する問題と捉えることができる。

しかし、この判決はあくまで、公立図書館において閲覧に供された著作物が不公正な取扱いによって廃棄されたという限定的な事案であり、公的助成の事案に射程を広げることはできないという考え方も存在する\footnote{中林暁生「判批」長谷部恭男ほか編『憲法判例百選\ajRoman{1}〔第7版』(有斐閣、2013年)153頁}。

\subsection{政府言論}

政府言論とは、公的機関の思想の自由市場における役割に着目したアメリカの判例法理である。政府言論とは、政府が思想の自由市場において、国民の表現活動を規制する「検閲者」として行動するのではなく、政府自らが表現活動を行う「話し手」となるということである\footnote{巻美矢紀「自由と給付」大石眞=石川健治『憲法の争点』(有斐閣、2008年)84頁}。この政府言論には、直接的な思想の自由市場への介入と、間接的な思想の自由市場への介入と二つに分けることができる。直接的な思想の自由市場への介入とは政府の広報のような場合であり、このような場合には政府の見解であることが外見からはっきりとわかるため問題とはならない。間接的な思想の自由市場への介入とは国家が私人を媒介にする場合であり、芸術の公的助成の場合などがこれに当たり、このような場合には、外見では政府の見解であることがわからないことが問題となる。政府言論の問題点はさらにある。それは思想の自由市場における影響力の差である。政府言論は私人の表現活動と比較して圧倒的な潜勢力を有しているとされる。その原因は主に3つあると阪口正二郎は分析している\footnote{阪口正二郎「芸術に対する国家の財政援助と表現の自由」法律時報74巻1号(2002年)32頁}。1つ目は、経済的資源の差である。国家は国会で決められた予算に基づいて活動しているため無制限とは言えないが、私人と比較した場合には、やはり圧倒的な差が生じていると考えられる。2つ目は、情報の保有の差である。私人は国家と比較して情報の保有量が少ないといえる。3つ目は、政府の言論には、私人の表現活動には備わっていない正統性があるということである。国家が言っていることは私人の言っていることよりも信用性が高いとされてしまうことであり、これによって思想の自由市場では、私人の思想が国家の思想によって淘汰されてしまい、思想の自由市場にゆがみが生じてしまうということである。このような問題点から、政府言論は一律に規制するべきだという見解も生じるかもしれないが、問題はそう単純ではない。政府言論は民主主義社会においては不可欠なものであるからである。民主主義においては、民衆は国家の行動を監視し評価することで健全に運営されている。このような監視や評価においては、その前提として、政府の行動やある問題についての政府の立場について正しい情報が必要である。そのため、アメリカの判例法理としての政府言論の法理では、政府の言論に関しては、国家はその内容が正しいものであるために必要な是正を内容に立ち入って行うことが許容される。

しかし、前述のように政府言論には私人のそれとは異なる影響力を有していることから、無制限に内容に立ち入った規制を許すことは本来好ましくない。そこで、政府言論であっても憲法による統制を行う必要である。「囚われの聴衆」に陥りやすい場所での政府言論に対して、多様な言論へのアクセスを保障したり、政府が「腹話術師」的に語る政府の匿名性を禁止したりすべきである\footnote{前掲注9}。また、政府言論ではない場合には、「給付」の場面では、パブリック・フォーラム論などの他の憲法的統制を行う必要がある。

政府言論で重要となるのは、どこまでが政府言論であり、どこから政府言論ではないという線引きが難しいということである。これは特に、思想の自由市場に間接的に介入する場合に問題となる。政府言論に当たるかどうかについては、以下のアメリカの判例で用いられている基準が参考になる\footnote{蟻川恒正「政府の言論の法理---教科書検定を素材として」駒村圭吾=鈴木秀美編『表現の自由\ajRoman{1} 状況へ』(尚学社、2011年)434頁}。

\UTF{2460}当該言論がそのもとで行われる政府プログラムの主要な目的(central
purpose)

\UTF{2461}当該言論の内容に対する政府の編集的コントロール(editorial
control)の程度

\UTF{2462}実際に当該言論を行う者(literal speaker)が政府であるかということ

\UTF{2463}当該言論の内容についての最終的な責任(ultimate responsibility)の所在

\UTF{2461}と\UTF{2463}の要素は政府言論であるというために有利な要素であり、\UTF{2460}と\UTF{2462}の要素は政府言論であるというために不利な要素である。具体的に当てはめてみる。政府の広報や首相・官房長官の記者会見では、\UTF{2460}から\UTF{2463}すべてを満たすために政府言論であると認められる。一方、あいちトリエンナーレは、その目的が、世界の文化芸術の発展への貢献、文化芸術の日常生活への浸透、地域の魅力の向上\footnote{ あいちトリエンナーレとは

  (
  \href{https://aichitriennale.jp/about/index.html\%20、2020}{{https://aichitriennale.jp/about/index.html 、2020}}年8月25日最終閲覧)}であることから、政府の見解を明らかにすることが目的ではないとわかる。また、あいちトリエンナーレは作品の選択等は芸術部門である芸術監督とキュレーターが行い、芸術部門には会長が知事である運営会議に対して自律性を有していることから、政府の編集的コントロールも弱い。また、当然実際に当該言論を行う者は、作品の製作者であり、その内容についての最終的な責任もその製作者にある。したがって、あいちトリエンナーレは政府言論ということはできない。

\subsection{憲法25条による統制}

近時、日本で提唱されている学説は、憲法25条の「文化的な最低限度」の基準で判断するということである\footnote{前掲注9}。憲法25条で規定されている生存権は朝日訴訟、堀木訴訟など主に経済的な場面で論じられてきた。しかし、公的助成の憲法的統制において憲法25条を用いようとする学者は、25条は経済的場面に限らず、「文化」に独自の意味を持たせようと考えている。「文化的な最低限度」の解釈については、複数の立場が存在している。政治的リベラリズムの立場から、パブリック・フォーラムの維持・供用を、人間の共同性の確保と実質のある〈生〉の充実を重視する立場からすれば、それぞれの価値理念の現実的基盤をなす文化共同体に対する便宜供与を要請するとされている\footnote{同上}。

しかし、「文化的な最低限度」の解釈は上記のようにさまざまであり、どこまでが最低限度であるかを決めることは困難である。また、生存権はもともと立法裁量が広い分野であることから、25条を公的助成への介入の統制の基準にしたとしても実質的には機能しないのではないかということが考えられる。

\section{個々の事案の検討}

\subsection{富山県立近代美術館事件}

富山県立近代美術館事件は、その美術館で所蔵している大浦信行氏製作の『遠近を抱えて』という作品に対して、その作品によって天皇を侮辱しているとして県議会議員が非難攻撃したことが始まりであり、それに呼応して同調する団体が本作品及び掲載されている図録の廃棄などを求めて抗議活動が激化し、最終的には本作品を売却し、本件図録を焼却した。これに対して、原告は本作品の特別観覧を拒否され、本件図録の閲覧を拒否されたことから、知る自由が侵害されたとして、損害賠償請求等を求めて訴訟を提起した事案である。地裁判決では以下のように判示している\footnote{富山地判平成10年12月16日判時1699号120頁}。

「富山県立近代美術館条例は、\ldots{}\ldots{}特別観覧制度を定めているが、これは、県立美術館に収蔵されている作品についての知る権利を具体化する趣旨のものである」「正当な理由なく特別観覧許可申請を不許可とするときは、憲法の保障する知る権利を不当に制限することになると解すべきである」

本判決は、「表現の不自由展・その後」とは異なり、観覧者による請求も行われた事件であり、観覧者の知る権利に基づいて請求の判断がなされている。また、本判決は泉佐野市民会館事件判決とは場面は異なるが、同様の論理を組み立てている。知る権利は憲法21条1項によって保障されているが、抽象的権利であることから裁判規範性を有するには、知る権利を具体化する規定が必要である。本判決では、特別観覧制度によって具体化されたとして、正当な理由なく拒否した場合には、知る権利を不当に侵害する恐れがあるとされた。そして、本件拒否事由が正当な理由に当たるか否かの判断においては、利益衡量論を展開し、その判断においては「明らかに差し迫った危険」基準によって行った。

一方で、控訴審判決では以下のように判示している\footnote{名古屋高金沢支判平成12年2月16日判時1726号111頁}。

「県立美術館についての右の美術品の特別観覧に係る条例等の規定は、美術館の開設趣旨やその規定の仕方、内容に照らしても、第一審原告らが主張するように憲法二十一条が保障する表現の自由あるいはそれを担保するための「知る権利」を具体化する趣旨の規定とまで解することは困難である」

本判決は、上記のように特別観覧制度に係る規定は知る権利を具体化する規定ではないと判断したことから、憲法問題とはせずに、地方自治法の解釈問題にとどめて判断している。

第一審判決と控訴審判決の違いは、やはり特別観覧制度に係る規定が知る権利を具体化しているかどうかによって、憲法問題になるか、法律問題にとどまるかが決まっているように思われる。これに対して、知る権利の具体化である情報公開制度と本件の特別観覧制度を比較して、控訴審判決の判断は正しいと評している研究者もいる\footnote{中林暁生「\ajRoman{3}判例を読む」法学教室343号28頁}が、公立の美術館は作品の製作者にとって重要な思想・信条を表現し、伝達する場であることを重視すると、第一審判決の方が妥当なのではないかと考えられる。

\subsection{9条俳句事件}

9条俳句事件とは、「梅雨空に 『九条守れ』の 女性デモ」という俳句が、公民館及び公民館だよりに掲載されなかったことに対して、掲載と損害賠償請求を求めた事件である。控訴審判決では以下のように判示している\footnote{東京高判平成30年5月18日判時2395号47頁}。

「本件合意に基づき本件俳句を本件たよりに掲載することを求めるXの請求は理由がない」「公民館は、住民の教養の向上、生活文化の振興、社会福祉に寄与すること等を目的とする公的な場」「公民館の職員は、\ldots{}\ldots{}住民の公民館の利用を通じた社会教育活動の実現につき、これを公正に取り扱う職務上の義務を負う」「そして、公民館の職員が、住民の公民館の利用を通じた社会教育活動の一環としてなされた学習成果の発表行為につき、その思想、信条を理由に、他の住民と比較して不公正な取扱いをしたときは、その学習成果を発表した住民の思想の自由、表現の自由が憲法上保障された基本的人権であり、最大限尊重されるべきものであることからすると、当該住民の人格的利益を侵害するものとして国家賠償法上違法となる」

本件判決は、損害賠償請求について船橋市西図書館事件最高裁判決の枠組みと同じように判断している。公民館を「公的な場」としたうえで、そこで働く職員に職務上の義務を課し、それに反しているかどうかを判断している。そして、職員が行った不公正な取扱いが正当な理由に基づくかについて検討している。本件では、俳句の公民館及び公民館だよりへの掲載は、その俳句に係れている内容について公共団体が賛同していると取られてしまうというものであったが、判決ではそのような場合であっても、ただ掲載しているだけであり、賛同を意味しているわけではないことから、正当な理由には当てはまらないと判示している。

\subsection{「表現の不自由展・その後」中止決定}

「表現の不自由展・その後」の愛知県知事による中止決定は、船橋市西図書館事件を参考に考えることができる。本件では前述のとおり目的が表現の場を提供することであるため政府言論ということはできない。また、すべての国民が自由に表現活動を行うことができるということも想定できていないため、従来のパブリック・フォーラムということもできない。しかし、船橋市西図書館事件において、収蔵されている図書の著作者にとって自らの思想・信条を公衆に伝達する公的な場であるのと同様に、本件展示会で作品の展示が決まっていた製作者にとってはその展示会が自らの思想・信条を観覧者に伝達する公的な場ということができ、そういう意味では製作者にとってはパブリック・フォーラムということができるのではないだろうか。そのうえで、一旦展示が決まっていることから、正当な理由なく展示を中止した場合には違法となると構成することができないだろうか。そのうえで、本件中止決定に至った事由が正当な理由にあたるかを検討する。本件では、電話・ファックス・メールによる抗議・脅迫、また脅迫などにより逮捕者が出ていることからすれば安全に展覧会を運営することはできないという事由は正当な理由に当たるということができるであろう。

\section{さいごに}

このレポートでは、公的助成への国家の介入に対してどのような憲法的統制を行うことができるのかについて、アメリカの判例法理や日本の最高裁判例、下級審判例などを取り上げて分析してきた。公的助成の場合、直ちに表現の自由など憲法の規定によって統制がなされるというわけではなく、その表現活動が政府言論といえるか、パブリック・フォーラム論の射程が及ぶかどうかなどによって、さまざまな統制が考えられる。また、日本ではまだ導入されていないがイギリスなどでは導入されている、アーツカウンシル制度などもある。この制度にも問題点はあるだろうが、個々の制度の問題点を踏まえつつ最善の制度設計を検討していくことで、より健全な国家による芸術支援がなされるのではないだろうか。

\section{参考文献}
\noindent
・蟻川恒正「政府の言論の法理---教科書検定を素材として」駒村圭吾=鈴木秀美編『表現の自由\ajRoman{1} 状況へ』(尚学社、2011年)417頁\\
・井出明「アートマネジメントから観た``表現の自由''再考―近代と脱近代の相克」法学セミナ―786号(2020年)17頁\\
・太下義之「文化専門職と表現の自由」法学セミナ―786号(2020年)55頁\\
・駒村圭吾「憲法問題としての芸術―表現の自由保障``生誕100年''に寄せて」法学セミナ―786号(2020年)10頁\\
・近藤崇晴「判解」最高裁判所判例解説民事篇平成7年度(上)282頁\\
・阪口正二郎「芸術に対する国家の財政援助と表現の自由」法律時報74巻1号(2002年)30頁\\
・中島徹「憲法で見る「表現の不自由展・その後」-契約は憲法を超えるか」法学セミナ―786号(2020年)34頁\\
・中林暁生「\ajRoman{3}判例を読む」法学教室343号22頁\\
・中林暁生「パブリック・フォーラム」駒村圭吾=鈴木秀美編『表現の自由\ajRoman{1} 状況へ』(尚学社、2011年)197頁\\
・中林暁生「判批」長谷部恭男ほか編『憲法判例百選\ajRoman{1}〔第7版〕』(有斐閣、2019年)153頁\\
・巻美矢紀「自由と給付」大石眞=石川健治『憲法の争点』(有斐閣、2008年)84頁\\
・山邨俊英「あいちトリエンナーレ2019問題の事案紹介」法学セミナ―786号(2020年)30頁\\
・横大道聡『現代国家における表現の自由―言論市場への国家の積極的関与とその憲法的統制』(弘文堂、2013年)\\
・横大道聡「表現の自由の現代的論点―〈表現の場〉の〈設定ルール〉について」法学セミナ―786号(2020年)24頁\\
・横大道聡「文化への助成と表現の自由」駒村圭吾=鈴木秀美編『表現の自由\ajRoman{1} 状況へ』(尚学社、2011年)352頁\\
・あいちトリエンナーレとは
(\href{https://aichitriennale.jp/about/index.html、2020年8月25}{{https://aichitriennale.jp/about/index.html}、2020年8月25}日最終閲覧)\\
\end{document}