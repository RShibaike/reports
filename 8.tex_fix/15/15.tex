\documentclass[twoside]{jsarticle}
\usepackage[dvipdfmx,hidelinks]{hyperref}
\usepackage{pxjahyper}
\usepackage[jis2004]{otf}
\usepackage[dvipdfmx]{graphicx}
\usepackage{url}
\setcounter{page}{1}
\usepackage{fancyhdr}
\begin{document}
\title{自民党の日本国憲法改正案における国家緊急権の論点}
\author{前山智耶}
\date{}
\maketitle
\tableofcontents
\clearpage
\pagestyle{fancy}
\lhead[自民党の日本国憲法改正案における国家緊急権の論点]{}
\rhead[]{\leftmark}
\section{はじめに}

新型コロナウイルスの感染拡大によって緊急事態宣言が発令され、新型インフルエンザ

等特別措置法に基づいて都道府県知事が外出自粛要請や休業要請が出された中、憲法改正

の議論を進めようとする動きが見られた。この状況下で特に注目されたのが自由民主党の日本国憲法改正草案第9章に規定された国家緊急権である。時事通信社の5月の世論調査では緊急時代宣言の創設について、賛成が41.9%、反対が54.9%であった\footnote{時事ドットコムニュース「緊急事態条項「反対」5割超 時事世論調査」2020年6月6日(最終閲覧日2020年8月23日)\\\href{https://www.jiji.com/jc/article?k=2020060600325\&g=cov}{{https://www.jiji.com/jc/article?k=2020060600325\&g=cov}}。}。賛成意見よりも反対意見の方が多いものの、その差は少なく、拮抗しているといえる。

またウイルスの感染拡大によって医療体制の逼迫が懸念される地方の首長からは住民の権利に対してより強い制限を加えられるように法改正を要求する声があがっている。

感染症の拡大は人間の生命を脅かす危険があるため、それを防止するために国民の権利を最小限の範囲で制限することはやむを得ないとする議論は成り立ちうると考えられる。しかし、国民の権利を保障するために国家を抑制するという役割を果たす憲法に、国民の権利を制限できる国家の権利、すなわち国家緊急権の規定をおくことは憲法の性質を考えると違和感を覚えざるを得ない。また国民が憲法に国家緊急権の規定を創設することを求めるということは、言い換えれば国民が国家に対して積極的に自らの権利を制限するように求めることであり、そのような光景は極めて異様なものである。

それでも憲法に国家緊急権の規定を設ける必要性があるという意見が一定数あることを踏まえ、以下において2012年4月に決定された自由民主党の日本国憲法改正案(以下、「改正案」とする。)における「第9章・緊急事態」に規定された各規定についての必要性・合理性や諸論点を整理、検討し、憲法に国家緊急権を規定することの是非について考察する。

\section{国家緊急権の定義}

国家緊急権とは、「戦争・内乱・恐慌ないし大規模な自然災害など、平時の統治機構をもってしては対応できない非常事態において、国家権力が、国家の存立を維持するために、立憲的な憲法秩序(人権保障と権力分立)を一時停止して、非常措置をとる権限のこと」をいう\footnote{芦部信喜『憲法学\ajRoman{1}憲法総論』(有斐閣、1992年)66頁。}。合憲的な法律の侵害または効力停止を国家緊急権の効果に含めることも考えられるが、改正案の条文に対象となるような文言がないため、範囲の対象外とする。

\section{明治憲法と日本国憲法における国家緊急権}

\subsection{明治憲法下}

明治憲法には、緊急事態に対処するための規定として、8条(緊急命令)、14条(戒厳令)、31条(非常大権)、70条(緊急財政処分)といった規定が置かれていた。しかし、これらの規定は明治憲法下において少なからず濫用されてきた。例えば、1928年に政府は罰則として死刑を加えるために治安維持法を改正しようとしたが、議会の反対にあったために緊急勅令を発令して法改正を図り、後に議会の承認を受けた。また緊急事態において国の統治作用の一部を軍部に移行させる戒厳令は、条文の規定する戒厳の要件を満たしていない場合であっても治安維持を理由として緊急命令の規定を通じて戒厳令を適用する行政戒厳という形で関東大震災や2.26事件でも発令され、軍部を暴走させる契機となった。臣民の権利義務を全面的に停止する非常大権は直接発動されることはなかったものの、これに近い性格の国家総動員法が成立する際に非常大権を定めた明治憲法の精神が参照されたとされている。

\subsection{日本国憲法下}

ア.緒論

明治憲法が上記のように国家緊急権を実定法化していたのに対して、日本国憲法にはそのような規定は見られない。日本国憲法制定議会において、金森徳次郎国務大臣がその理由について述べている。

「民主政治を徹底させて国民の権利を十分擁護致します為には、左様な場合の政府の一存において行いまする処置は、極力これを防止しなければならぬのであります。言葉を非常と云うことに籍りて、その大いなる途を残して置きますなら、どんなに精綴なる憲法を定めましでも、口実をそこに入れて又破壊せらるる虞れ絶無とは断言し難いと思います。闘って此の憲法は左様な非常なる特例を以て―謂わば行政権の自由判断の余地をできるだけ少なくするように考えた訳であります。随って、特殊の必要が起こりますれば、議会を召集してこれに応ずる措置をする。又衆議院が解散後であって処置の出来ない時は、参議院の緊急集会を促して暫定の処置をする。同時に他の一面に於て、実際の特殊の場合に応ずる具体的な必要な規定は、平素から濫用の虞なき姿に於て準備するように規定を完備して置くことが適当だろうと思う訳であります。」\footnote{清水伸編『逐条日本国憲法審議録第二巻(有斐閥、1962年)221頁。}

また、学説では日本国憲法が国家緊急権を規定していない意味について意見が分かれている。以下において、4つの学説を個別に検討し、この章のまとめとして最後に私見を述べる。

イ.否定説

この説は日本国憲法が国家緊急権について規定していないのは、明治憲法の緊急権が君権絶対のイデオロギーと不可分であったという歴史的事情に対する反省、交戦力と戦力保持を禁止した徹底的な平和主義といった観点から、国家緊急権を「あえて置かなかった」と考え、国家緊急権を否認しているからであると考える説である\footnote{小林直樹『国家緊急権』(学陽書房、1979年)181頁。}。緊急事態下においても、国家は憲法の定める(平時の)権限の範囲内で活動しなければならない。

ウ.欠缺説

この説は、「日本国憲法に非常事態に関する規定を置いていないことは、むしろ法の欠陥である」とする説である\footnote{大西芳雄『憲法の基礎理論』(有斐閣、1975年)223頁。}。この説では平時の法制度の枠組みでは処理しきれない緊急事態が発生した際に実定法を無視し、超越した国家緊急権を認めることは、国家権力の濫用につながるおそれがあり、だからこそあらかじめ緊急事態に対応する規定を憲法に定めておくべきであり、またそれが立憲主義の要請にかなうとする。

エ.容認説

この説は、不文の国家緊急権を否定する上記の2説とは対照的に、日本国憲法の国家緊急権についての沈黙は、不文の国家緊急権を排するものではなく、緊急事態においては、憲法の存続を図るために国家が非常措置を講ずることを「不文の法理として肯定しなければならない」とする説である。ただし、非常措置は単に国家の存立のためではなく、個人の自由と権利の保障を核とする憲法秩序の維持ないし回復をはかる目的でされるものでなくてはならず(目的の明確性の原則)、必要最小限度で行使され、濫用防止のために責任の所在が明確でなければならない。\footnote{佐藤幸治『日本国憲法論』(成文堂、2011年)50
  頁。}

オ.事実説

この説は、緊急事態において国家が、実定法上の規定がなくても国家固有の自然権としての国家緊急権を行使できるという考え方を、緊急権の発動を事実上国家の恣意に委ねることを容認するものであるとして批判する。その上で緊急事態において実定法上の根拠無く行使される非常措置は、「法の問題ではなく、事実ないし政治の問題である」とする\footnote{芦部・前掲注2)66頁。}。

\subsection{小括}

ア.否定説について

日本国憲法制定会議における金森国務大臣の発言を元に考えると、日本国憲法が国家緊急権の規定を置いていないのは、明治憲法下での事例に対する反省からあらゆる状況でも民主主義を徹底させるという意思を明確にしたためである、すなわち否定説の考え方をとったと理解するのが自然である。しかし、明治憲法下での日本と同じくワイマール憲法下で独裁体制が敷かれて大規模な人権侵害が行われたドイツにおいて、戦後極めて詳細に規定された国家緊急権の規定が整備され、現在その行使には抑制的であるように、歴史的事情に対する反省の有無や強度は国家緊急権の規定が設けられているか否かによって必ずしも決定されるものではない。また交戦力と戦力不保持といっても放棄されたのは侵略戦争であり、自衛のための武力行使は認められ、さらに2015年には安全保障関連法が成立し、日本と密接な関係にある他国が武力攻撃された場合に日本が直接武力攻撃を受けていなくても条件を満たせば自衛隊が武力行使できるようになったことを考慮すると、恒久的平和主義といってもその解釈はある程度流動的である。このような状況下で実定法の枠組で対処できないような緊急事態が発生した場合には、国家の活動は停止することになる。人々の生命・身体・財産などの基本的権利を守るために創設されたはずの国家が、その役割を放棄することになると、国家の存在理由そのものがなくなってしまうことになる。以上より、日本国憲法は国家緊急権の規定を排除していないと考えるのが、国家の安全保障を考えるに当たって現実的な判断であると思われる。

イ.欠缺説について

次に憲法上に国家緊急権の存在や行使の要件を明記する必要があるか否かということが問題となる。欠缺説をとる場合には、実定法に基づいた規定によって国家緊急権が行使されることによって濫用の防止が規定されるものの、どのような規定の仕方をするのかが問題となる。ドイツ基本法のように詳細な場面設定や要件を定めるとき、逆に定められなかった事態に対しては、国家の活動は停止されることになる。「緊急事態」というのであるから、事前に想定することができないような事態にも対処できるような制度の構築が必要である。だからといって包括的な要件を定めた国家緊急権を置くと、国家による恣意的な運用が可能になるおそれがあり、また歴史上国家緊急権の濫用が行われてきたのは上述の通りである。またそのような場合においては憲法上に国家緊急権の規定があることが国民の権利を制限するための大義名分となり、逆に国家緊急権の積極的な運用を促してしまう危険がある。サイバーや宇宙空間など、科学技術の発展によって安全保障の舞台が移り変わり、また他国に対する攻撃の方法が匿名性・非対称性・越境性を帯びたものになってきている、すなわち平時と有事の違いが不明確になってきていることを踏まえれば、この危険は一層重大なものになる。以上のような理由から憲法上に国家緊急権の明示の規定を設けるのは、技術的な問題、濫用の危険性の観点より、控えるべきであると考える。

ウ.事実説について

緊急事態における国家による非常措置は法の問題ではなく、事実ないし政治の問題であるとすると、非常事態において、国家の実定法上の根拠に基づかない権力が発動されることを容認することになる。これはむしろ政府に対して無拘束の国家緊急権を認めることになりかねない\footnote{工藤達朗「国家緊急権と抵抗権」中央ロージャーナル第16巻第1号(2019年)106頁。}。また非常措置を政治の問題とするのであれば、国会による責任の追及がされることになるが、治安維持法の改正が国会の反対を受けながら緊急命令によって施行され、結局国会が事後承認したことを考えると、責任追及できる範囲は限られていると解さざるを得ない。

エ.容認説とその規制

以上より、緊急事態に対する国家の対処の必要性、国家緊急権の詳細な規定を設けることの不可能性を踏まえると、容認説をとるのが現実的であると考える。容認説は国家緊急権を不文の法理(自然権)として認めることになる。しかし、国家は自然人ではないし、国家緊急権の行使によって制限されるのは国民の権利であることを考えると、自然人の権利と同等に扱うことは適切ではなく、国家緊急権も一定の制約を受けると解することができる。国家による国家緊急権の濫用を防止するために、目的の明確性の原則に基づき、必要最低限度の範囲で、かつ責任の所在を明確にしつつ、不文の法理の中にも一定の方向性を示すようなルールを整備することが妥当である。

\section{外国法制度における国家緊急権}

憲法に国家緊急権を導入すべきとする立場からは、「諸外国の憲法には国家緊急権が規定されているのだから、日本国憲法に国家緊急権の規定がないのはおかしい」という主張が展開される。憲法における国家緊急権の規定の方法としては、\UTF{2460}国家緊急権を発動する要件・手続・効果等を比較的詳細に定める方式と、\UTF{2461}特定の国家機関に包括的な権限を授権する方式がある\footnote{工藤・注8)102頁。}。

\subsection{国家緊急権を発動する要件・手続・効果等を比較的詳細に定める方式}

\UTF{2460}の方式の国家緊急権を規定する憲法として代表的なのが、ドイツ連邦共和国基本法(以下、ドイツ基本法とする。)である。ドイツ基本法では、緊急事態を対外的緊急事態と対内的緊急事態に分けた上で、前者をさらに防衛事態、緊迫事態、同盟事態に分類し、後者を憲法上の憲法事態と災害事態に分類する。それぞれの事態は定義づけされており、それぞれの事態に対応した連邦や州の制限、さらには基本権の制限を規定する。ドイツ基本法の定める国家緊急権がこのように規定されたのは、包括的な国家緊急権を定めたワイマール憲法下での共和国崩壊とナチス独裁体制に対する反省があるからである。

\subsection{特定の国家機関に包括的な権限を授権する方式}

\UTF{2460}の方式とは異なって緊急事態について詳細な要件を定めずに包括的に規定するのが\UTF{2461}の方式である。ワイマール憲法では48条2項において、「ドイツ国内において公共の安全及び秩序に著しい障害が生じ、又はそのおそれがあるとき」は、大統領がその状態から回復させるために必要な措置をとることができ、必要がある場合には、武力兵力を用いて介入することができ、一時的に国民の諸権利を停止することができるとされた。緊急事態に認定されうる具体的な場面は法律によって規定されることとされたが、実際には法律の制定はなされずに、大統領の判断で国家緊急権の行使がなされた。ヒトラーは1933年に「民族と国家を保護するための大統領令」を発し、共産主義的な国家公安を害する暴力行為を防止するために、人身の自由や表現の自由といった諸権利を制限できる旨の規定を定めた。共産党を弾圧した後、政権を獲得したナチスは「民族及び国家の危難を除去するための法律(授権法)」を制定し、独裁体制を樹立した。

フランス第5共和制憲法16条にも国家緊急権が規定されている。それによれば、「共和国の制度、国の独立、その領土の一体性あるいは国際協約の履行が重大かつ直接的に脅かされ、かつ、憲法上の公権力の適正な運営が中断されるとき」に大統領が必要とされる措置をとるとされる。この国家緊急権は1961年の将軍による反乱の際に発令された。反乱自体はすぐに鎮圧されたものの、国家緊急権の規定はその後約5ヶ月にわたって適用され、その間に必要最低限度を上回る人権の制約があったことが指摘されている。

\subsection{小括}

以上の例より、諸外国の規定を比較した場合における国家緊急権については以下の2点が指摘できる。

一つ目は、国家緊急権と言っても、各国が規定する内容は様々である。上記の分類のように、緊急事態の定義の緻密さには違いがあり、またドイツ基本法は緊急事態に対する対処の際にも議会の権限が強く残されている一方でフランス第5共和制憲法では国会は当然に閉会する。そもそもすべての国が国家緊急権の規定を憲法に設けているわけではない。そうすると、権利の内容が明確に定まっていない諸外国のさまざまな国家緊急権を一様にまとめあげ、諸外国の憲法に規定されているという理由だけで我が国の憲法に規定しようとする主張は無理があるように思われる。

二つ目に国家緊急権は、歴史的に見て濫用される危険性があるということである。国家緊急権を認めるとするならば、それを制約する方法も同時に考える必要がある。

\section{改正案における国家緊急権に関する論点}

以上の議論を踏まえて改正案における国家緊急権の規定に関する論点を整理し、検討する。

\subsection{対象となる緊急事態と既存の法律との関係}

改正案の国家緊急権が規定する緊急事態は、「我が国に対する外部からの武力攻撃」、「内乱などによる社会秩序の混乱」、「地震などによる大規模な自然災害」といったとものが上げられている(改正草案98条1項)。一方で法律のレベルではこれらの緊急事態に対して対処する規定が置かれている。以下ではこれらの緊急事態とそれに対応する法律を見ていくことにする。

ア.「我が国に対する外部からの武力攻撃」

前提として、恒久的平和主義を規定する日本国憲法の趣旨を踏まえると、いかなる紛争も平和的手段で解決が図られるべきである。その上で、他国からの武力攻撃に対しては自衛隊法や武力攻撃事態等における国民の保護のための措置に関する法律(国民保護法)をはじめとして様々な法制度が整備されている。内閣総理大臣は、我が国に対する外部からの武力攻撃が発生した事態又は我が国に対する外部からの武力攻撃が発生する明白な危険が切迫していると認められるに至った場合、我が国と密接な関係にある他国に対する武力攻撃が発生し、これにより我が国の存立が脅かされ、国民の生命、自由及び幸福追求の権利が根底から覆される明白な危険がある場合には、自衛隊の全部又は一部の出動を命ずることができる(自衛隊法76条)。武力攻撃時において、国民は、国民の保護のための措置の実施に関し協力を要請されたときは、必要な協力をするよう努めるものとする(国民保護法4条1項)。

イ.「内乱などによる社会秩序の混乱」

内乱など、国内における緊急事態においては、警察法がその対処の方法について規定している。すなわち、内閣総理大臣は、騒乱などの緊急事態に際して、治安の維持のため特に必要があると認めるときは、国家公安委員会の勧告に基き、全国又は一部の区域について緊急事態の布告を発することができる(警察法71条1項)。そして緊急事態の布告が発せられたときは、本章の定めるところに従い、一時的に警察を統制する(同法72条前段)。また警察だけでは対処しきれないような緊急事態においては、自衛隊法に基づく治安出動や、内閣総理大臣の海上保安庁の指揮権が認められている。

ウ.「地震等による大規模な自然災害」

大規模な自然災害については、新型コロナウイルス感染拡大防止のための外出自粛要請の根拠規定にもなった新型インフルエンザ特別措置法をはじめとして、災害対策基本法、警察法などの法律が整備されている。例えば、災害対策基本法の規定によれば、非常災害が発生し、かつ、当該災害が国の経済及び公共の福祉に重大な影響を及ぼすべき異常かつ激甚なものである場合に、内閣総理大臣は、災害緊急事態の布告を発することができる(災害対策基本法105条1項)。また内閣は、国の経済秩序を維持する等の緊急の必要がある場合において、国会が閉会中又は衆議院が解散中であり、かつ、臨時会の招集を決定し、又は参議院の緊急集会を求めてその措置を待ついとまがないときは、緊急措置として政令を制定することができる(同法109条)。

エ.小括

以上を踏まえると、この章の最初に述べたように、法律レベルでは緊急事態に対処するために国家(中央政府)に権限を集中させる制度が整備されているといえる。したがって法律に定められている緊急事態に関しては、憲法に国家緊急権の規定がないからといって国家が何もすることができないと主張することは無理があるように思われる。日本国憲法が国家に不文の国家緊急権を認めることを排していないとする容認説をとるならば国家緊急権は、これらの法律が規定する緊急事態に関しては、法制度に沿った形で行使される必要があると考えることができる。

また、東日本大震災において政府が初動時に迅速に対応出来なかったことを理由に緊急事態条項(国家緊急権)を憲法上創設すべきとの見解があるが、その原因は高度に整備さ

れた法制度があるにもかかわらず、平時から災害に備えた事前の準備がほとんどなされていなかったことによる\footnote{日本弁護士会「日本国憲法に緊急事態条項(国家緊急権)を創設することに反対する意見書」(2017
  年)10頁。}。すなわち、災害対策においては平時の行政の防災・減災面での役割が重要となってくるのであり、また災害発生時においては現場の実情を一番把握し、対応策をとるのは中央政府ではなく地方公共団体であることを考えると、中央政府に権限を集中させるには疑問である。

\subsection{解散権の制限及び議員の任期等の特例}

改正案では、「緊急事態の宣言が発せられた場合においては、法律の定めるところにより、その宣言が効力を有する期間、衆議院は解散されないものとし、両議院の議員の任期及びその選挙期日の特例を設けることができる」としている(改正案99条4項)。以下において、任期及び選挙期日の延長について各場合に分けて検討する。

ア.参議院の場合

参議院議員の任期は、6年とし、3年ごとに議員の半数を改選する」(憲法46条)のであるから、緊急事態において参議院議員の任期が満了したとしても半数は残っていることになり、参議院の定足数が3分の1であることを踏まえると(憲法56条)、任期が満了した議員なしに議決をとることは可能であり、任期を延長する必要性はない。

イ.衆議院解散の場合

衆議院が解散されたとしても、「内閣は、国に緊急の必要があるときは、参議院の緊急集会を求めることができる」(憲法54条2項)のであり、上述のように参議院がどのような状況においても機能しないことはないから、参議院の緊急集会において緊急事態への対処が可能である。

なお、「衆議院が解散されたときは、解散の日から40日以内に、衆議院議員の総選挙を行い、その選挙の日から30日以内に、国会を召集しなければならない」(憲法54条1項)とされているところ、衆議院が解散された後に緊急事態が発生し、40日以内に総選挙を行うことが困難になった場合における対応が問題となる。しかし、国会が長い間存在しないことが国民主権の観点から見て望ましくないことを考えると、上記の期間は厳格に守られるべきであり、議員の任期延長よりも簡易な方法で選挙を行う方法を設ける方法を検討するべきである。また、仮に緊急事態によって40日以内に選挙が行うことができず期間経過後に選挙が行われたとしても、やむを得ない事由があっての遅延であることを踏まえると「事情判決の法理」(行政事件訴訟法31条)の趣旨を読みこんで選挙の効力を認めるといった考え方\footnote{山内敏弘「緊急事態条項導入論をめぐる問題点」龍谷法学51巻3号(2019
  年)87頁。}も可能である。いずれにせよ参議院の緊急集会があるため、国会の機能が停止することはなく、任期の延長を認める必要は無い。

ウ.衆議院任期満了の場合

参議院の緊急集会の規定は衆議院が解散された場合の規定であるから、衆議院任期満了の場合には適用されない。しかし緊急集会の規定が、衆議院が存在しない状況においても国会の活動を維持する趣旨で設けられたことを考慮すると、任期満了の場合にも準用することは可能であると解される。また、衆議院の任期延長は戦前、挙国一致体制を目指すために利用されたように内閣の権限濫用のおそれがあり、憲法上に任期延長の特例を規定することは妥当でない。繰延投票(公職選挙法57条)の実施など、法律レベルでの対応は可能である。

エ.小括

以上より、解散権の制限及び議員の任期等の特例を憲法に規定することは、そもそも各場面において問題となることがなく、法律レベルでの対応が可能であることを踏まえると、必要性がないことのように思われる。

\subsection{濫用の危険について}

ア.99条1項における内閣の権限

緊急事態の宣言が発せられたときに内閣(または内閣総理大臣)が執れる措置として、\UTF{2460}法律と同一の効力を有する政令の制定、\UTF{2461}財政上必要な支出その他の処分、\UTF{2462}地方自治体の長に対する必要な指示がある(改正案99条1項)。

しかし、\UTF{2460}発することのできる政令の内容は無限定であり、上述のように国会の事後の承認による統制には限界があるし、承認が無かった場合の政令の効果が亡くなる旨の記述がない。また、\UTF{2461}財政支出または処分に関しても国会の事後の承認(同条2項)がなかった場合に効力を失う旨の規定はない。これは財政民主主義(憲法83条)との関係で問題となるし、予備費や補正予算での対応も可能である。さらに\UTF{2462}上述のように緊急事態においては中央に権限を集中させるのではなく、現場の実情を知る地方との協力関係を重視し、共同して対処することが望ましい。

イ.制約される人権

改正案99条3項は、「緊急事態の宣言が発せられた場合には、何人も、法律の定めるところにより、当該宣言にかかる事態において国民の生命、身体及び財産を守るために行われる措置に関して発せられる国その他の公の機関の指示に従わなければならない」として国民の公的機関の指示に従う義務を規定している。

この規定は国民に対して国家の非常措置に関する指示に従うことを強制しており、努力義務を定めた国民保護法4条の規定を超えるものであって人権の制約の態様が大きい。

また、同項は、「この場合においても、第14条(法の下の平等)、第18条(奴隷的拘束の禁止)、第19条(思想・良心の自由)、第21条(表現の自由)その他の基本的人権に関する規定は,最大限に尊重されなければならない。」と定めている。

こちらでも保障ではなく、「尊重」という言葉が使用されており、基本的人権が無制限に制約される危険がある。

以上より、改正案が定める緊急事態における内閣または内閣総理大臣がとることのできる非常措置に関する規定は合理性に乏しく、濫用の危険性が高いといえる。

\section{総括}

平時の法制度では対応できないような緊急事態に陥った場合、国家と憲法秩序を守るために国家に国家緊急権を与え、権限を集中させることは仕方の無いことのように思われる。しかしながら、事前に想定して憲法に規定しておける緊急事態には限界があり、国家緊急権を憲法に規定することは政府の積極的な運用による権限濫用の危険がある。改正案における国家緊急権はこれまで見てきたように、国家緊急権を憲法に規定する必要性、あるいはとることのできる非常措置の合理性を有する規定はなく、濫用の危険を一層強め、国民の基本的人権を不当に制約するおそれがある。

緊急事態における不文の国家緊急権を国家に対して認めつつも、想定できる緊急事態に対しては法制度の維持・整備を行い、当該事態が発生した場合に国家の権限を制限し、国家緊急権を適切に行使させるように備えることが、現実的かつ適切な方法であると解される。

\section{参考文献}
\noindent
・芦部信喜『憲法学\ajRoman{1}憲法総論』(有斐閣、1992年)。\\
・清水伸編『逐条日本国憲法審議録第二巻(有斐閥、1962年)。\\
・小林直樹『国家緊急権』(学陽書房、1979年)。\\
・大西芳雄『憲法の基礎理論』(有斐閣、1975年)。\\
・佐藤幸治『日本国憲法論』(成文堂、2011年)。\\
・工藤達朗「国家緊急権と抵抗権」中央ロージャーナル第16巻第1号(2019年)。\\
・日本弁護士会「日本国憲法に緊急事態条項(国家緊急権)を創設することに反対する意見書」(2017
年)。\\
・山内敏弘「緊急事態条項導入論をめぐる問題点」龍谷法学51巻3号(2019年)。\\
・飯島滋明「緊急事態条項の是非について」名古屋学院大学研究年報28
巻(2015 年)
\end{document}